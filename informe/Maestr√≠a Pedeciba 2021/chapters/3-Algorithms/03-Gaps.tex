
% \vspace{-10pt}
% \clearpage
\section{Encoding of Gaps in the Masking Variants}
\label{algo:maskmodes}


\newcommand{\xOneN}{x_1...x_n}
\newcommand{\StringSeq}{x_1...x_{i-1}}
\newcommand{\xiiminus}{p(x_i|\StringSeq)}
\newcommand{\xiiminustwo}{p(\cdot|\StringSeq)}
\vspace{-5pt}
% variant \maskalgo of an algorithm
We recall that the masking variant of an algorithm starts by losslessly encoding the position of all the gaps in the data, independently for each data column (\Line \gapLine\ in Figure \ref{pseudoCodeCommon}). We describe the position of the gaps in a column by encoding a sequence of binary symbols, $\xOneN$, each symbol $x_i$ indicating the presence ($0$) or absence ($1$) of a sample in the $i$-th timestamp of the column, in chronological order. To this end we use an arithmetic coder (AC) \cite{ac2, Cover2005}. Given a sequence of probability assignments, $\xiiminustwo$, for the symbol in position $i$ given the past symbols $x_1...x_{i-1}$, $1\leq i \leq n$, an AC generates a lossless encoding bit stream for $\xOneN$, of length $-\log P(\xOneN) + O(1)$, where $P(\xOneN)=\prod_{i=1}^{n}\xiiminus$. This code length is optimal for this probability assignment, up to an additive constant \cite{arcoding}.


\clearpage


For a sequence $x$ of independent and identically distributed random binary symbols (with unknown probability distribution), the Krichevsky–Trofimov probability assignment \cite{ktestimator}, which we define next, yields an asymptotically optimal code length for the (unknown) probability distribution, in the sense that the worst case redundancy for such code is asymptotically minimized~\cite{unicodinginfo}.


\begin{defcion}
\label{def:ktestimator}
Given a string $x$ over an alphabet $A = \{0, 1\}$, the \textit{Krichevsky–Trofimov (KT) probability assignment} assigns the following probabilities for each symbol position $i, 1\leq i \leq n$
\vspace{-2pt}
\begin{equation}
\label{eq:ktestimator}
p(0|\StringSeq) = \frac{n_0 + 1/2}{i}, \hspace{+20pt} p(1|\StringSeq) = \frac{n_1 + 1/2}{i},
\end{equation}
\end{defcion}
\vspace{-8pt}
where $n_0$ and $n_1$ denote the number of occurrences of 0 and 1 in $\StringSeq$, respectively.


\vspace{+5pt}
Analyzing the experimental datasets presented in Chapter~\ref{datasets}, we notice that the positions of the data gaps follow different patterns for different datasets, but, in general, the gaps occur in bursts. Thus, it makes sense to consider a simple binary Markov model, such as the one defined next, which captures the burstiness of data gap occurrences~\cite{markovBurst}.


\begin{table}[h]
\begin{minipage}{0.62\textwidth}
\setstretch{1.1}
% \vspace{-35pt}
The first-order Markov model has two states, $S_0$ and $S_1$, and we say that $x_i$ occurs in state $S_b$ iff the previous symbol, $x_{i-1}$, equals $b$. We arbitrarily let $S_1$ be the initial state (i.e. the state in which $x_1$ occurs). In Figure~\ref{tikz:markov} we present a diagram for this Markov model. A KT probability assignment for a first-order Markov model is obtained by applying (\ref{eq:ktestimator}) separately for the subsequence of symbols that occur in states $S_0$ and $S_1$. This is implemented by maintaining two pairs of symbol occurrence counters, $n_0$, $n_1$, one pair for each state.

\end{minipage}
\hspace{0.02\textwidth}
\begin{minipage}{0.32\textwidth}
% \vspace{+5pt}
\centering
\begin{tikzpicture}[initial text=initial]
% \node[initial, state] (A) {};
\node[state] (s0) {$S_0$};
\node[initial above, state, right=of s0] (s1) {$S_1$};
\draw[every loop]
        (s0) edge[loop left, auto=left] node {``0''} (s0)
        (s1) edge[loop right, auto=left] node {``1''} (s1)
        (s0) edge[bend left, auto=left] node {``1''} (s1)
        (s1) edge[bend left, auto=left] node {``0''} (s0);
        
\end{tikzpicture}
\captionof{figure}{Markov process diagram.}
\label{tikz:markov}

\end{minipage}
\end{table}

