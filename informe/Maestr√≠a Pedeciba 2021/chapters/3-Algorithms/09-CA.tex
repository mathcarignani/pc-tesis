
\clearpage

\section{Algorithm CA}
\label{algo:ca}


Algorithm \textit{\CAfull} \cite{coder:ca} is a linear model algorithm that supports lossless and near-lossless compression. \WindowParam We implement both variants, \maskalgo and \NOmaskalgo.


Since CA is a linear model algorithm, its encoding process involves the encoding of a sequence of line segments. In Figure~\ref{pseudoCoderCAM} we show the coding routine for variant \maskalgo, in which all the column entries are integer values (the gaps are encoded separately). It consists of a loop that iterates over all column entries, parsing them into an alternating sequence of a single sample point followed by a window of variable size (up to a maximum size $\win$), such that all the sample points in the same window lie within vertical distance \maxerror from the segment whose endpoints correspond to the single sample point and the last point of the window. The routine operates by successively encoding a single sample point (using an auxiliary routine \CAWinStart, shown in Figure~\ref{caWinStart}), and then growing a window of subsequent sample points, adding one point at a time until it is complete, and finally encoding the size and the last point of the window (using an auxiliary routine \CAWinEnd, shown in Figure~\ref{caWinEnd}).




\newcommand{\defineRays}{Let \smin\ and \smax\ be two rays}

\beginAlgorithm
\onlyInput{\columnInput\\\outVarM{CA}\\\cInputThreshold\\\cInputWindowAPCA\\\inputTSCol}
\createWindowOne\\
\defineRays\\
\ForEach{\forEachEntryCoder}{
    \obtainTS\\
    \letPointCA\\
    
    \uIf{\firstIteratCA}{
        \letArchived, then \encodeWinStaCA\\
    }
    \uElseIf{\windowEmptyCA}{
        \letSnapshot, then \addSnapshot\\
        \letSminSmax\\
    }
    \Else{
        Let \EseE\ be the segment with initial point \archived\ that passes through point \incoming\\
        \uIf{\nott\,\validCondCA \orr \windowFull\ }{
            \encodeWinEndCA\\
            \createWindowTwo\\
            \letArchived, then \encodeWinStaCA\\
        }
        \Else{
            \letSnapshot, then \addSnapshot\\
            \letSminSmaxOld\\
            \letSminSmax\\
            \letSminSmaxUpdate\\
        }
    }
}
\If{\winNoEmpty}{
  \encodeWinEndCA\\
}
\EndAlgorithmVariantM{Coding}{\coderCA}{\label{pseudoCoderCAM}}




\clearpage


The algorithm determines when to stop growing the window, i.e., when the current window $\window$ is complete, based upon three sample points: \textit{incoming (\incoming)} is the point corresponding to the column entry for the current iteration; \textit{archived (\archived)} is the point most recently encoded, and it corresponds to a single sample that precedes a window; \textit{snapshot (\snapshot)} is the point most recently added to $\window$. The algorithm maintains a cone, determined by two rays departing from \archived, \smin and \smax, such that \incoming is added to $\window$ as long as it lies within the cone. This cone, in turn, is defined so that if \incoming is indeed added to $\window$ (thus becoming point \snapshot), all the points in $\window$ are at a vertical distance at most \maxerror from the line segment $(\archived, \snapshot)$. Since the encoding process involves the encoding of a sequence of these segments $(\archived, \snapshot)$, the absolute error between the encoded and the original samples is guaranteed to be at most \maxerror.


In figures \ref{example:caIntro1} and \ref{example:caIntro2} we present an example that illustrates the key steps of the coding algorithm. In Figure~\ref{example:caIntro1}, the three points, \incoming, \archived, and \snapshot, as well as the two rays departing from \archived, \smin and \smax, are shown. \archived is defined (and encoded) in the first iteration (\Line 7), while \snapshot and the two rays are defined in the second iteration (\Lines 9-10). \incoming is the incoming point in the third iteration, where line segment $(\archived, \incoming)$ is defined in \Line 12. The slope of this segment is between the slopes of \smin and \smax, i.e. \incoming lies within the cone determined by the two rays, so the condition in \Line 13 evaluates to false (we assume that $|\window|<\win$), i.e. $\window$ is not complete. Thus, \Lines 18-21 are executed: \snapshot is made equal to \incoming, and it is added to $\window$ (\Line 18), and \smin and \smax are updated (\Lines 19-21). The updated information is shown in Figure~\ref{example:caIntro2}. 


\vspace{+5pt}
\exampleCA


\vspace{+5pt}
In the successive iterations, the angle of the cone keeps decreasing, until eventually either point \incoming lies outside of the cone, or $\window$ reaches its maximum size. When this occurs, $\window$ is complete, and the condition in \Line 13 becomes true, so the size and the last point of $\window$, i.e. point \snapshot, are encoded (\Line 14), $\window$ is emptied (\Line 15), and the process starts again, by defining and encoding a new point \archived (\Line 16). We point out that an iteration in which point \archived is updated is always followed by an iteration in which \smin and \smax are updated. Also, notice that point \archived, and rays \smin and \smax, are always initialized in the first two iterations, respectively, so the cone is properly defined every time the condition in \Line 13 is checked.


\vspace{+5pt}
The coding routine for variant \maskalgo invokes two auxiliary routines, \CAWinStart and \CAWinEnd, which are presented in figures \ref{caWinStart} and \ref{caWinEnd}, respectively. Notice that, since the x-coordinates for both point \archived and the last point of $\window$ are already encoded in the timestamp column (recall \Line 4 in Figure~\ref{pseudoCodeCommon}), it suffices to encode their respective y-coordinates. In routine \CAWinEnd, the size of $\window$ must be encoded so that the decoder is able to figure out which timestamp (x-coordinate) corresponds to its last point.


\clearpage




\beginAlgorithm
\onlyInput{\pointInputCA\\\outVarM{CA}}
% \encodeWindowSizeeOne\\
% Let \valuev\ be the y-coordinate of point \archived\\
\encodeSpecific{the y-coordinate of point \archived}\\
\EndAuxVarM{\CAWinStart}{CA}{\label{caWinStart}}



\vspace{-9pt}


\beginAlgorithm
\onlyInput{\windowInput\\\outVarM{CA}\\\cInputWindowAPCA}
\encodeWindowSizee\\
% Let \valuev\ be the y-coordinate of the last point in \window\\
\encodeSpecific{the y-coordinate of the last point in \window}\\
\EndAuxVarM{\CAWinEnd}{CA}{\label{caWinEnd}}



%%%%%%%%%%%%%%%%%%%%%%%%%%%%%%%%%%%%%%%%%%%%%%%%%%%%%%%%%%%%%%%%%%%%%%%%%%%%%%%%%%%%%%

\vspace{-7pt}
The decoding routine for variant \maskalgo is shown in Figure~\ref{pseudoDecoderCA}. It consists of a loop that keeps running until every entry in the column has been decoded, which occurs when condition in \Line~2 becomes false. In each iteration, a point is obtained, decoding its y-coordinate, \valuev, from the binary file, and reading its x-coordinate from the timestamp column (\Lines 3-4). If $\decoAP$ is true, \valuev is the result of an encoding produced by routine \CAWinStart, so it corresponds to the y-coordinate of an archived point (\archived). In this case, point \archived is saved and \valuev is written to the decoded CSV data file (\Lines 6-7). On the other hand, if $\decoAP$ is false, \valuev comes from an encoding produced by routine \CAWinEnd, so it corresponds to the y-coordinate of a snapshot point (\snapshot), whose x-coordinate is obtained from the timestamp column given the decoded window $\sizee$ (\Lines 11-12). In this case, the points in the window are modeled by a line segment, whose endpoints are \archived and \snapshot, so the auxiliary routine \decodeSegment is called with their respective coordinates as inputs (\Line 13). Routine \decodeSegment (recall Figure~\ref{pseudoDecoLinear}) returns a list consisting of the (integer) sample values that are decoded from the segment, which are then written in the decoded CSV data file (\Lines 14-16).


\vspace{-8pt}


\newcommand{\tvaluev}{$t_\valuev$}
\newcommand{\TTrue}{\textnormal{true}}
\newcommand{\FFalse}{\textnormal{false}}

\beginAlgorithm
\onlyInput{\inVarM{CA}\\\doutputfile\\\cInputWindowAPCA\\\cInputColSize\\\inputTSCol}
Let \nEqZero, and $\decoAP = \TTrue$\\
\While{\whileColumnNotDecoded}{
    \decodeSpecific{\valuev}\\
    Obtain timestamp for $\valuev$, \tvaluev, from \tscol\\
    \uIf{$\decoAP$}{
        \letPointCoor{\archived}{\tvaluev}{\valuev}\\
        Output $\valuev$ to \out\\
        Let $\sizee = 1$\\
        Let $\decoAP = \FFalse$\\
    }
    \Else{
        \decodeWindowwSize\\
        \letPointCoor{\snapshot}{\tvaluev}{\valuev}\\
        $\decodedSamplesTwo =  \decodeSegment$(\tscol, \archived.\xx, \snapshot.\xx, \archived.\yy, \snapshot.\yy) // routine in Figure~\ref{pseudoDecoLinear}\\
        \ForEach{\forEachEntryDecoded}{
            Output $\valuev$ to \out
        }
        Let $\decoAP = \TTrue$\\
    }
    \addNToSize\\
}
\EndAlgorithmVariantM{Decoding}{\coderCA}{\label{pseudoDecoderCA}}





%%%%%%%%%%%%%%%%%%%%%%%%%%%%%%%%%%%%%%%%%%%%%%%%%%%%%%%%%%%%%%%%%%%%%%%%%%%%%%%%%%%%
%%%%%%%%%%%%%%%%%%%%%%%%%%%%%%%%%%%%%%%%%%%%%%%%%%%%%%%%%%%%%%%%%%%%%%%%%%%%%%%%%%%%
%%%%%%%%%%%%%%%%%%%%%%%%%%%%%%%%%%%%%%%%%%%%%%%%%%%%%%%%%%%%%%%%%%%%%%%%%%%%%%%%%%%%


\clearpage
\subsection{Example}
\label{algo:ca:example}


\exampleIntro{\ref{example:ca:1}}


This example involves encoding 12 samples, so no window can reach the maximum size (256). Therefore, an incoming point (\incoming) is added to the current window, $\window$, as long as it lies within the cone determined by the two rays departing from the archived point (\archived).


In the first iteration, the condition in \Line 6 is satisfied, so \archived is made equal to \incomingP{1}, and it is encoded via the auxiliary routine \CAWinStart, using $\logWinSize = \logWinSizeOpt{256}=8$ bits for encoding a 1, and \tobitexp for encoding its y-coordinate, also 1. In the second iteration, $\window$ is empty, so the condition in \Line 8 is satisfied. Therefore, the snapshot point (\snapshot) is made equal to \incomingP{2}, it is added to $\window$, and two rays, \smin and \smax, are defined. Figure~\ref{example:ca:1} shows both saved points, \archived and \snapshot, as well as both rays, \smin and \smax, after the second iteration is completed.


\vspace{+3pt}
\exampleStep{ca}{1}{\label{example:ca:1}}{1}


In the third iteration, another sample value equal to 1 is processed, but in this case $\window$ is not empty. In Figure~\ref{example:ca:2}, point \incomingP{3}, as well as segment \EseEP{3}, defined in \Line 11, is shown.


\vspace{+3pt}
\exampleStep{ca}{2}{\label{example:ca:2}}{2}


\clearpage


Since \incomingP{3} lies within the cone determined by the two rays departing from \archived, the condition in \Line 12 evaluates to false. Therefore, \snapshot is made equal to \incomingP{3}, it is added to $\window$, and \smin and \smax are updated (\Lines 18-21). Figure~\ref{example:ca:3} shows the information after the third iteration is completed.


\vspace{+5pt}
\exampleStep{ca}{3}{\label{example:ca:3}}{3}


The following two iterations are similar to the previous one. In each iteration, the respective incoming point is added to the window, and both rays, \smin and \smax, are updated, decreasing the angle of the cone. Figure~\ref{example:ca:4} shows the information after the 5th iteration is completed.


\vspace{+5pt}
\exampleStep{ca}{4}{\label{example:ca:4}}{4}


\clearpage


In the 6th iteration, sample value 2 is processed. In Figure~\ref{example:ca:5}, point \incomingP{6} and segment \EseEP{6} are shown.


\exampleStep{ca}{5}{\label{example:ca:5}}{5}


\incomingP{6} lies within the cone, so the condition in \Line 13 evaluates to false. Therefore, \snapshot is made equal to \incomingP{6}, and it is added to $\window$. In this case, \smin is updated, but \smax remains unchanged. Figure~\ref{example:ca:6} shows the information after the 6th iteration is completed.


\vspace{+5pt}
\exampleStep{ca}{6}{\label{example:ca:6}}{6}


\clearpage

In the 7th iteration, sample value 3 is processed. In Figure~\ref{example:ca:7}, point \incomingP{7} and segment \EseEP{7} are shown.


\exampleStep{ca}{7}{\label{example:ca:7}}{7}


For the first time, an incoming point, \incomingP{7}, lies outside of the cone. In this case, \EseEP{7} has a greater slope than \smax, so the condition in \Line 13 becomes true. Therefore, auxiliary routine \CAWinEnd is invoked, which uses $\logWinSize = 8$ bits for encoding the size (5) of $\window$, and \tobitexp for encoding the y-coordinate (2) of \snapshot, its last point. Next, $\window$ is emptied, \archived is made equal to \incomingP{7}, and it is encoded via the routine \CAWinStart, using $\logWinSize =8$ bits for encoding a 1, and \tobitexp for encoding its y-coordinate (3). In the 8th iteration, $\window$ is empty, so the condition in \Line 7 is satisfied. Thus, \snapshot is made equal to \incomingP{8}, it is added to $\window$, and \smin and \smax are defined once more. Figure~\ref{example:ca:8} shows the information after the 8th iteration is completed.


\vspace{+5pt}
\exampleStep{ca}{8}{\label{example:ca:8}}{8}


\clearpage


In Figure~\ref{example:ca:9} we present the information after the coding routine has finished. After the last iteration of the loop is executed, $\window$ is not empty, so routine \CAWinEnd is invoked in \Line 26. Besides showing the encoded points and their associated line segments, in Figure~\ref{example:ca:9} we also display the values of the decoded samples, which are the values that the decoding routine, shown in Figure~\ref{pseudoDecoderCA}, writes to the decoded CSV data file. 


\vspace{-5pt}
\exampleStep{ca}{9}{\label{example:ca:9}}{9}


%%%%%%%%%%%%%%%%%%%%%%%%%%%%%%%%%%%%%%%%%%%%%%%%%%%%%%%%%%%%%%%%%%%%%%%%%%%%%%%%%%%%
%%%%%%%%%%%%%%%%%%%%%%%%%%%%%%%%%%%%%%%%%%%%%%%%%%%%%%%%%%%%%%%%%%%%%%%%%%%%%%%%%%%%
%%%%%%%%%%%%%%%%%%%%%%%%%%%%%%%%%%%%%%%%%%%%%%%%%%%%%%%%%%%%%%%%%%%%%%%%%%%%%%%%%%%%


\vspace{-22pt}
\subsection{Non-Masking (\NOmaskalgo)\ Variant}
\label{algo:ca:nmvariant}


\vspace{-6pt}
The coding and decoding routines for variant \NOmaskalgo of algorithm CA are similar to their variant \maskalgo counterparts, the difference being that the former routines are able to handle both sample values and gaps. Recall that, in the coding routine for variant \maskalgo, a window is encoded when point \incoming lies outside of the cone determined by the two rays departing from point \archived, or when the maximum window size is reached (\Line 13 in Figure~\ref{pseudoCoderCAM}). The coding routine for variant \NOmaskalgo must also encode a window if the incoming column entry is character \noData\ (gap in the data) and the other entries in the window are sample points, or vice versa. We point out that, in the former case, it doesn't make sense defining point \archived, so \Line 16 is not executed in this scenario. For the same reason, \Line 7 should not be executed if the first column entry is character \noData. Auxiliary routine \CAWinEnd uses the same number of bits for encoding a window that consists of gaps than for encoding a window that consists of sample points, i.e. $\logWinSize+\BeCe$, where $\logWinSize$ bits are used for encoding its size, and $\BeCe$ bits are used for encoding the special integer \nodata. We point out that, in variant \NOmaskalgo, every gap is encoded via auxiliary routine \CAWinEnd; this is true even for a single gap burst appearing between two sample values, in which case the corresponding encoded window size is one.

