
\section{Algorithm Compression Performance}
\label{secX:codersmask}


In this section, we compare the compression performance of the coding algorithm variants presented in Chapter~\ref{algo}, by encoding the various datasets introduced in Chapter~\ref{datasets}. We begin by comparing the variants with each other, and in Subsection~\ref{secX:gzip} we compare them with gzip \cite{gzip}, a popular general-purpose lossless compression algorithm. In Subsection~\ref{secX:timestampComp} we study the lossless compression of the timestamp column. 


We analyze the performance of the algorithm variants on complete datasets (not individual files), so we always apply definitions~\ref{eq:coding-size-dataset}--\ref{def:relative-difference-dataset}. Following the results obtained in sections~\ref{secX:rendimiento-relativo} and~\ref{secX:windows}, we only consider the masking variants of the evaluated algorithms (i.e. set $V^{*}$), and we always set the window size parameter to the \owsns\ (recall Definition~\ref{def:ows}).


For each data type $z$ of each dataset $d$, and each coding algorithm variant $a_v \in V^{*}$ and error parameter $e \in E$, we calculate the CR of $c_{<a_v, w_{global}^{*}, e>}$, as defined in (\ref{eq:compression-rate-dataset}), where \WGlobal\~$=\owsns(a_v, e, z, d)$. The following definition is useful for analyzing which CAI obtains the best compression result for a specific data type.


\newcommand{\tasaCompTwo}{\CR(c_{<a_v, w_{global}^{*}, e>}, z, d)}
\begin{defcion}
\label{def:bestcai}
Let $z$ be a data type of a certain dataset $d$, and let $e \in E$ be an error parameter. We denote by $c^{b}(z, d, e)$ the \textit{best CAI} for $z, d, e$, and define it as the CAI that minimizes the CR among all the CAIs in CI, i.e.,
\vspace{-3pt}
\begin{equation}
\label{eq:eqbestcai}
c^{b}(z, d, e) = \argmin_{\algo \ \in \ca} \biggl\{ \tasaCompTwo \biggr\}.
\end{equation}
When $c^{b}(z, d, e) = c_{<a^{b}_v, w_{global}^{b*}, e>}$, we refer to $a^{b}_v$ and $w_{global}^{b*}$ as the \textit{best coding variant} and the \textit{best window size} for $z, d, e$, respectively.
\end{defcion}


Our experiments include a total of 21 data types, in 8 different datasets (recall this information from Table~\ref{datasets:table:overview}). As an example, in Figure~\ref{fig:algo-per-1} we show the $\tasaCompTwo$ and the window size parameter \WGlobal, as a function of the error parameter $e$, obtained for each algorithm $a_v \in V^{*}$, for the data type $z=$ ``Sea Temp." of the dataset $d=$ \datasetelnino, presented in Section~\ref{datasets:elnino}. For each error parameter $e \in E$, we use blue circles to highlight the markers for the minimum CR value and the best window size (in the respective plots corresponding to the best coding variant). For instance, for $e=0$, the best CAI achieves a CR equal to $0.33$ using algorithm PCA with a window size of 256. Thus, in this case, algorithm PCA is the best coding variant, and 256 is the best window size. For the remaining 7 values of the error parameter, the blue circles indicate that in every case the best algorithm is APCA, and the best window size ranges from 4 up to 32.


\clearpage


\threefoursingle{5-ElNino-7}{\datasetelnino}{Sea Temp.}{\ For each error parameter $e \in E$, we use blue circles to highlight the markers for the minimum CR value and the best window size (in the respective plots corresponding to the best coding variant)}{\label{fig:algo-per-1}}


\clearpage


Table~\ref{experiments:mask-results-overview1} summarizes the compression performance results obtained by the evaluated coding algorithms, for each data type of each dataset. Each row contains information relative to certain data type. For example, the 13th row shows summarized results for the data type ``Sea Temp." of the dataset \datasetelnino, which are presented in more detail in Figure~\ref{fig:algo-per-1}. For each error parameter value, the first column shows the CR obtained by the best CAI, the second column shows the base-2 logarithm of its window size parameter, and the cell color identifies the best algorithm.
\vspace{+5pt}


\newcommand{\legendsone}{
\begin{tabular}{| C{1.5cm} | C{1.5cm} | C{1.5cm} | C{1.5cm} |}
\hline
  \multicolumn{1}{|>{\centering\arraybackslash}m{1.5cm}|}{\cpca PCA} 
& \multicolumn{1}{>{\centering\arraybackslash}m{1.5cm}|}{\capca APCA} 
& \multicolumn{1}{>{\centering\arraybackslash}m{1.5cm}|}{\cca CA} 
& \multicolumn{1}{>{\centering\arraybackslash}m{1.5cm}|}{\cfr FR}\\
\toprule[0.1mm]
\end{tabular}
\vspace{+10pt}

}

\newcommand{\legendstwo}{
\begin{tabular}{| C{1.5cm} | C{1.5cm} | C{1.5cm} | C{1.5cm} |}
% \begin{tabular}{| C{1.5cm} | C{1.5cm} | C{1.5cm} | C{1.5cm} | C{1.5cm} |}
\hline
  \multicolumn{1}{|>{\centering\arraybackslash}m{1.5cm}|}{\cgzip GZIP} 
& \multicolumn{1}{>{\centering\arraybackslash}m{1.5cm}|}{\cpca PCA} 
& \multicolumn{1}{>{\centering\arraybackslash}m{1.5cm}|}{\capca APCA} 
% & \multicolumn{1}{>{\centering\arraybackslash}m{1.5cm}|}{\cca CA} 
& \multicolumn{1}{>{\centering\arraybackslash}m{1.5cm}|}{\cfr FR}\\
\toprule[0.1mm]
\end{tabular}
\vspace{+10pt}

}

\newcommand{\FirstSentence}{Compression performance of the best evaluated coding algorithm, for various error values on each data type of each dataset}
\newcommand{\SecondSentence}{Each row contains information relative to certain data type. For each error parameter value, the first column shows the minimum CR, and the second column shows the base-2 logarithm of the best window size for the best algorithm (the one that achieves the minimum CR), which is identified by a certain cell color described in the legend above the table}

\newcommand{\captionzero}{Lossless compression performance of algorithm variants \NonMaskVar{PCA} and \NonMaskVar{APCA}, for the timestamp column of each dataset. For each variant, the first column shows the minimum CR, and the second column shows the base-2 logarithm of the best window size (the window size that achieves the minimum CR). For each dataset, the cells corresponding to the algorithm that obtains the best result are colored, and the last column shows the RD between the corresponding CAIs, as defined in (\ref{eq:relative-difference-dataset}).
}

\newcommand{\captionone}{
\FirstSentence. \SecondSentence.
}

\newcommand{\captiontwo}{
\FirstSentence, including the results obtained by gzip. Each row contains information relative to certain data type. \SecondSentence. Algorithm gzip doesn't have a window size parameter, so the cell is left blank in these cases.
}

\newcommand{\captionminmaxone}{
$\text{\maxRD}(a, e)$ obtained for every pair of coding algorithm variant $a_v \in V^*$ and error parameter $e \in E$. For each $e$, the cell corresponding to the $\text{\minmaxRD}(a)$ value is highlighted.  
}

\newcommand{\captionminmaxtwo}{
$\text{\maxRD}(a, e)$ obtained for every pair of coding algorithm variant $a_v \in V^* \cup \{\text{gzip}\}$ and error parameter $e \in E$. For each $e$, the cell corresponding to the $\text{\minmaxRD}(a)$ value is highlighted.  
}


\begin{sidewaystable}[ht]
\newcommand{\cpca}{\cellcolor{cyan!20}}
\newcommand{\capca}{\cellcolor{green!20}}
\newcommand{\cfr}{\cellcolor{yellow!25}}
\newcommand{\cgzip}{\cellcolor{orange!20}}
\newcommand{\cpwlhint}{\cellcolor{violet!25}}
\newcommand{\cpwlh}{\cellcolor{violet!50}}
\newcommand{\cca}{\cellcolor{brown!20}}
\centering
\legendsone
\begin{tabular}{| l | l | c | c || c | c || c | c || c | c || c | c || c | c || c | c || c | c |}
\cline{3-18}
\multicolumn{1}{c}{}& \multicolumn{1}{c|}{} & \multicolumn{2}{c||}{e = 0} & \multicolumn{2}{c||}{e = 1} & \multicolumn{2}{c||}{e = 3} & \multicolumn{2}{c||}{e = 5} & \multicolumn{2}{c||}{e = 10} & \multicolumn{2}{c||}{e = 15} & \multicolumn{2}{c||}{e = 20} & \multicolumn{2}{c|}{e = 30} \\\hline
{Dataset} & {Data Type} & {\footnotesize CR} & {\footnotesize w} & {\footnotesize CR} & {\footnotesize w} & {\footnotesize CR} & {\footnotesize w} & {\footnotesize CR} & {\footnotesize w} & {\footnotesize CR} & {\footnotesize w} & {\footnotesize CR} & {\footnotesize w} & {\footnotesize CR} & {\footnotesize w} & {\footnotesize CR} & {\footnotesize w} \\\hline\hline



We observe that there are four algorithms (PCA, APCA, CA, and FR) that obtain the best compression results in at least one of the 168 possible data type and error parameter combinations. Algorithm APCA is the best algorithm in exactly 132 combinations ($79\%$), including every case in which $e \geq 10$, and most of the cases in which $e \in [1, 3, 5]$. PCA is the best algorithm in 31 combinations ($18\%$), including most of the lossless cases. CA and FR are the best algorithms in only 2 (1\%) and 3 (2\%) combinations, respectively, in each case for a single data type.


Since there is not a single algorithm that obtains the best compression performance for every data type, it is useful to analyze how much is the RD between the best algorithm and the rest, for every experimental combination. With that in mind, next we define a pair of metrics.


\clearpage


\begin{defcion}
\label{eq:maxRD}
The \textit{maximum RD (\maxRDit)} of a coding algorithm $a\in A$ for certain error parameter $e\in E$ is given by
\vspace{-4pt}
\begin{equation}
\maxRD(a, e) = \maxi_{z, d} \ \biggl\{ \RD(c^{b}(z,d,e), c_{<a_v, w_{global}^{*}, e>}) \biggr\},
\end{equation}
where the maximum is taken over all the combinations of data type $z$ and dataset $d$, and we recall that $c^{b}(z,d,e)$ is the best CAI for $z, d, e$.
\end{defcion}


The \maxRD\ metric is useful for assessing the compression performance of a coding algorithm $a$ on the set of data types as a whole. Notice that \maxRD\ is always non-negative. A satisfactory result (i.e. close to zero) can only be obtained when $a$ achieves a good compression performance \textit{for every data type}. In other words, bad compression performance even \textit{on a single data type} yields a poor result for the \maxRD\ metric altogether. When \maxRD\ is equal to zero, $a$ achieves the best compression performance for every combination. Analyzing the results in Table~\ref{experiments:mask-results-overview1}, we observe that $\text{\maxRD}(\text{APCA}, e)=0$ for every $e \geq 10$. Since the best algorithm is unique for every combination (i.e. exactly one algorithm obtains the minimum CR in every case), it is also true that, when $a \neq \text{APCA}$, $\text{\maxRD}(a, e)>0$ for every $e \geq 10$.


\vspace{+10pt}
\begin{defcion}
\label{eq:minmaxRD}
The \textit{minmax RD (\minmaxRDit)} for certain error parameter $e\in E$ is given~by
\vspace{-4pt}
\begin{equation}
\minmaxRD(e) = \mini_{a \in A} \ \biggl\{ \maxRDit(a, e) \biggr\},
\end{equation}
and we refer to $\argmin_{a \in A}$ as the \textit{\minmaxca} for $e$.
\end{defcion}


\vspace{-3pt}
Again, \minmaxRD\ is always always non-negative. Notice that $\text{\minmaxRD}(e)=0$ for certain $e$, if and only if there exists a \minmaxca\ $a$ such that $\text{\maxRD}(a, e)=0$. Continuing the analysis from the previous paragraph, it should be clear that APCA is the \minmaxca\ for every $e \geq 10$, since $\text{\maxRD}(\text{APCA}, e)=0$ for every $e \geq 10$.


Table~\ref{experiments:minmaxone} shows the $\text{\maxRD}(a, e)$ obtained for every pair of coding algorithm variant $a_v \in V^*$ and error parameter $e \in E$. For each $e$, the cell corresponding to the $\text{\minmaxRD}(e)$ value (i.e. the minimum value in the column) is highlighted.
\vspace{+3pt}


\begin{table}[h]
\newcommand{\cpca}{\cellcolor{cyan!20}}
\newcommand{\capca}{\cellcolor{green!20}}
\newcommand{\cfr}{\cellcolor{yellow!25}}
\newcommand{\cgzip}{\cellcolor{orange!20}}
\newcommand{\best}{\cellcolor{gray!30}}
\centering\hspace*{0cm}\begin{tabular}{| l | c | c | c | c | c | c | c | c |}\cline{2-9}\multicolumn{1}{c|}{}& \multicolumn{8}{c|}{maxRD (\%)}\\\hline
{Algorithm} & {e = 0} & {e = 1} & {e = 3} & {e = 5} & {e = 10} & {e = 15} & {e = 20} & {e = 30} \\\hline
{PCA\cpca} & {40.51} & {42.27} & {53.11} & {62.01} & {71.71} & {75.28} & {77.15} & {80.21} \\\hline
{APCA\capca} & {33.25} & {\best15.64} & {\best9.00} & {\best29.96} & {\best0} & {\best0} & {\best0} & {\best0} \\\hline
{CA} & {38.28} & {38.28} & {54.68} & {63.12} & {65.44} & {72.94} & {77.21} & {81.84} \\\hline
{PWLH} & {73.46} & {72.93} & {72.52} & {82.14} & {83.24} & {86.86} & {88.94} & {91.19} \\\hline
{PWLHInt} & {\best29.72} & {34.00} & {49.94} & {68.95} & {76.68} & {69.96} & {74.72} & {79.89} \\\hline
{FR\cfr} & {48.75} & {49.85} & {52.21} & {52.70} & {54.82} & {55.35} & {54.48} & {64.72} \\\hline
{SF} & {85.84} & {85.77} & {85.26} & {85.17} & {84.36} & {83.64} & {83.17} & {82.88} \\\hline
{GAMPS} & {49.62} & {57.46} & {79.36} & {86.57} & {92.99} & {95.23} & {96.38} & {97.39} \\\hline
\end{tabular}
\caption{\captionminmaxone}
\label{experiments:minmaxone}
\end{table}



% \vspace{+3pt}
In the lossless case, CA is the minmax coding algorithm, with \minmaxRD \ being equal to $29.55\%$. This value is rather high, which means that none of the considered algorithms achieves a CR that is close to the minimum simultaneously \textit{for every data type}. 
% Algorithms APCA ($33.25\%$) and CA ($38.28\%$) obtain the second and third best \maxRD\ values, respectively. 
Recalling the results from Table~\ref{experiments:mask-results-overview1} we notice that $e=0$ is the only error parameter value for which the minmax coding algorithm doesn't obtain the minimum CR in any combination. In other words, when $e=0$, CA is the algorithm that minimizes the RD with the best algorithm among every data type, even though it itself is not the best algorithm for any data type.


When $e \in [1, 3, 5]$, the minmax coding algorithm is always APCA, and the \minmaxRD \ values are $15.64\%$, $9.00\%$ and $29.96\%$, respectively. Again, these values are fairly high, so we would select the most convenient algorithm depending on the data type we want to compress. 
%The second best \maxRD\ values are obtained by algorithms PWLHInt ($34.00\%$ and $49.94\%$) and FR ($52.70\%$), respectively. 
Notice that in the closest case (algorithm FR for $e=5$), the second best \maxRD\ ($52.70\%$) is about $75\%$ larger than the \minmaxRD, which is a much bigger difference than in the lossless case.


When $e \geq 10$, the minmax coding algorithm is also always APCA, but in these cases the \minmaxRD \ values are always 0. In the closest case (algorithm FR for $e=20$) the second best \maxRD\ is $54.48\%$. If we wanted to compress any data type with any of these error parameter values, we would pick algorithm APCA, since according to our experimental results, it always obtains the best compression results with a significant difference over the remaining algorithms.

