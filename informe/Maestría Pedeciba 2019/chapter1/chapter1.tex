% Chapter 1

\chapter{Título del capítulo} % Main chapter title

\label{Capitulo1} % For referencing the chapter elsewhere, use \ref{Chapter1} 

\lhead{Capítulo 1. \emph{Título del capítulo}} % This is for the header on each page - perhaps a shortened title

\newcommand{\maskalgo}{\textit{M}}
\newcommand{\NOmaskalgo}{\textit{NM}}
\newcommand{\difrelativa}{\textit{DiferenciaRelativa}}


\clearpage


\begin{table}[h]
\vspace{+5pt}
\begin{center}
    \begin{tabular}{| C{2cm} || C{2.5cm} | C{1.3cm} |  C{1.3cm} |  C{5.5cm} |}
    \hline
      \multicolumn{1}{|>{\centering\arraybackslash}m{2cm}||}{\textbf{Dataset}}
    & \multicolumn{1}{>{\centering\arraybackslash}m{2.5cm}|}{\textbf{Dataset Characteristic}} 
    & \multicolumn{1}{>{\centering\arraybackslash}m{1.3cm}|}{\textbf{\#Files}} 
    & \multicolumn{1}{>{\centering\arraybackslash}m{1.3cm}|}{\textbf{\#Types}}
    & \multicolumn{1}{>{\centering\arraybackslash}m{5.5cm}|}{\textbf{Data Types}}\\
    \hline
    \datasetirkis\ \cite{dataset:irkis, dataset:irkis2}   & Many gaps     & 7  & 1 & VWC \\\hline
    \datasetsst\ \cite{dataset:sst1}      & Many gaps     & 3  & 1 & SST \\\hline
    \datasetadcp\ \cite{dataset:sst1}     & Many gaps     & 3  & 1 & Vel \\\hline
    \datasetelnino\ \cite{dataset:elnino} & Many gaps     & 1  & 7 & \datasetelninocols \\\hline
    \datasetsolar\ \cite{dataset:solar}   & Few gaps      & 4  & 3 & \datasetsolarcols \\\hline
    \datasethail\ \cite{dataset:spc}      & No gaps       & 1  & 3 & \datasethailcols \\\hline
    \datasettornado\ \cite{dataset:spc}   & No gaps       & 1  & 2 & \datasettornadocols \\\hline
    \datasetwind\ \cite{dataset:spc}      & No gaps       & 1  & 3 & \datasetwindcols \\\hline
    \toprule[0.1mm]
    \end{tabular}
    \caption{Datasets overview. The second column indicates the characteristic of each dataset, in terms of the amount of gaps. The third column shows the number of files. The fourth and fifth columns show the number of data types and their names, respectively.}
    \label{datasets:table:overview}
\end{center}
\end{table}



En los experimentos realizados se codificaron los datasets combinando estos cuatro parámetros:
\vspace{-8pt}
\begin{itemize}
    \item 21 tipos de dato: ver tabla (\ref{tabla:resumen-de-los-datasets})
    \item 13 codificadores: 
        \begin{itemize}
            \item CoderBase
            \item CoderPCA-NM y CoderPCA-M
            \item CoderAPCA-NM y CoderAPCA-M
            \item CoderCA-NM y CoderCA-M
            \item CoderPWLH-NM y CoderPWLH-M
            \item CoderPWLHInt-NM y CoderPWLHInt-M
            \item CoderGampsLimit-NM y CoderGampsLimit-M
        \end{itemize}
    \item 8 umbrales de error: 0 (sin pérdida), y 1, 3, 5, 10, 15, 20 y 30 (con pérdida)
    \item 7 tamaños de ventana: 4, 8, 16, 32, 64, 128 y 256.
\end{itemize}

Algunas consideraciones a tener en cuenta:
\vspace{-8pt}
\begin{itemize}
    \item El codificador CoderBase solamente codifica sin pérdida e ignora el parámetro del tamaño de ventana. 
    \item Para los codificadores CoderPCA-NM y CoderPCA-M el tamaño de ventana es fijo, mientras que en el resto de los algoritmos (salvo CoderBase) el tamaño de ventana es variable y el parámetro indica su tamaño máximo.
\end{itemize}

\clearpage

Para comparar el rendimiento relativo de los algoritmos con ($\maskalgo$) y sin ($\NOmaskalgo$) máscara, utilizamos la siguiente ecuación
\vspace{-8pt}
\newcommand{\nmbits}{\NOmaskalgo_{\textit{S}}}
\newcommand{\mbits}{\maskalgo_\textit{S}}
\begin{equation}
\label{eq:diferencia-relativa}
%
 \difrelativa(\mbits, \nmbits)  =
  \begin{cases}
   100\times\frac{\nmbits - \mbits}{ \nmbits }, \quad & \text{si } \nmbits \ne \mbits, \\
   0,                   & \text{si } \nmbits = \mbits,
  \end{cases}
%  
\end{equation}
donde $\mbits$ y $\nmbits$ son los tamaños de los archivos codificados con los respectivos algoritmos. El algoritmo $\maskalgo$ logra una mejor tasa de compresión que el algoritmo $\NOmaskalgo$ cuando el resultado de la ecuación \ref{eq:diferencia-relativa} es mayor a cero. Mientras mayor sea dicho valor, mejor será el rendimiento relativo del algoritmo $\maskalgo$ respecto al algoritmo $\NOmaskalgo$.

[Se considera el dataset de manera global, tomando la ventana óptima global por algoritmo en cada caso. NOTA: para un umbral de error y modo fijo, la ventana óptima global no tiene por qué ser la misma para todos los algoritmos.]

En la tabla \ref{tabla:rendimiento-relativ-NM-M} se muestra un resumen de los resultados obtenidos al comparar el rendimiento relativo de los algoritmos $\NOmaskalgo$ y $\maskalgo$ para cada cada dataset. En la tercera y cuarta columnas aparece el porcentaje de las combinaciones <tipo de dato, codificador, umbral> con las que se obtiene la mejor tasa con cada algoritmo. En la última columna se muestra el rango en el que varía el resultado de la ecuación $\difrelativa$ para dichas combinaciones.

En los datasets que tienen muchos gaps siempre se obtienen mejores tasas al utilizar los algoritmos $\maskalgo$. En cambio, en los datasets sin gaps siempre se tiene mejor rendimiento con los algoritmos $\NOmaskalgo$. En el dataset con pocos gaps, en cada mitad de las combinaciones se obtienen mejores tasas con algoritmos diferentes.\\

\vspace{-5pt}

\begin{table}[h]
\begin{center}
    \begin{tabular}{| C{2.2cm} || C{2.5cm} | C{4.4cm} | C{3.0cm} |}
    \hline
      \multicolumn{1}{|>{\centering\arraybackslash}m{2.2cm}||}{\textbf{Dataset}} 
    & \multicolumn{1}{>{\centering\arraybackslash}m{2.5cm}|}{\textbf{Dataset Characterstic}} 
    & \multicolumn{1}{>{\centering\arraybackslash}m{4.4cm}|}{\textbf{Cases where $a_\maskalgo$ outperforms $a_\NOmaskalgo$ (\%)}}
    & \multicolumn{1}{>{\centering\arraybackslash}m{3.0cm}|}{\textbf{RD (\%) Range}}\\
    \hline
    \datasetirkis   & Many gaps     & 48/48 (100\%) & (0; 36.88]                    \\\hline
    \datasetsst     & Many gaps     & 48/48 (100\%) & (0; \textcolor{red}{50.60}]  \\\hline
    \datasetadcp    & Many gaps     & 48/48 (100\%) & (0; 17.35]                    \\\hline
    \datasetelnino  & Many gaps     & 336/336 (100\%) & (0; 50.52]                    \\\hline
    \datasetsolar   & Few gaps      & 73/144 (50.7\%) & [-0.25; 1.77]                 \\\hline
    \datasethail    & No gaps       & 0/144 (0\%)   & [-0.04; 0)                    \\\hline
    \datasettornado & No gaps       & 0/96 (0\%)   & [\textcolor{blue}{-0.29}; 0)   \\\hline
    \datasetwind    & No gaps       & 0/144 (0\%)   & [-0.12; 0)                    \\\hline
    \toprule[0.1mm]
    \end{tabular}
    \caption{Relative difference between the masking and non-masking variants of each algorithm. The results are aggregated by dataset. In the last column we highlight\\the maximum (red) and minimum (blue) values taken by the RD.}
    \label{tabla:rendimiento-relativ-NM-M}
\end{center}
\end{table}

\vspace{-10pt}

Observamos que, en los casos en los que se obtienen mejores tasas con el algoritmo $\NOmaskalgo$, la diferencia relativa siempre está cerca de 0. En la figura \ref{fig:diff-tornado} vemos que la mejor diferencia relativa a favor de $\NOmaskalgo$ se obtiene para el tipo de dato ``Longitude" del dataset NOAA-SPC-tornado, con el codificador CoderAPCA-$\NOmaskalgo$ y umbral de error 30\%. Como se observa en la tabla \ref{tabla:rendimiento-relativ-NM-M}, dicho valor es~$-0,29$.

Por otro lado, cuando se logran mejores tasas con el algoritmo $\maskalgo$ las diferencias relativas son mucho mayores, alcanzando un máximo de 50,60 para el tipo de dato ``VWC" del dataset NOAA-SST. En la figura \ref{fig:diff-sst} vemos que dicho resultado se obtiene con el codificador CoderPCA-$\maskalgo$ y umbral de error 30\%.

Teniendo en cuenta los resultados presentados, si quisiéramos codificar un dataset que a priori supiéramos tiene muchos gaps, obviamente nos convendría utilizar el algoritmo $\maskalgo$. Pero aún si el dataset no tuviera gaps, la diferencia de rendimiento a favor del algoritmo $\NOmaskalgo$ sería despreciable. Como el algoritmo $\maskalgo$ es más robusto y funciona mejor en general, en las próximas secciones nos vamos a enfocar en su estudio.

\begin{figure}
\hspace{-35pt}
\includegraphics[scale=0.35]{chapter1/7-NOAA-SPC-tornado-2.png}
\hspace{+10pt}
\caption{Tasa de compresión y Diferencia relativa\\para las distintas combinaciones <codificador, umbral de error>\\para el tipo de dato ``Longitude" del dataset Tornado.}
\label{fig:diff-tornado}
\end{figure}

\begin{figure}
\hspace{-35pt}
\includegraphics[scale=0.35]{chapter1/Global-2-NOAA-SST.png}
\hspace{+10pt}
\caption{Tasa de compresión y Diferencia relativa\\para las distintas combinaciones <codificador, umbral>\\para el dataset SST.}
\label{fig:diff-sst}
\end{figure}


