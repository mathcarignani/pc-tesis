
\vspace{-15pt}
\section{Summary}
\label{datasets:summary}


\vspace{-5pt}
In Table~\ref{datasets:table:overview} we summarize information of the eight datasets presented in the previous sections of this chapter. The second column indicates, qualitatively, the characteristic of each dataset, in terms of the number of gaps. The third column shows the number of CSV files corresponding to each dataset. The last two columns show the number of data types in each dataset, and their names, respectively. In many files there are multiple data columns with the same data type.




\begin{table}[h]
\vspace{+5pt}
\begin{center}
    \begin{tabular}{| C{2cm} || C{2.6cm} | C{1.25cm} |  C{1.5cm} |  C{5.45cm} |}
    \hline
      \multicolumn{1}{|>{\centering\arraybackslash}m{2cm}||}{\textbf{Dataset}}
    & \multicolumn{1}{>{\centering\arraybackslash}m{2.6cm}|}{\textbf{Characteristics}} 
    & \multicolumn{1}{>{\centering\arraybackslash}m{1.25cm}|}{\textbf{\#Files}} 
    & \multicolumn{1}{>{\centering\arraybackslash}m{1.25cm}|}{\textbf{\#Types}}
    & \multicolumn{1}{>{\centering\arraybackslash}m{5.45cm}|}{\textbf{Data Types}}\\
    \hline
    \datasetirkis\ \cite{dataset:irkis}   & Many gaps     & 7  & 1 & VWC \\\hline
    \datasetsst\ \cite{dataset:sst1}      & Many gaps     & 3  & 1 & SST \\\hline
    \datasetadcp\ \cite{dataset:sst1}     & Many gaps     & 3  & 1 & Vel \\\hline
    \datasetelnino\ \cite{dataset:elnino} & Many gaps     & 1  & 7 & \datasetelninocols \\\hline
    \datasetsolar\ \cite{dataset:solar}   & Few gaps      & 4  & 3 & \datasetsolarcols \\\hline
    \datasethail\ \cite{dataset:spc}      & No gaps       & 1  & 3 & \datasethailcols \\\hline
    \datasettornado\ \cite{dataset:spc}   & No gaps       & 1  & 2 & \datasettornadocols \\\hline
    \datasetwind\ \cite{dataset:spc}      & No gaps       & 1  & 3 & \datasetwindcols \\\hline
    \toprule[0.1mm]
    \end{tabular}
    \caption{Datasets overview. 
    % The second column indicates the characteristic of each dataset, in terms of the number of gaps. The third column shows the number of files. The fourth and fifth columns show the number of data types and their names, respectively.
    }
    \label{datasets:table:overview}
\end{center}
\end{table}




\clearpage



In Table~\ref{datasets:table:scale} we show the measuring unit, scale, and range of each data type. Recall that this information is included in the CSV file for each dataset. We also show, in the last column, the value of the column-dependent parameter $\BeCe$, obtained for the data columns of each data type. It represents the number of bits required to represent either an integer sample value or a data gap indicator in the respective data column. We will formally define and discuss $\BeCe$ in Section~\ref{algo:details}. We point out that data types with the same name can have different measuring unit, scale, and/or range, among different datasets. 

\vspace{+5pt}


\begin{table}[h]
\vspace{+5pt}
\begin{center}
    \begin{tabular}{| L{1.4cm} || L{2cm} | C{2.4cm} |  C{1.1cm} |  C{1.7cm} |  C{1.7cm} | C{1.2cm} |}
    \hline
      \multicolumn{1}{|>{\centering\arraybackslash}m{1.4cm}||}{\textbf{Dataset}}
    & \multicolumn{1}{>{\centering\arraybackslash}m{2cm}|}{\textbf{Data Type}} 
    & \multicolumn{1}{>{\centering\arraybackslash}m{2.4cm}|}{\textbf{Unit}} 
    & \multicolumn{1}{>{\centering\arraybackslash}m{1.1cm}|}{\textbf{Scale}}
    & \multicolumn{1}{>{\centering\arraybackslash}m{1.7cm}|}{\textbf{Minimum}}
    & \multicolumn{1}{>{\centering\arraybackslash}m{1.7cm}|}{\textbf{Maximum}}
    & \multicolumn{1}{>{\centering\arraybackslash}m{1.2cm}|}{\textbf{$\BeCe$ (bits)}}\\
    \hline
    \datasetirkis    & Time Delta  & minutes         & 1     & 0       & 131,071 & 17 \\\hline
                     & VWC         & dimensionless   & 1,000 & 0       & 600     & 10 \\\hline
    \datasetsst      & Time Delta  & seconds         & 1     & 0       & 131,071 & 17 \\\hline
                     & SST         & °C              & 1,000 & 0       & 40,000  & 16 \\\hline
    \datasetadcp     & Time Delta  & minutes         & 1     & 0       & 131,071 & 17 \\\hline
                     & Vel         & m/s             & 1,000 & -1,100  & 2,700   & 12 \\\hline
    \datasetelnino   & Time Delta  & hours           & 1     & 0       & 131,071 & 17 \\\hline
                     & Lat         & coord. degrees  & 100   & -1,000  & 1,000   & 11 \\\hline
                     & Long        & coord. degrees  & 100   & -18,000 & 18,000  & 16 \\\hline
                     & Zon. Wind   & m/s             & 10    & -150    & 150     &  9 \\\hline
                     & Mer. Wind   & m/s             & 10    & -150    & 150     &  9 \\\hline
                     & Humidity    & \%              & 10    & 0       & 1,000   & 10 \\\hline
                     & Air Temp.   & °C              & 100   & 0       & 4,000   & 12 \\\hline
                     & Sea Temp.   & °C              & 100   & 0       & 4,000   & 12 \\\hline
    \datasetsolar    & Time Delta  & minutes         & 1     & 0       & 131,071 & 17 \\\hline
                     & GHI         & \unitSolar      & 1     & 0       & 1,020   & 10 \\\hline
                     & DNI         & \unitSolar      & 1     & 0       & 970     & 10 \\\hline
                     & DHI         & \unitSolar      & 1     & 0       & 800     & 10 \\\hline
    \datasethail     & Time Delta  & minutes         & 1     & 0       & 131,071 & 17 \\\hline
                     & Lat         & coord. degrees  & 100   & 2,500   & 5,000   & 12 \\\hline
                     & Long        & coord. degrees  & 100   & -12,500 & -6,700  & 13 \\\hline
                     & Size        & 1/100 inch      & 1     & 0       & 700     & 10 \\\hline
    \datasettornado  & Time Delta  & minutes         & 1     & 0       & 131,071 & 17 \\\hline
                     & Lat         & coord. degrees  & 100   & 2,400   & 5,000   & 12 \\\hline
                     & Long        & coord. degrees  & 100   & -12,500 & -6,800  & 13 \\\hline
    \datasetwind     & Time Delta  & minutes         & 1     & 0       & 131,071 & 17 \\\hline
                     & Lat         & coord. degrees  & 100   & 0       & 5,000   & 13 \\\hline
                     & Long        & coord. degrees  & 100   & -12,500 & 0       & 14 \\\hline
                     & Speed       & mph             & 10    & 0       & 2,400   & 12 \\\hline
    \toprule[0.1mm]
    \end{tabular}
    \caption{Data types overview.}
    \label{datasets:table:scale}
\end{center}
\end{table}



