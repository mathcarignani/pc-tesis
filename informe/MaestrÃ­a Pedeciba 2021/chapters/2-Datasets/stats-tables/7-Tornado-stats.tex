
\begin{table}[h]
\vspace{+5pt}
\begin{center}
    \begin{tabular}{| L{2cm} || C{2.2cm} | C{1.2cm} | C{1.2cm} | C{1.2cm} | C{1.4cm} | C{1.4cm} |}
    \hline
      \multicolumn{1}{|>{\centering\arraybackslash}m{2cm}||}{\textbf{Data Type}}
    & \multicolumn{1}{>{\centering\arraybackslash}m{2.2cm}|}{\textbf{Gaps (\%)}}
    & \multicolumn{1}{>{\centering\arraybackslash}m{1.2cm}|}{\textbf{Min}}
    & \multicolumn{1}{>{\centering\arraybackslash}m{1.2cm}|}{\textbf{Max}}
    & \multicolumn{1}{>{\centering\arraybackslash}m{1.2cm}|}{\textbf{Mdn}}
    & \multicolumn{1}{>{\centering\arraybackslash}m{1.4cm}|}{\textbf{Mean}}
    & \multicolumn{1}{>{\centering\arraybackslash}m{1.4cm}|}{\textbf{SD}}\\
    \hline
%% SCRIPT OUTPUT BELOW HERE
Lat & 0 (0) & \ \ \ 2,456 & \ \ 4,891 & \ \ 3,688 & \ \ 3,693.9 & 491.1 \\\hline
Long & 0 (0) & \ -12,397 & \ -6,822 & \ -9,313 & \ -9,260.0 & 787.9 \\\hline
%% SCRIPT OUTPUT ABOVE HERE
    \toprule[0.1mm]
    \end{tabular}
    \caption{Number of gaps (total and percentual), and minimum, maximum, median, and standard deviation, of the sample values (in coordinate degrees), for each data type of the dataset Tornado. \noGaps}
    \label{datasets:table:tornado}
\end{center}
\end{table}

