

\newcommand{\forEachColumnCoder}{\column \textnormal{ in \file.\dataColumns}}

\beginAlgorithm
\inputAndOutput{\cInputFile\\ \cInputVariant\\ \cInputThresholdOpt\\ \cInputWindowOpt}{\cOutputFileA}
Create output file \out\\
Encode an algorithm identification key, and parameter \win\ (if applies)\\
Encode the header of the input file\\
Encode the number of rows and columns in the input file\\
Encode the timestamp column using a lossless code\\
\If{\maskMode}{
Encode gap locations in each signal column of the input file (independently)\\
}
\uIf{\textnormal{we are using a constant or linear model algorithm}}{
    Encode each signal column of the input file (independently), using a coding routine for a specific algorithm (i.e. Base, PCA, APCA, PWLH, PWLHInt, CA, SF, FR)
}
\ElseIf{\textnormal{we are using a correlation model algorithm}}{
    Group the signal columns of the input file into disjoint subsets, then encode each subset independently, using a coding routine for a specific algorithm (i.e. GAMPS)
}
\returnn \out\\
\EndPseudo{General encoding scheme for every evaluated algorithm.}{\label{pseudoCodeCommon}}


\newcommand{\encodedColumns}{\text{encoded\_columns}}
\newcommand{\forEachColumnDecoder}{\encodedColumn \textnormal{ in \file.\encodedColumns}}
\newcommand{\whileFileLeftToDecode}{$\notCond\ \file.\textnormal{reached\_eof?}$}

\newcommand{\decodeColParams}{\file, \out, \win, \textit{\tsColumn}}