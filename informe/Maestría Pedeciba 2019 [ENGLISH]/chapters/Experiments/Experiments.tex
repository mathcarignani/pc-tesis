% Chapter ?

\chapter{Experimental Results} % Main chapter title

\label{experiments} % For referencing the chapter elsewhere, use \ref{Chapter1} 

\lhead{Chapter 3. \emph{Experimental Results}} % This is for the header on each page - perhaps a shortened title

\newcommand{\maskalgo}{\textit{M}}
\newcommand{\NOmaskalgo}{\textit{NM}}
\newcommand{\coder}{\textit{c}}
\newcommand{\difrelativa}{\textit{RD}}
\newcommand{\tasacompresion}{\textit{CR}}
\newcommand{\nmbits}{\NOmaskalgo_{\textit{S}}}
\newcommand{\mbits}{\maskalgo_\textit{S}}
\newcommand{\cmaskalgo}{$c_\maskalgo$}
\newcommand{\cNOmaskalgo}{$c_\NOmaskalgo$}
\newcommand{\ca}{\textit{CI}}
\newcommand{\algo}{\textit{c}}




In this chapter we present our experimental results. The main goal of our experiments is to analyze the performance of each of the coding algorithms presented in Chapter~\ref{coders}, by encoding the various datasets introduced in Chapter~\ref{datasets}. In Section~\ref{experiments:experiments} we describe our experimental setting, defining the evaluated combinations of algorithms and parameter values, and the figures of merit used for comparison. In Section~\ref{secX:windows} we analyze how the window size parameter affects the performance of the algorithms. In Section~\ref{secX:codersmask} we compare the performance of the different algorithms among each other, while in Section~\ref{secX:gzip} we compare them with the gzip algorithm.



\clearpage
\section{Experimental Setting}
\label{experiments:experiments}


We denote by $A$ the set of all the coding algorithms presented in Chapter~\ref{algo}. For an algorithm $a \in A$, we denote by $a_v$ its variant $v$, where $v$ can be $\maskalgo$ (masking) or $\NOmaskalgo$ (non-masking). Recall that there exist some $a \in A$ for which either $a_\maskalgo$ or $a_\NOmaskalgo$ is invalid (see Table~\ref{algo:table:overview}). We denote by $V$ the set of variants consisting of every \textit{valid} variant $a_v$ for every algorithm $a \in A$. Also, we denote by $A_M$ the subset of algorithms from $A$ consisting of every algorithm for which both variants, $a_\maskalgo$ and $a_\NOmaskalgo$, are valid.


We evaluate the compression performance of every algorithm $a \in A$ on the datasets described in Chapter~\ref{datasets}. Considering that, in datasets with heterogeneous data, selecting different compression algorithms for different columns may lead to a better overall compression, we report our experimental results by data type. For each algorithm we test every valid variant $a_v$. We also test several combinations of algorithm parameters. Specifically, for the algorithms that admit a window size parameter $w$ (every algorithm except Base), we test all the values of $w$ in the set $W = \{4, 8, 16, 32, 64, 128, 256\}$. For the encoders that admit a near-lossless compression mode with an error threshold parameter~$\maxerror$ (every encoder except Base), the value of $\maxerror$ is calculated as a percentage fraction, denoted $e$, of the standard deviation of the data being encoded. For example, for certain data with a standard deviation of 20, if $e=10$, then $\maxerror=2$. We test all the values of the parameter $e$ in the set $E= \{1, 3, 5, 10, 15, 20, 30\}$. 


\vspace{+5pt}
\begin{defcion}
We refer to a specific combination of a coding algorithm variant and its parameter values as a \textit{coding algorithm instance (CAI)}. We define \textit{CI} as the set of all the CAIs obtained by combining each of the variants $a_v \in V$ with the parameter values (from $W$ and $E$) that are suitable for algorithm $a$. We denote by $c_{<a_v, w, e>}$ the CAI obtained by setting a window size parameter equal to $w$ and an error parameter equal to $e$ on algorithm variant $a_v$.
\end{defcion}


We assess the compression performance of a CAI through the compression ratio, which we define next. For this definition, we regard Base as a trivial CAI that serves as a base ground for compression performance comparison (recall the definition of algorithm Base from Section~\ref{algo:base}).


\vspace{+5pt}
\begin{defcion}
Let $f$ be a file and $z$ a data type of a certain dataset. We define $f_z$ as the subset of data of type $z$ from file $f$. For example, for the dataset \datasethail, presented in Section~\ref{datasets:hail}, the data type $z$ may be Latitude, Longitude, or Size.
\end{defcion}


\vspace{+2pt}
\begin{defcion}
\label{eq:coding-size}
Let $f$ be a file and $z$ a data type of a certain dataset. Let $\algo \in \ca$ be a CAI. We define $|\algo(z, f)|$ as the size in bits of the resulting bit stream obtained by coding $f_z$ with $\algo$.
\end{defcion}


\vspace{+2pt}
\begin{defcion}
\label{def:compression-rate}
The \textit{compression ratio (\CRit)} of a CAI $\algo \in \ca$ \ for the data type $z$ of a certain file $f$ is the fraction of $|\algo(z, f)|$ with respect to $|\coderBase(z, f)|$, i.e.,
\vspace{-5pt}
\begin{equation}
\label{eq:compression-rate}
\CR(\algo, z, f) = \frac{|\algo(z, f)|}{|\coderBase(z, f)|}.
\end{equation}
\end{defcion}


Notice that smaller values of \CR\ correspond to better performance. Our main goals are to analyze which CAIs yield the smallest values in (\ref{eq:compression-rate}) for the different data types, and to study how the \CR\ depends on the different algorithms, their variants and the parameter values.


\clearpage


To compare the compression performance between two CAIs we calculate the relative difference, which we define next.


\vspace{+5pt}
\begin{defcion}
\label{relative-difference}
The \textit{relative difference (\RDit)} two CAIs $\algo_1, \algo_2 \in {\ca}$ \ for the data type $z$ of a certain file $f$ is given by
\vspace{-5pt}
\begin{equation}
\label{eq:relative-difference}
\RD(\algo_1, \algo_2, z, f)  =
100\times\frac{|\algo_2 (z, f)| - |\algo_1 (z, f)|}{ |\algo_2 (z, f)| }.
\end{equation}
\end{defcion}


Notice that $\algo_1$ has a better performance than $\algo_2$ if (\ref{eq:relative-difference}) is positive.


\vspace{+3pt}
In some of our experiments we consider the performance of algorithms on complete datasets, rather than individual files. With this in mind, we extend the definitions~\ref{eq:coding-size}--\ref{relative-difference} to datasets, as follows.


\vspace{+5pt}
\begin{defcion}
\label{eq:coding-size-dataset}
Let $z$ be a data type of a certain dataset $d$. We define $F(d, z)$ as the set of files $f$ from dataset $d$ for which $f_z$ is not empty.
\end{defcion}


\begin{defcion}
Let $z$ be a data type of a certain dataset $d$. Let $\algo \in \ca$ \ be a CAI. We define $|\algo(z, d)|$ as
\vspace{-5pt}
\begin{equation}
\label{eq:dataset-size}
|\algo(z, d)|  = \sum_{f \in F(d, z)}^{} |\algo(z, f)|.
\end{equation}
\end{defcion}


\vspace{+3pt}
\begin{defcion}
The \textit{compression ratio (\CRit)} of a CAI $\algo \in \ca$ \ for the data type $z$ of a certain dataset $d$ is given by
\vspace{-5pt}
\begin{equation}
\label{eq:compression-rate-dataset}
\CR(\algo, z, d) = \frac{|\algo(z, d)|}{|\coderBase(z, d)|}.
\end{equation}
\end{defcion}


\vspace{+3pt}
\begin{defcion}
\label{def:relative-difference-dataset}
The \textit{relative difference (\RDit)} between a pair of CAIs $\algo_1, \algo_2 \in {\ca}$ \ for the data type $z$ of a certain dataset $d$ is given by
\vspace{-5pt}
\begin{equation}
\label{eq:relative-difference-dataset}
\RD(\algo_1, \algo_2, z, d)  =
100\times\frac{|\algo_2 (z, d)| - |\algo_1 (z, d)|}{ |\algo_2 (z, d)| }.
\end{equation}
\end{defcion}




\clearpage
\section{Comparison of Masking and Non-Masking Variants}
\label{secX:rendimiento-relativo}


In this section, we compare the compression performance of the masking and non-masking variants of every coding algorithm in $A_M$ (recall this definition from the first paragraph in Section~\ref{experiments:experiments}). Specifically, we compare:


\vspace{-6pt}
\newcommand{\against}[1]{$\text{{#1}}_\textit{M}$ against $\text{{#1}}_\textit{NM}$}
\begin{itemize}
    \item \against{PCA}
    \item \against{APCA}
    \item \against{CA}
    \item \against{PWLH}
    \item \against{PWLHInt}
    \item \against{GAMPS}
\end{itemize}


\vspace{+3pt}
For each algorithm $a \in A_M$, and each error parameter $e \in E$, we compare the performance of $a_\maskalgo$ and $a_\NOmaskalgo$. For the purpose of this comparison, we choose the most favorable window size for each variant $a_v$, in the sense of the following definition.


\vspace{+5pt}
\begin{defcion}
\label{def:ows}
The \textit{optimal window size (\owsit)} of a coding algorithm variant $a_v \in V$, and an error parameter $e \in E$, for the data type $z$ of a certain dataset $d$, is given by
\begin{equation}
\label{eq:ows}
\ows(a_v, e, z, d) = \argmin_{w\ \in \ W} \biggl\{ \CR(c_{<a_v, w, e>}, z, d) \biggr\},
\end{equation}
where we break ties in favor of the smallest window size.
\end{defcion}


For each data type $z$ of each dataset $d$, and each coding algorithm $a \in A_M$ and error parameter $e \in E$, we calculate the RD between $c_{<a_\maskalgo, w_\maskalgo^{*}, e>}$ and $c_{<a_\NOmaskalgo, w_\NOmaskalgo^{*}, e>}$, as defined in~(\ref{eq:relative-difference-dataset}), where $w_\maskalgo^{*}=\owsns(a_\maskalgo, e, z, d)$ and $w_\NOmaskalgo^{*}=\owsns(a_\NOmaskalgo, e, z, d)$.


\vspace{+2pt}
As an example, in figures~\ref{fig:diff-sst} and~\ref{fig:diff-tornado} we show the CR and the RD, as a function of the error parameter, obtained for two data types of two different datasets. Figure~\ref{fig:diff-sst} shows the results for the data type $z=$~``SST" of the dataset $d=$~\datasetsst, presented in Section~\ref{datasets:sst}, and Figure~\ref{fig:diff-tornado} shows the results for the data type $z=$~``Longitude" of the dataset $d=$~\datasettornado, presented in Section~\ref{datasets:tornado}. In Figure~\ref{fig:diff-sst} we observe a large RD favoring the masking variant for all tested algorithms. On the other hand, in Figure~\ref{fig:diff-tornado} we observe that the non-masking variant outperforms the masking variant for all algorithms. We notice, however, that the RD is very small in the latter case.


\clearpage

%%%%%%%%%%%%%%%%%%%%%%%%%%%%%%%%%%%%%
%%%%%%%%%%%%%%%% 3.2 %%%%%%%%%%%%%%%%
%%%%%%%%%%%%%%%%%%%%%%%%%%%%%%%%%%%%%
\newcommand{\threetwocommon}[4]{
In the relative difference plot for algorithm {#1} we made a {#2} circle around the marker with the {#3} value obtained for all of the tested CAIs ({#4}).
}
\newcommand{\threetwomost}{\threetwocommon{PCA}{red}{maximum}{50.60\%}}
\newcommand{\threetwoleast}{\threetwocommon{APCA}{blue}{minimum}{-0.29\%}}

\newcommand{\threetwosingle}[5]{
    \clearpage
    \begin{figure}
    \hspace{-90pt} % trim=left bottom right top
    \includegraphics[clip,trim=0 2.9cm 0 3.5cm,height=23.5cm]{appendices/1-imagesPlots3.2/{#1}}
    \hspace{+5pt}
    \caption{Compression ratio and relative difference plots for every pair of algorithm variants $a_\maskalgo, a_\NOmaskalgo \in A$, for the data type ``{#3}" of the dataset {#2}.{#4}}
    {#5}
    \end{figure}
}

%%%%%%%%%%%%%%%%%%%%%%%%%%%%%%%%%%%%%
%%%%%%%%%%%%%%%% 3.3 %%%%%%%%%%%%%%%%
%%%%%%%%%%%%%%%%%%%%%%%%%%%%%%%%%%%%%

\newcommand{\threethreecommon}[4]{
In the relative difference plot for algorithm {#1} we made a {#2} circle around the marker with the {#3} value obtained for all the tested CAIs ({#4}).
}

\newcommand{\threethreemost}{\threethreecommon{PCA}{red}{maximum}{10.68\%}}

\newcommand{\threethreesingle}[6]{
    \clearpage
    \begin{figure}
    \hspace{-100pt} % trim=left bottom right top
    \includegraphics[clip,trim=0 1.8cm 0 2.5cm,height=19.5cm]{appendices/2-imagesPlots3.3/{#1}}
    \hspace{+5pt}
    \caption{Global and local window sizes, and relative difference plots for every algorithm, for the data type ``{#2}" of the file ``{#3}" of the dataset {#4}.{#6}}
    {#5}
    \end{figure}
}

\threetwosingle{2-NOAA-SST-1}{\datasetsst}{SST}{\threetwomost}{\label{fig:diff-sst}}
\threetwosingle{7-NOAA-SPC-tornado-2}{\datasettornado}{Longitude}{\threetwoleast}{\label{fig:diff-tornado}}

\clearpage


\newcommand{\errParVal}{8 different error parameter values}
We analyze the experimental results to compare the performance of the masking and non-masking variants of each algorithm. For each data type, we iterate through each algorithm $a \in A_M$, and each error parameter $e \in E$, and we calculate the RD between the CAIs $c_{<a_\maskalgo, w_\maskalgo^{*}, e>}$ and $c_{<a_\NOmaskalgo, w_\NOmaskalgo^{*}, e>}$, obtained by setting the OWS for the masking variant $a_\maskalgo$ and the non-masking variant $a_\NOmaskalgo$, respectively. Since we consider \errParVal\ and there are 6 algorithms in $A_M$, for each data type we compare a total of 48 pairs of CAIs. Table~\ref{tabla:rendimiento-relativ-NM-M} summarizes the results of these comparisons, aggregated by dataset. The number of pairs of CAIs evaluated for each dataset depends on the number of different data types it contains.


\vspace{+5pt}

\begin{table}[h]
\begin{center}
    \begin{tabular}{| C{2.2cm} || C{2.5cm} | C{4.4cm} | C{3.0cm} |}
    \hline
      \multicolumn{1}{|>{\centering\arraybackslash}m{2.2cm}||}{\textbf{Dataset}} 
    & \multicolumn{1}{>{\centering\arraybackslash}m{2.5cm}|}{\textbf{Dataset Characterstic}} 
    & \multicolumn{1}{>{\centering\arraybackslash}m{4.4cm}|}{\textbf{Cases where $a_\maskalgo$ outperforms $a_\NOmaskalgo$ (\%)}}
    & \multicolumn{1}{>{\centering\arraybackslash}m{3.0cm}|}{\textbf{RD (\%) Range}}\\
    \hline
    \datasetirkis   & Many gaps     & 48/48 (100\%) & (0; 36.88]                    \\\hline
    \datasetsst     & Many gaps     & 48/48 (100\%) & (0; \textcolor{red}{50.60}]  \\\hline
    \datasetadcp    & Many gaps     & 48/48 (100\%) & (0; 17.35]                    \\\hline
    \datasetelnino  & Many gaps     & 336/336 (100\%) & (0; 50.52]                    \\\hline
    \datasetsolar   & Few gaps      & 73/144 (50.7\%) & [-0.25; 1.77]                 \\\hline
    \datasethail    & No gaps       & 0/144 (0\%)   & [-0.04; 0)                    \\\hline
    \datasettornado & No gaps       & 0/96 (0\%)   & [\textcolor{blue}{-0.29}; 0)   \\\hline
    \datasetwind    & No gaps       & 0/144 (0\%)   & [-0.12; 0)                    \\\hline
    \toprule[0.1mm]
    \end{tabular}
    \caption{Relative difference between the masking and non-masking variants of each algorithm. The results are aggregated by dataset. In the last column we highlight\\the maximum (red) and minimum (blue) values taken by the RD.}
    \label{tabla:rendimiento-relativ-NM-M}
\end{center}
\end{table}

\vspace{-5pt}


Consider, for example, the results for the dataset Wind, in the last row. The second column shows that there are no gaps in any of the data types of the dataset (recall the dataset information from Table~\ref{datasets:table:overview}). Since the dataset has three data types, we compare a total of $3\times48=144$ pairs of CAIs. The third column reveals that in none of these comparisons the masking variant $a_\maskalgo$ outperforms the non-masking variant $a_\NOmaskalgo$, i.e. the RD is always negative. The last column shows the range for the values attained by the RD for those tested CAIs.


Observing the last column of Table~\ref{tabla:rendimiento-relativ-NM-M}, we notice that, in every case in which the non-masking variant performs best, the RD is close to zero. The minimum value it takes is $-0.29\%$, which is obtained for the data type ``Longitude" of the dataset \datasettornado, with algorithm APCA, and error parameter $e=30$. In Figure~\ref{fig:diff-tornado} we highlight the marker associated to this minimum with a blue circle. On the other hand, we observe that, for the datasets in which the masking variant performs best, the RD reaches high absolute values. The maximum ($50.78\%$) is obtained for the data type ``VWC" of the dataset \datasetsst, with algorithm PCA, and error parameter $e=30$, which is highlighted in Figure~\ref{fig:diff-sst} with a red circle.


\newcommand{\vaster}{V^*}
The experimental results presented in this section suggest that if we were interested in compressing a dataset with many gaps, we would benefit from using the masking variant of an algorithm, $a_\maskalgo$. However, even if the dataset didn't have any gaps, the performance would not be significantly worse than that obtained by using the non-masking variant of the algorithm, $a_\NOmaskalgo$. Therefore, since masking variants are, in general, more robust in this sense, in the sequel we focus on the set of variants $\vaster$ that we define next.


\vspace{+5pt}
\begin{defcion}
\label{defcion:vaster}
We denote by $\vaster$ the set of all the masking algorithm variants $a_\maskalgo$ for $a \in A$.
\end{defcion}


Notice that $\vaster$ includes a single variant for each algorithm. Therefore, in what follows we sometimes refer to the elements of $\vaster$ simply as algorithms.




\clearpage
\section{Window Size Parameter}
\label{secX:windows}

In this section, we analyze the extent to which the window size parameter impacts the performance of the coding algorithms. We only consider the four datasets that consist of multiple files, i.e. \datasetirkis, \datasetsst, \datasetadcp \ and \datasetsolar. For each file, we compare the compression performance when using the optimal window size for the dataset, as defined in (\ref{eq:ows}), and the local optimal window size, defined next.


\newcommand{\lows}{\textit{LOWS}}
\begin{defcion}
The \textit{local optimal window size (\lows)} of a coding algorithm $a \in A$ and a threshold parameter $e \in E$, for the data type $z$ of a certain file $f$ is given by
\begin{equation}
\lows(a, e, z, f) = \argmin_{w\ \in \ W} \biggl\{ \tasacompresion(c_{<a, w, e>}, z, f) \biggr\},
\end{equation}
where we break ties in favor of the smallest window size.
\end{defcion}


For each data type $z$ of each dataset $d$, and each file $f \in F(d, z)$, coding algorithm $a \in A$ and threshold parameter $e \in E$, we calculate the relative difference between $c_{<a, w_{global}^{*}, e>}$ and $c_{<a, w_{local}^{*}, e>}$, as defined in~(\ref{eq:relative-difference}), where $w_{global}^{*}=OWS(a, e, z, d)$ and $w_{local}^{*}=LOWS(a, e, z, f)$. In what follows, we refer to $w_{global}^{*}$ and $w_{local}^{*}$ as the global and local window size, respectively.


As an example, in Figures~\ref{fig:window-compare-1202} and~\ref{fig:window-compare-1203} we display the global and local window sizes and the relative difference, as a function of the threshold, obtained for the data type ``VWC", for two different files of the dataset \datasetirkis. Figure~\ref{fig:window-compare-1202} shows the results for the file ``vwc\_1202.dat.csv", while Figure~\ref{fig:window-compare-1203} shows the results for ``vwc\_1203.dat.csv". Observe that the global window sizes are repeated for every matching plot of both figures, which is expected, since both figures consider the same data type of the same dataset.


In Figure~\ref{fig:window-compare-1202} we notice, for instance, that in the APCA algorithm case both window sizes match for every threshold parameter $e$, except 3 and 10. The global window is larger than the local window when $e=3$, but it is smaller when $e=10$. In those two cases the relative difference values are 1.52 and 1.76, respectively. We observe that the relative difference is non-negative in every plot, which makes sense, since the compression ratio obtained when using the global window cannot be lower than the compression ratio obtained when using the local window.


\clearpage

\threethreesingle{1-IRKIS-1-1}{VWC}{vwc\_1202.dat.csv}{\datasetirkis}{\label{fig:window-compare-1202}}{}
\threethreesingle{1-IRKIS-2-1}{VWC}{vwc\_1203.dat.csv}{\datasetirkis}{\label{fig:window-compare-1203}}{\threethreemost}

\clearpage


NO LEER ESTA PARTE, TODAVÍA NO TERMINÉ.

TODO:
\vspace{-10pt}
\begin{itemize}
    \item Reescribir el párrafo que presenta a la tabla
    \item Analizar los datos de la tabla
    \item Escribir un párrafo final con conclusiones
\end{itemize}

Table~\ref{tabla:windows-comparison} summarizes the results obtained for each combination of algorithm, error threshold parameter, data type and file. In 88.7\% of the cases both window sizes match, and so the relative difference is 0. In the remaining cases, most of the times the relative difference is smaller than 1, and there are only six cases in which the relative difference is larger than 5.


\vspace{+5pt}




\begin{table}[h]

\begin{center}

    \begin{tabular}{| C{2.5cm} || C{2.2cm} | C{1.5cm} | C{1.5cm} | C{1.5cm} | C{1.5cm} |}

    \hline

    \multicolumn{1}{|>{\centering\arraybackslash}m{2.5cm}||}{}

    & \multicolumn{5}{>{\centering\arraybackslash}m{9cm}|}{RD (\%) Range}\\

    \hline

      \multicolumn{1}{|>{\centering\arraybackslash}m{2.5cm}||}{\textbf{Algorithm}}

    & \multicolumn{1}{>{\centering\arraybackslash}m{2.2cm}|}{\textbf{0}}

    & \multicolumn{1}{>{\centering\arraybackslash}m{1.5cm}|}{\textbf{(0,1]}}

    & \multicolumn{1}{>{\centering\arraybackslash}m{1.5cm}|}{\textbf{(1,2]}}

    & \multicolumn{1}{>{\centering\arraybackslash}m{1.5cm}|}{\textbf{(2,5]}}

    & \multicolumn{1}{>{\centering\arraybackslash}m{1.5cm}|}{\textbf{(5,14]}}\\

    \hline\hline

    PCA & 186 (93\%) & 3 (1.5\%) & 4 (2\%) & 2 (1\%) & 5 (2.5\%) \\\hline
    APCA & 174 (87\%) & 13 (6.5\%) & 7 (3.5\%) & 6 (3\%) & 0 \\\hline
    CA & 172 (86\%) & 16 (8\%) & 6 (3\%) & 6 (3\%) & 0 \\\hline
    FR & 171 (85.5\%) & 14 (7\%) & 8 (4\%) & 7 (3.5\%) & 0 \\\hline
    PWLH & 184 (92\%) & 13 (6.5\%) & 3 (1.5\%) & 0 & 0 \\\hline
    PWLHInt & 173 (86.5\%) & 9 (4.5\%) & 13 (6.5\%) & 4 (2\%) & 1 (0.5\%) \\\hline
    GAMPS & 161 (80.5\%) & 16 (8\%) & 0 & 10 (5\%) & 13 (6.5\%) \\\hline
    SF & 199 (99.5\%) & 1 (0.5\%) & 0 & 0 & 0 \\\hline\hline
    Total & 1,420 (88.7\%) & 85 (5.3\%) & 41 (2.6\%) & 35 (2.2\%) & 19 (1.2\%) \\\hline
    \toprule[0.1mm]

    \end{tabular}

    \caption{RD between the \ows and \lows variants of each CAI.\\The results are aggregated by algorithm and the range to which the RD belongs.}

    \label{tabla:windows-comparison}

\end{center}

\end{table}


\vspace{-5pt}


In Figure~\ref{fig:window-compare-1203} we display the plots obtained for the ``VWC" data type of the ``vwc\_1203.dat.csv" file. For CoderPCA and $e=15$ the relative difference is 10.68, which is the largest value obtained for all of the combinations. The next four largest relative differences (9.79, 9.22, 7.20, and 5.51) are also obtained with the CoderPCA algorithm. These results support the idea that the performance of the CoderPCA algorithm is more sensible to the window size parameter than the rest of the algorithms.














\clearpage
\section{Mask Coders Performance}
\label{secX:codersmask}

In this section we analyze the performance of every one of the mask coders implemented in Chapter \ref{coders}. Once again, the compression rate (equation~(\ref{eq:compression-rate})) and the relative difference (equation~(\ref{eq:relative-difference})) will be the metrics we use for comparing the coders between each other.

We considered the results obtained when coding the different data types of the datasets introduced in Chapter \ref{datasets}. For example, in Figure \ref{fig:mask-irkis} we can see the graphs obtained for the ``VWC" data type of the \datasetirkis \ dataset. For each <$\coder \in C$, $e \in E$> combination we plot two values: the window size which minimizes the compression rate and said compression rate.

Easily, after observing the plots we noticed that in general the compression rate for coders \textit{CoderPWLH-M}, \textit{CoderGAMPSLimit-M} and \textit{CoderSF-M} was worst than the rest.

Analyze the data and discard these 

PONER OTRA GRAFICA DE OTRO TIPO DE DATO

\clearpage


\newcommand{\legendsone}{
\begin{tabular}{| C{1.5cm} | C{1.5cm} | C{1.5cm} |}
\hline
  \multicolumn{1}{|>{\centering\arraybackslash}m{1.5cm}|}{\cpca PCA} 
& \multicolumn{1}{>{\centering\arraybackslash}m{1.5cm}|}{\capca APCA} 
& \multicolumn{1}{>{\centering\arraybackslash}m{1.5cm}|}{\cfr FR}\\
\toprule[0.1mm]
\end{tabular}
\vspace{+30pt}
}



\begin{sidewaystable}[ht]
\newcommand{\cgzip}{\cellcolor{orange!20}}
\newcommand{\cfr}{\cellcolor{yellow!25}}
\newcommand{\cpca}{\cellcolor{cyan!20}}
\newcommand{\capca}{\cellcolor{green!20}}
\centering

\legendsone

\begin{tabular}{| l | l | c | c || c | c || c | c || c | c || c | c || c | c || c | c || c | c |}
\cline{3-18}
\multicolumn{1}{c}{}& \multicolumn{1}{c|}{} & \multicolumn{2}{c||}{e = 0} & \multicolumn{2}{c||}{e = 1} & \multicolumn{2}{c||}{e = 3} & \multicolumn{2}{c||}{e = 5} & \multicolumn{2}{c||}{e = 10} & \multicolumn{2}{c||}{e = 15} & \multicolumn{2}{c||}{e = 20} & \multicolumn{2}{c|}{e = 30} \\\hline
{Dataset} & {Data Type} & {\footnotesize OWS} & {\footnotesize CR} & {\footnotesize OWS} & {\footnotesize CR} & {\footnotesize OWS} & {\footnotesize CR} & {\footnotesize OWS} & {\footnotesize CR} & {\footnotesize OWS} & {\footnotesize CR} & {\footnotesize OWS} & {\footnotesize CR} & {\footnotesize OWS} & {\footnotesize CR} & {\footnotesize OWS} & {\footnotesize CR} \\\hline\hline
{\datasetirkis} & {VWC} & {\capca4} & {\capca20.32} & {\capca4} & {\capca18.35} & {\capca5} & {\capca12.37} & {\capca6} & {\capca6.77} & {\capca7} & {\capca3.07} & {\capca8} & {\capca2.22} & {\capca8} & {\capca1.71} & {\capca8} & {\capca1.21} \\\hline
{\datasetsst} & {SST} & {\cpca8} & {\cpca60.84} & {\capca3} & {\capca28.12} & {\capca5} & {\capca13.64} & {\capca6} & {\capca8.88} & {\capca7} & {\capca4.63} & {\capca8} & {\capca3.15} & {\capca8} & {\capca2.39} & {\capca8} & {\capca1.72} \\\hline
{\datasetadcp} & {Vel} & {\cpca8} & {\cpca68.22} & {\cpca8} & {\cpca68.22} & {\capca2} & {\capca66.8} & {\capca2} & {\capca61.07} & {\capca2} & {\capca48.44} & {\capca2} & {\capca40.9} & {\capca3} & {\capca34.9} & {\capca3} & {\capca25.93} \\\hline
{\datasetsolar} & {GHI} & {\cpca2} & {\cpca77.65} & {\capca3} & {\capca76.1} & {\capca4} & {\capca71.39} & {\capca4} & {\capca67.2} & {\capca4} & {\capca58.52} & {\capca4} & {\capca52.41} & {\capca4} & {\capca47.03} & {\capca4} & {\capca37.78} \\\hline
{} & {DNI} & {\cpca2} & {\cpca75.93} & {\capca4} & {\capca72.22} & {\capca4} & {\capca65.75} & {\capca4} & {\capca61.37} & {\capca4} & {\capca53.98} & {\capca4} & {\capca48.55} & {\capca4} & {\capca43.36} & {\capca4} & {\capca35.66} \\\hline
{} & {DHI} & {\cpca2} & {\cpca77.66} & {\cpca2} & {\cpca77.43} & {\capca4} & {\capca71.62} & {\capca4} & {\capca67.6} & {\capca4} & {\capca60.12} & {\capca4} & {\capca53.62} & {\capca4} & {\capca47.86} & {\capca4} & {\capca38.71} \\\hline
{\datasetelnino} & {Lat} & {\capca4} & {\capca15.96} & { } & { } & {\capca4} & {\capca15.82} & {\capca4} & {\capca15.11} & {\capca4} & {\capca12.34} & {\capca5} & {\capca9.89} & {\capca5} & {\capca8.61} & {\capca6} & {\capca5.76} \\\hline
{} & {Long} & {\capca3} & {\capca17.36} & {\capca4} & {\capca17.05} & {\capca4} & {\capca13.04} & {\capca5} & {\capca11.75} & {\capca6} & {\capca8.65} & {\capca6} & {\capca6.56} & {\capca7} & {\capca4.93} & {\capca8} & {\capca2.37} \\\hline
{} & {Zonal Winds} & {\cpca8} & {\cpca31.46} & { } & { } & {\cpca8} & {\cpca31.46} & {\cpca8} & {\cpca31.46} & {\capca2} & {\capca27.36} & {\capca2} & {\capca23.5} & {\capca2} & {\capca20.54} & {\capca3} & {\capca16.44} \\\hline
{} & {Merid. Winds} & {\cpca8} & {\cpca31.46} & { } & { } & {\cpca8} & {\cpca31.46} & {\cpca8} & {\cpca31.46} & {\capca2} & {\capca29.16} & {\capca2} & {\capca25.86} & {\capca2} & {\capca23.33} & {\capca2} & {\capca19.15} \\\hline
{} & {Humidity} & {\cpca8} & {\cpca23.1} & {\cpca8} & {\cpca23.1} & {\cpca8} & {\cpca23.1} & {\cpca8} & {\cpca23.1} & {\capca2} & {\capca20.51} & {\capca2} & {\capca18.14} & {\capca2} & {\capca16.01} & {\capca2} & {\capca12.94} \\\hline
{} & {AirTemp} & {\cpca8} & {\cpca32.68} & {\cpca8} & {\cpca32.68} & {\capca2} & {\capca30.33} & {\capca2} & {\capca27.39} & {\capca2} & {\capca22.42} & {\capca3} & {\capca19.24} & {\capca3} & {\capca16.76} & {\capca4} & {\capca13.31} \\\hline
{} & {SST} & {\cpca8} & {\cpca32.91} & {\capca2} & {\capca30.96} & {\capca2} & {\capca24.6} & {\capca2} & {\capca20.61} & {\capca3} & {\capca14.17} & {\capca4} & {\capca10.66} & {\capca4} & {\capca8.21} & {\capca5} & {\capca5.42} \\\hline
{\datasethail} & {Lat} & {\cpca8} & {\cpca\color{red}100.04} & {\cpca8} & {\cpca\color{red}100.04} & {\capca2} & {\capca89.83} & {\capca2} & {\capca82.62} & {\capca2} & {\capca71.49} & {\capca3} & {\capca64.62} & {\capca3} & {\capca57.49} & {\capca3} & {\capca46.75} \\\hline
{} & {Long} & {\cpca8} & {\cpca\color{red}100.03} & {\cpca8} & {\cpca\color{red}100.03} & {\capca2} & {\capca85.91} & {\capca2} & {\capca77.5} & {\capca2} & {\capca65.06} & {\capca3} & {\capca55.38} & {\capca3} & {\capca48.72} & {\capca4} & {\capca38.74} \\\hline
{} & {Size} & {\capca2} & {\capca80.61} & {\capca2} & {\capca80.59} & {\capca2} & {\capca80.59} & {\capca2} & {\capca80.58} & {\capca2} & {\capca80.56} & {\capca2} & {\capca80.53} & {\capca2} & {\capca80.52} & {\capca3} & {\capca64.35} \\\hline
{\datasettornado} & {Lat} & {\cpca8} & {\cpca\color{red}100.05} & {\capca2} & {\capca85.43} & {\capca2} & {\capca70.63} & {\capca2} & {\capca65.17} & {\capca3} & {\capca54.17} & {\capca3} & {\capca46.78} & {\capca4} & {\capca41.95} & {\capca4} & {\capca33.48} \\\hline
{} & {Long} & {\cpca8} & {\cpca\color{red}100.11} & {\capca2} & {\capca82.12} & {\capca2} & {\capca65.09} & {\capca3} & {\capca57.66} & {\capca3} & {\capca45.55} & {\capca4} & {\capca39.88} & {\capca4} & {\capca34.84} & {\capca4} & {\capca28.41} \\\hline
{\datasetwind} & {Lat} & {\cpca8} & {\cpca\color{red}100.03} & {\cpca8} & {\cpca\color{red}100.03} & {\capca2} & {\capca88.74} & {\capca2} & {\capca81.29} & {\capca2} & {\capca69.82} & {\capca3} & {\capca62.44} & {\capca3} & {\capca56.18} & {\capca3} & {\capca47.15} \\\hline
{} & {Long} & {\cpca8} & {\cpca\color{red}100.03} & {\capca2} & {\capca95.41} & {\capca2} & {\capca80.29} & {\capca2} & {\capca73.21} & {\capca3} & {\capca62.06} & {\capca3} & {\capca54.33} & {\capca3} & {\capca48.52} & {\capca4} & {\capca39.73} \\\hline
{} & {Speed} & {\cfr4} & {\cfr65.49} & {\capca3} & {\capca43.82} & {\cfr6} & {\cfr25.9} & {\cfr7} & {\cfr16.79} & {\capca5} & {\capca15.71} & {\capca6} & {\capca12.29} & {\capca6} & {\capca10.33} & {\capca6} & {\capca8.21} \\\hline
\end{tabular}
\caption{Mask results overview (1).}
\label{experiments:mask-results-overview1}
\end{sidewaystable}



\clearpage
\input{chapters/Experiments/other/mask-results-overview2.tex}


\clearpage
\begin{figure}
\hspace{-35pt}
\includegraphics[scale=0.50]{chapters/Experiments/images/1-IRKIS.png}
\hspace{+10pt}
\caption{Compression rate and Window size graphs for the different combinations\\<$\coder \in C$, $w \in W, e \in E$> for the ``VWC" data type of the \datasetirkis \ dataset.}
\label{fig:mask-irkis}
\end{figure}

\clearpage







\clearpage
\section{Comparison with the gzip Algorithm}
\label{secX:gzip}