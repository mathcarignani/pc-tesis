\section{Experimental Setting}
\label{experiments:experiments}


We evaluate the compression performance of all the coding algorithms presented in Chapter~\ref{coders} on the datasets described in Chapter~\ref{datasets}. For each algorithm we test both the masking and the non-masking modes (except for $\coderBase$, \textit{CoderFR} and \textit{CoderSF}, which only operate in non-masking mode).


We also test several combinations of algorithm parameters. Specifically, for the algorithms that admit a window size parameter $w$ (every algorithm except $\coderBase$ and \textit{CoderSF}), we test all the values of $w$ in the set $W = \{4, 8, 16, 32, 64, 128, 256\}$. For the encoders that admit a lossy compression mode with a threshold parameter $e$ (every encoder except $\coderBase$), we test all the values of $e$ in the set $E= \{1, 3, 5, 10, 15, 20, 30\}$, where each threshold is expressed as a percentage fraction of the standard deviation of the data type being coded. For example, for a certain data type with a standard deviation of 20, taking $e=10$ implies that the lossy compression allows for a maximum sample distortion of 2 sampling units.


\vspace{+5pt}
\begin{defcion}
We refer to a specific combination of a coding algorithm and its parameter values as a \textit{coding algorithm instance (CAI)}. We define \textit{CI} as the set of all the CAIs obtained by combining each of the algorithms presented in Chapter~\ref{coders} with the parameter values (from $W$ and $E$) which are suitable for that algorithm. We denote by $c_{<a, w, e>}$ the CAI obtained by setting a window size equal to $w$ and a threshold parameter equal to $e$ on the algorithm $a$.
\end{defcion}


We assess the compression performance of a CAI mainly through the compression ratio, which we define next. For that definition, we recall that $\coderBase$ is a trivial encoder that serves as a base ground for compression performance comparison.


\clearpage


\begin{defcion}
Let $f$ be a file and let $z$ be a data type of a certain dataset. We define $f_z$ as the subset of data of type $z$ from file $f$.
\end{defcion}


\vspace{+2pt}
\begin{defcion}
The \textit{compression ratio (CR)} of a CAI $\algo \in \ca$ for the data of type $z$ of a certain file $f$ is given by
\vspace{-5pt}
\begin{equation}
\label{eq:compression-rate}
\tasacompresion(\algo, f_z) = 100\times\frac{|\algo(f_z)|}{|\coderBase(f_z)|},
\end{equation}
where $|\algo(f_z)|$ and $|\coderBase(f_z)|$ are the sizes of the resulting files obtained when coding $f_z$ with $\algo$ and $\coderBase$, respectively.
\end{defcion}


The performance of $\algo$ improves as $|\algo(f_z)|$ decreases. Thus, our main goals are to analyze which CAIs minimize (\ref{eq:compression-rate}) for the different data types, and to study how the CR depends on the different coding algorithms and parameter values.


To compare the compression performance between a pair of CAIs we calculate the relative difference, which we define next. In general, it only makes sense to compare CAIs that have the same threshold parameter $e$.


\vspace{+5pt}
\begin{defcion}
The \textit{relative difference (RD)} between a pair of CAIs $\algo_1, \algo_2 \in {\ca}$ for the data of type $z$ of a certain file $f$ is given by
\vspace{-5pt}
\begin{equation}
\label{eq:relative-difference}
\difrelativa(\algo_1, \algo_2, f_z)  =
100\times\frac{|\algo_2 (f_z)| - |\algo_1 (f_z)|}{ |\algo_2 (f_z)| },
\end{equation}
where $|\algo_1(f_z)|$ and $|\algo_2(f_z)|$ are the sizes of the resulting files obtained when coding $f_z$ with $\algo_1$ and $\algo_2$, respectively. Notice that $\algo_1$ has a better performance than $\algo_2$ when (\ref{eq:relative-difference}) is positive.
\end{defcion}

