
\clearpage
\section{Window Size Parameter}
\label{secX:windows}


\newcommand{\lows}{\text{LOWS\ }}
\newcommand{\lowsns}{\text{LOWS}}
In this section, we analyze the extent to which the window size parameter impacts on the performance of the coding algorithms. We only consider the four datasets that consist of multiple files, i.e. \datasetirkis, \datasetsst, \datasetadcp \ and \datasetsolar. For each file, we compare the compression performance when using the \ows for the dataset, as defined in (\ref{eq:ows}), and the \lows for the file, defined next.


\newcommand{\lowsit}{\textit{LOWS}}
\begin{defcion}
The \textit{local optimal window size (\lowsit)} of a coding algorithm $a \in A$ and a threshold parameter $e \in E$, for the data type $z$ of a certain file $f$ is given by
\begin{equation}
\lowsit(a, e, z, f) = \argmin_{w\ \in \ W} \biggl\{ \tasacompresion(c_{<a, w, e>}, z, f) \biggr\},
\end{equation}
where we break ties in favor of the smallest window size.
\end{defcion}


For each data type $z$ of each dataset $d$, and each file $f \in F(d, z)$, coding algorithm $a \in A$ and threshold parameter $e \in E$, we calculate the RD between $c_{<a, w_{global}^{*}, e>}$ and $c_{<a, w_{local}^{*}, e>}$, as defined in~(\ref{eq:relative-difference}), where $w_{global}^{*}=\owsns(a, e, z, d)$ and $w_{local}^{*}=\lowsns(a, e, z, f)$. In what follows, we refer to $w_{global}^{*}$ and $w_{local}^{*}$ as the \ows and the \lows, respectively.


As an example, in figures~\ref{fig:window-compare-1202} and~\ref{fig:window-compare-1203} we show the \ows and the \lows, and the RD, as a function of the threshold, obtained for the data type ``VWC", for two different files of the dataset \datasetirkis. Figure~\ref{fig:window-compare-1202} shows the results for the file ``vwc\_1202.dat.csv", and Figure~\ref{fig:window-compare-1203} shows the results for ``vwc\_1203.dat.csv". Observe that the \ows values are the same for both figures, which is expected, since both are obtained from the same data type of the same dataset.


In Figure~\ref{fig:window-compare-1202} we notice, for instance, that in the algorithm APCA the \ows and \lows values match for every threshold parameter $e$, except 3 and 10. The \ows is larger than the \lows when $e=3$, but it is smaller when $e=10$. In these two cases, the RD values are 1.52 and 1.76, respectively. Notice that the RD is non-negative in every plot, which makes sense, since the CR obtained with the \ows cannot be lower than the CR obtained with the \lows.


\clearpage

\threethreesingle{1-IRKIS-1-1}{VWC}{vwc\_1202.dat.csv}{\datasetirkis}{\label{fig:window-compare-1202}}{}
\threethreesingle{1-IRKIS-2-1}{VWC}{vwc\_1203.dat.csv}{\datasetirkis}{\label{fig:window-compare-1203}}{\threethreemost}

\clearpage


We analyze the experimental results to evaluate the impact of using the \ows instead of the \lows on the compression performance of the tested coding algorithms. For each algorithm, we iterate through each threshold parameter, and each data type of each file, and we calculate the RD between the CAI with the \ows and the CAI with the \lows. Since we consider 8 threshold parameters and there are 13 files with a single data type and 4 files with 3 different data types each, for each algorithm we compare a total of $8 \times (13 + 4\times3) = 200$ pairs of CAIs. Table~\ref{tabla:windows-comparison} summarizes the results of these comparisons, aggregated by algorithm and the range to which the RD belongs. 


\vspace{+5pt}




\begin{table}[h]

\begin{center}

    \begin{tabular}{| C{2.5cm} || C{2.2cm} | C{1.5cm} | C{1.5cm} | C{1.5cm} | C{1.5cm} |}

    \hline

    \multicolumn{1}{|>{\centering\arraybackslash}m{2.5cm}||}{}

    & \multicolumn{5}{>{\centering\arraybackslash}m{9cm}|}{RD (\%) Range}\\

    \hline

      \multicolumn{1}{|>{\centering\arraybackslash}m{2.5cm}||}{\textbf{Algorithm}}

    & \multicolumn{1}{>{\centering\arraybackslash}m{2.2cm}|}{\textbf{0}}

    & \multicolumn{1}{>{\centering\arraybackslash}m{1.5cm}|}{\textbf{(0,1]}}

    & \multicolumn{1}{>{\centering\arraybackslash}m{1.5cm}|}{\textbf{(1,2]}}

    & \multicolumn{1}{>{\centering\arraybackslash}m{1.5cm}|}{\textbf{(2,5]}}

    & \multicolumn{1}{>{\centering\arraybackslash}m{1.5cm}|}{\textbf{(5,14]}}\\

    \hline\hline

    PCA & 186 (93\%) & 3 (1.5\%) & 4 (2\%) & 2 (1\%) & 5 (2.5\%) \\\hline
    APCA & 174 (87\%) & 13 (6.5\%) & 7 (3.5\%) & 6 (3\%) & 0 \\\hline
    CA & 172 (86\%) & 16 (8\%) & 6 (3\%) & 6 (3\%) & 0 \\\hline
    FR & 171 (85.5\%) & 14 (7\%) & 8 (4\%) & 7 (3.5\%) & 0 \\\hline
    PWLH & 184 (92\%) & 13 (6.5\%) & 3 (1.5\%) & 0 & 0 \\\hline
    PWLHInt & 173 (86.5\%) & 9 (4.5\%) & 13 (6.5\%) & 4 (2\%) & 1 (0.5\%) \\\hline
    GAMPS & 161 (80.5\%) & 16 (8\%) & 0 & 10 (5\%) & 13 (6.5\%) \\\hline
    SF & 199 (99.5\%) & 1 (0.5\%) & 0 & 0 & 0 \\\hline\hline
    Total & 1,420 (88.7\%) & 85 (5.3\%) & 41 (2.6\%) & 35 (2.2\%) & 19 (1.2\%) \\\hline
    \toprule[0.1mm]

    \end{tabular}

    \caption{RD between the \ows and \lows variants of each CAI.\\The results are aggregated by algorithm and the range to which the RD belongs.}

    \label{tabla:windows-comparison}

\end{center}

\end{table}


\vspace{-5pt}


For example, consider the results for algorithm CA, in the third row. The first column indicates that the RD is equal to 0 for exactly 172 (86\%) of the 200 evaluated pairs of CAIs for algorithm CA. The second column reveals that for 16 pairs of CAIs (8\%), the RD takes values greater than 0 and less than or equal to 1\%. The remaining three columns cover other ranges of RD. Notice that for every row (except the last one), the values add up to a total of 200, since we compare exactly 200 pairs of CAIs for each algorithm.


The last row of Table~\ref{tabla:windows-comparison} is obtained by adding the values of the previous rows, which combines the results for all algorithms. We notice that in 88.7\% of the total number of evaluated pairs of CAIs, the RD is equal to 0. In these cases, in fact, the \ows and the \lows coincide. In 97.8\% of the cases, the RD is less than or equal to 2\%. This means that, for the vast majority of CAI pairs, either the \ows and the \lows match or they yield roughly the same compression performance. This result suggests that we could fix in advance the window size parameter, for example by optimizing over a training set, without compromising the performance of the coding algorithm. This is relevant, since calculating the LOWS for a file is, in general, computationally expensive.


We notice that there are only 6 cases (0.4\%) in which the RD falls in the range (5, 11], most of which (5 cases) involve the algorithm PCA. The maximum value taken by RD (10.68\%) is obtained for the data type ``VWC" of the file ``vwc\_1203.dat.csv" of the dataset \datasetsst, with algorithm PCA, and error parameter $e=15$. In Figure~\ref{fig:window-compare-1203} we highlight this maximum value with a red circle. In this case, the \ows is 16 and the \lows is 8. According to these results, the performance of algorithm PCA seems to be more sensible to the window size parameter than the rest of the algorithms. Except for these few cases, we observe that, in general, the impact of using the \ows instead of the \lows on the compression performance of coding algorithms is rather small. Therefore, in the following section, in which we compare the algorithms performance, we always use the \owsns.


\clearpage

