
\section{Algorithms Performance}
\label{secX:codersmask}


In this section we compare the compression performance of the coding algorithms presented in Chapter~\ref{algo}, by encoding the various datasets introduced in Chapter~\ref{datasets}. We begin by comparing the encoders among each other and later we compare them with gzip, a popular lossless compression algorithm. We analyze the performance of the algorithms on datasets as a whole, rather than individual files, so we always apply Definitions~\ref{eq:coding-size-dataset}--\ref{def:relative-difference-dataset}.


We consider the scenario in which the parameters of a CAI are adjusted in the most convenient way, so that the compression ratio is minimized. For this reason, following the results obtained in Sections~\ref{secX:rendimiento-relativo} and~\ref{secX:windows}, we only consider the masking variants of the evaluated algorithms, and we always use the global window.


For each data type $z$ of each dataset $d$, and each coding algorithm $a \in A$ and threshold parameter $e \in E$, we calculate the compression ratio of $c_{<a, w_{global}^{*}, e>}$, as defined in (\ref{eq:compression-rate-dataset}), where $w_{global}^{*}=OWS(a, e, z, d)$ is the global window size. The following definition is useful for analyzing which CAI obtains the best compression result for a specific data type.


\begin{defcion}
\label{eq:bestcai}
Let $z$ be a data type of a certain dataset $d$. The \textit{best CAI} for a certain threshold parameter $e \in E$, is the CAI $c^{b}_{e} \in CI$ which minimizes the compression ratio, i.e.

\hspace{+10pt}$c^{b}_{e} = c_{<a^{b}, w^{b}, e>} \text{ such that } \tasacompresion(c^{b}_{e}, z, d) \leq \tasacompresion(c_{<a, w, e>}, z, d) \text{ for every } a \in A$, $w \in W$.

We denote $\tasacompresion(c^{b}_{e}, z, d)$ as the \textit{minimum CR}, $a^{b} \in A$ as the \textit{best algorithm}, and $w^{b} \in W$ as the \textit{best window parameter}.
\end{defcion}


Our experiments include a total of 21 data types, in 8 datasets. As an example, in Figure~\ref{fig:algo-per-1} we show the compression ratio and the window size parameter, as a function of the threshold, obtained for the data type ``SST" of the dataset \datasetelnino. For each threshold parameter $e \in E$, we made blue circles around the markers for the minimum CR value and the best window parameter (in the plots corresponding to the best algorithm). For instance, for $e=0$, the minimum CR is $32.91\%$, the best algorithm is PCA, and the best window parameter is 256. For the remaining seven threshold parameters, the blue circles indicate that the best algorithm is always APCA, and the best window parameter values range from 4 to 32.


\clearpage
\threefoursingle{5-ElNino-7}{\datasetelnino}{SST}{\ For each threshold parameter $e \in E$, we made blue circles around the markers for the minimum CR value and the best window parameter (in the plots corresponding to the best algorithm)}{\label{fig:algo-per-1}}
\clearpage


Table~\ref{experiments:mask-results-overview1} summarizes the compression performance results for each data type of each dataset, regarding the best CAI obtained for each threshold parameter. Each row contains information pertaining a certain data type. For example, the 13th row shows the results obtained for the data type ``SST" of the dataset \datasetelnino, which were analyzed through Figure~\ref{fig:algo-per-1} in the previous paragraph. For each threshold, the first column shows the minimum CR, the second column shows $\log_2$ of the best window parameter value, and the best algorithm is represented by a certain cell color. We use red font to mark the cases in which the minimum CR is greater than $100\%$. In these cases the best algorithm has a compression performance which is inferior to that of the trivial CAI, \textit{Base}.


We observe that there are three algorithms (PCA, APCA, and FR) which are the best algorithm for at least one of the 168 possible data type and threshold parameter combinations. APCA is the best algorithm in exactly 134 combinations ($80\%$), including every case in which $e \geq 10$, and most of the cases in which $e \in [1, 3, 5]$. PCA is the best algorithm in 31 combinations ($18\%$), including most of the lossless cases, while FR is the best algorithm in only 3 combinations ($2\%$), all of them concerning data type ``Speed" of the dataset \datasetwind.


\newcommand{\legendsone}{
\begin{tabular}{| C{1.5cm} | C{1.5cm} | C{1.5cm} |}
\hline
  \multicolumn{1}{|>{\centering\arraybackslash}m{1.5cm}|}{\cpca PCA} 
& \multicolumn{1}{>{\centering\arraybackslash}m{1.5cm}|}{\capca APCA} 
& \multicolumn{1}{>{\centering\arraybackslash}m{1.5cm}|}{\cfr FR}\\
\toprule[0.1mm]
\end{tabular}
\vspace{+30pt}

}

\newcommand{\legendstwo}{
\begin{tabular}{| C{1.5cm} | C{1.5cm} | C{1.5cm} | C{1.5cm} |}
\hline
  \multicolumn{1}{|>{\centering\arraybackslash}m{1.5cm}|}{\cgzip GZIP} 
& \multicolumn{1}{>{\centering\arraybackslash}m{1.5cm}|}{\cpca PCA} 
& \multicolumn{1}{>{\centering\arraybackslash}m{1.5cm}|}{\capca APCA} 
& \multicolumn{1}{>{\centering\arraybackslash}m{1.5cm}|}{\cfr FR}\\
\toprule[0.1mm]
\end{tabular}
\vspace{+30pt}

}

\newcommand{\legendsfive}{
\begin{tabular}{| C{1.5cm} | C{1.5cm} |}
\hline
  \multicolumn{1}{|>{\centering\arraybackslash}m{1.5cm}|}{\cpca PCA} 
& \multicolumn{1}{>{\centering\arraybackslash}m{1.5cm}|}{\capca APCA}\\
\toprule[0.1mm]
\end{tabular}
\vspace{+30pt}

}





\newcommand{\captionone}{
Compression performance results for each data type of each dataset, regarding the best CAI obtained for each threshold parameter. Each row contains information pertaining a certain data type. For each threshold, the first column shows the minimum CR, the second column shows $\log_2$ of the best window parameter value, and the best algorithm is represented by a certain cell color. We use red font to mark the cases in which the minimum CR is greater than $100\%$.
}

\newcommand{\captiontwo}{
\captionone
}
\begin{sidewaystable}[ht]
\newcommand{\cpca}{\cellcolor{cyan!20}}
\newcommand{\capca}{\cellcolor{green!20}}
\newcommand{\cfr}{\cellcolor{yellow!25}}
\newcommand{\cgzip}{\cellcolor{orange!20}}
\newcommand{\cpwlhint}{\cellcolor{violet!25}}
\newcommand{\cpwlh}{\cellcolor{violet!50}}
\newcommand{\cca}{\cellcolor{brown!20}}
\centering
\legendsone
\begin{tabular}{| l | l | c | c || c | c || c | c || c | c || c | c || c | c || c | c || c | c |}
\cline{3-18}
\multicolumn{1}{c}{}& \multicolumn{1}{c|}{} & \multicolumn{2}{c||}{e = 0} & \multicolumn{2}{c||}{e = 1} & \multicolumn{2}{c||}{e = 3} & \multicolumn{2}{c||}{e = 5} & \multicolumn{2}{c||}{e = 10} & \multicolumn{2}{c||}{e = 15} & \multicolumn{2}{c||}{e = 20} & \multicolumn{2}{c|}{e = 30} \\\hline
{Dataset} & {Data Type} & {\footnotesize CR} & {\footnotesize w} & {\footnotesize CR} & {\footnotesize w} & {\footnotesize CR} & {\footnotesize w} & {\footnotesize CR} & {\footnotesize w} & {\footnotesize CR} & {\footnotesize w} & {\footnotesize CR} & {\footnotesize w} & {\footnotesize CR} & {\footnotesize w} & {\footnotesize CR} & {\footnotesize w} \\\hline\hline


\clearpage

TODO: Analize the RD between PCA and APCA (and FR?). Table with data in next page.
% \clearpage
% \begin{sidewaystable}[ht]
\newcommand{\cpca}{\cellcolor{cyan!20}}
\newcommand{\capca}{\cellcolor{green!20}}
\newcommand{\cfr}{\cellcolor{yellow!25}}
\newcommand{\cgzip}{\cellcolor{orange!20}}
\centering
\legendsone
\begin{tabular}{| l | l | c | c || c | c || c | c || c | c || c | c || c | c || c | c || c | c |}
\cline{3-18}
\multicolumn{1}{c}{}& \multicolumn{1}{c|}{} & \multicolumn{2}{c||}{e = 0} & \multicolumn{2}{c||}{e = 1} & \multicolumn{2}{c||}{e = 3} & \multicolumn{2}{c||}{e = 5} & \multicolumn{2}{c||}{e = 10} & \multicolumn{2}{c||}{e = 15} & \multicolumn{2}{c||}{e = 20} & \multicolumn{2}{c|}{e = 30} \\\hline
{Dataset} & {Data Type} & {\footnotesize CR} & {\footnotesize w} & {\footnotesize CR} & {\footnotesize w} & {\footnotesize CR} & {\footnotesize w} & {\footnotesize CR} & {\footnotesize w} & {\footnotesize CR} & {\footnotesize w} & {\footnotesize CR} & {\footnotesize w} & {\footnotesize CR} & {\footnotesize w} & {\footnotesize CR} & {\footnotesize w} \\\hline\hline
{\datasetirkis} & {VWC} & {\cpca34.15} & {\cpca2} & {\cpca31.79} & {\cpca2} & {\cpca25.05} & {\cpca2} & {\cpca16.87} & {\cpca3} & {\cpca10.87} & {\cpca3} & {\cpca8.99} & {\cpca4} & {\cpca7.49} & {\cpca4} & {\cpca6.15} & {\cpca4} \\\hline
{\datasetsst} & {SST} & {\cpca60.84} & {\cpca8} & {\cpca41.79} & {\cpca2} & {\cpca26.98} & {\cpca2} & {\cpca22.26} & {\cpca2} & {\cpca14.61} & {\cpca3} & {\cpca11.91} & {\cpca3} & {\cpca10.39} & {\cpca4} & {\cpca8.03} & {\cpca4} \\\hline
{\datasetadcp} & {Vel} & {\cpca68.22} & {\cpca8} & {\cpca68.22} & {\cpca8} & {\cpca68.22} & {\cpca8} & {\cpca68.22} & {\cpca8} & {\cpca65.08} & {\cpca2} & {\cpca59.58} & {\cpca2} & {\cpca53.76} & {\cpca2} & {\cpca43.52} & {\cpca2} \\\hline
{\datasetsolar} & {GHI} & {\cpca77.65} & {\cpca2} & {\cpca76.51} & {\cpca2} & {\cpca75.65} & {\cpca2} & {\cpca74.95} & {\cpca2} & {\cpca72.61} & {\cpca2} & {\cpca70.42} & {\cpca2} & {\cpca68.26} & {\cpca2} & {\cpca64.0} & {\cpca2} \\\hline
{} & {DNI} & {\cpca75.93} & {\cpca2} & {\cpca74.35} & {\cpca2} & {\cpca72.75} & {\cpca2} & {\cpca71.49} & {\cpca2} & {\cpca69.79} & {\cpca2} & {\cpca68.18} & {\cpca2} & {\cpca66.14} & {\cpca2} & {\cpca62.13} & {\cpca2} \\\hline
{} & {DHI} & {\cpca77.66} & {\cpca2} & {\cpca77.43} & {\cpca2} & {\cpca76.11} & {\cpca2} & {\cpca75.33} & {\cpca2} & {\cpca73.98} & {\cpca2} & {\cpca72.64} & {\cpca2} & {\cpca70.74} & {\cpca2} & {\cpca66.68} & {\cpca2} \\\hline
{\datasetelnino} & {Lat} & {\cpca25.05} & {\cpca2} & {\cpca25.05} & {\cpca2} & {\cpca24.89} & {\cpca2} & {\cpca24.09} & {\cpca2} & {\cpca20.79} & {\cpca2} & {\cpca18.06} & {\cpca2} & {\cpca15.94} & {\cpca3} & {\cpca12.17} & {\cpca3} \\\hline
{} & {Long} & {\cpca27.24} & {\cpca2} & {\cpca26.83} & {\cpca2} & {\cpca21.87} & {\cpca2} & {\cpca20.44} & {\cpca2} & {\cpca16.42} & {\cpca3} & {\cpca13.59} & {\cpca3} & {\cpca11.33} & {\cpca3} & {\cpca8.13} & {\cpca3} \\\hline
{} & {Zonal Winds} & {\cpca31.46} & {\cpca8} & {\cpca31.46} & {\cpca8} & {\cpca31.46} & {\cpca8} & {\cpca31.46} & {\cpca8} & {\cpca31.46} & {\cpca8} & {\cpca30.82} & {\cpca2} & {\cpca29.48} & {\cpca2} & {\cpca25.92} & {\cpca2} \\\hline
{} & {Merid. Winds} & {\cpca31.46} & {\cpca8} & {\cpca31.46} & {\cpca8} & {\cpca31.46} & {\cpca8} & {\cpca31.46} & {\cpca8} & {\cpca31.46} & {\cpca8} & {\cpca31.38} & {\cpca2} & {\cpca30.6} & {\cpca2} & {\cpca28.19} & {\cpca2} \\\hline
{} & {Humidity} & {\cpca23.1} & {\cpca8} & {\cpca23.1} & {\cpca8} & {\cpca23.1} & {\cpca8} & {\cpca23.1} & {\cpca8} & {\cpca23.1} & {\cpca8} & {\cpca22.91} & {\cpca2} & {\cpca22.09} & {\cpca2} & {\cpca19.88} & {\cpca2} \\\hline
{} & {AirTemp} & {\cpca32.68} & {\cpca8} & {\cpca32.68} & {\cpca8} & {\cpca32.68} & {\cpca8} & {\cpca31.93} & {\cpca2} & {\cpca28.83} & {\cpca2} & {\cpca25.89} & {\cpca2} & {\cpca23.73} & {\cpca2} & {\cpca20.67} & {\cpca2} \\\hline
{} & {SST} & {\cpca32.91} & {\cpca8} & {\cpca32.91} & {\cpca8} & {\cpca31.05} & {\cpca2} & {\cpca28.46} & {\cpca2} & {\cpca22.91} & {\cpca2} & {\cpca18.98} & {\cpca2} & {\cpca16.1} & {\cpca2} & {\cpca12.68} & {\cpca2} \\\hline
{\datasethail} & {Lat} & {\cpca\color{red}100.04} & {\cpca8} & {\cpca\color{red}100.04} & {\cpca8} & {\cpca99.52} & {\cpca2} & {\cpca97.25} & {\cpca2} & {\cpca91.04} & {\cpca2} & {\cpca85.63} & {\cpca2} & {\cpca80.74} & {\cpca2} & {\cpca71.26} & {\cpca2} \\\hline
{} & {Long} & {\cpca\color{red}100.03} & {\cpca8} & {\cpca\color{red}100.03} & {\cpca8} & {\cpca98.88} & {\cpca2} & {\cpca95.54} & {\cpca2} & {\cpca86.75} & {\cpca2} & {\cpca79.54} & {\cpca2} & {\cpca73.26} & {\cpca2} & {\cpca64.17} & {\cpca2} \\\hline
{} & {Size} & {\cpca94.07} & {\cpca2} & {\cpca94.05} & {\cpca2} & {\cpca94.05} & {\cpca2} & {\cpca94.05} & {\cpca2} & {\cpca94.03} & {\cpca2} & {\cpca94.02} & {\cpca2} & {\cpca94.02} & {\cpca2} & {\cpca87.23} & {\cpca2} \\\hline
{\datasettornado} & {Lat} & {\cpca\color{red}100.05} & {\cpca8} & {\cpca99.76} & {\cpca2} & {\cpca93.9} & {\cpca2} & {\cpca89.21} & {\cpca2} & {\cpca79.92} & {\cpca2} & {\cpca73.98} & {\cpca2} & {\cpca69.53} & {\cpca2} & {\cpca61.64} & {\cpca2} \\\hline
{} & {Long} & {\cpca\color{red}100.11} & {\cpca8} & {\cpca98.88} & {\cpca2} & {\cpca89.2} & {\cpca2} & {\cpca83.18} & {\cpca2} & {\cpca73.42} & {\cpca2} & {\cpca67.64} & {\cpca2} & {\cpca62.02} & {\cpca2} & {\cpca54.99} & {\cpca2} \\\hline
{\datasetwind} & {Lat} & {\cpca\color{red}100.03} & {\cpca8} & {\cpca\color{red}100.03} & {\cpca8} & {\cpca99.29} & {\cpca2} & {\cpca96.6} & {\cpca2} & {\cpca89.73} & {\cpca2} & {\cpca83.81} & {\cpca2} & {\cpca79.48} & {\cpca2} & {\cpca71.4} & {\cpca2} \\\hline
{} & {Long} & {\cpca\color{red}100.03} & {\cpca8} & {\cpca\color{red}100.03} & {\cpca8} & {\cpca96.07} & {\cpca2} & {\cpca91.45} & {\cpca2} & {\cpca83.21} & {\cpca2} & {\cpca77.38} & {\cpca2} & {\cpca72.14} & {\cpca2} & {\cpca64.03} & {\cpca2} \\\hline
{} & {Speed} & {\cpca\color{red}100.04} & {\cpca8} & {\cpca67.73} & {\cpca2} & {\cpca55.23} & {\cpca2} & {\cpca44.21} & {\cpca2} & {\cpca37.59} & {\cpca2} & {\cpca35.05} & {\cpca3} & {\cpca32.79} & {\cpca3} & {\cpca30.46} & {\cpca3} \\\hline
\end{tabular}
\caption{\captionone}
\label{experiments:mask-results-overview1}
\end{sidewaystable}

% \clearpage
% \begin{sidewaystable}[ht]
\newcommand{\cpca}{\cellcolor{cyan!20}}
\newcommand{\capca}{\cellcolor{green!20}}
\newcommand{\cfr}{\cellcolor{yellow!25}}
\newcommand{\cgzip}{\cellcolor{orange!20}}
\centering
\legendsone
\begin{tabular}{| l | l | c | c || c | c || c | c || c | c || c | c || c | c || c | c || c | c |}
\cline{3-18}
\multicolumn{1}{c}{}& \multicolumn{1}{c|}{} & \multicolumn{2}{c||}{e = 0} & \multicolumn{2}{c||}{e = 1} & \multicolumn{2}{c||}{e = 3} & \multicolumn{2}{c||}{e = 5} & \multicolumn{2}{c||}{e = 10} & \multicolumn{2}{c||}{e = 15} & \multicolumn{2}{c||}{e = 20} & \multicolumn{2}{c|}{e = 30} \\\hline
{Dataset} & {Data Type} & {\footnotesize CR} & {\footnotesize w} & {\footnotesize CR} & {\footnotesize w} & {\footnotesize CR} & {\footnotesize w} & {\footnotesize CR} & {\footnotesize w} & {\footnotesize CR} & {\footnotesize w} & {\footnotesize CR} & {\footnotesize w} & {\footnotesize CR} & {\footnotesize w} & {\footnotesize CR} & {\footnotesize w} \\\hline\hline
{\datasetirkis} & {VWC} & {\capca20.32} & {\capca4} & {\capca18.35} & {\capca4} & {\capca12.37} & {\capca5} & {\capca6.77} & {\capca6} & {\capca3.07} & {\capca7} & {\capca2.22} & {\capca8} & {\capca1.71} & {\capca8} & {\capca1.21} & {\capca8} \\\hline
{\datasetsst} & {SST} & {\capca66.1} & {\capca2} & {\capca28.12} & {\capca3} & {\capca13.64} & {\capca5} & {\capca8.88} & {\capca6} & {\capca4.63} & {\capca7} & {\capca3.15} & {\capca8} & {\capca2.39} & {\capca8} & {\capca1.72} & {\capca8} \\\hline
{\datasetadcp} & {Vel} & {\capca77.52} & {\capca2} & {\capca74.51} & {\capca2} & {\capca66.8} & {\capca2} & {\capca61.07} & {\capca2} & {\capca48.44} & {\capca2} & {\capca40.9} & {\capca2} & {\capca34.9} & {\capca3} & {\capca25.93} & {\capca3} \\\hline
{\datasetsolar} & {GHI} & {\capca82.64} & {\capca2} & {\capca76.1} & {\capca3} & {\capca71.39} & {\capca4} & {\capca67.2} & {\capca4} & {\capca58.52} & {\capca4} & {\capca52.41} & {\capca4} & {\capca47.03} & {\capca4} & {\capca37.78} & {\capca4} \\\hline
{} & {DNI} & {\capca78.91} & {\capca2} & {\capca72.22} & {\capca4} & {\capca65.75} & {\capca4} & {\capca61.37} & {\capca4} & {\capca53.98} & {\capca4} & {\capca48.55} & {\capca4} & {\capca43.36} & {\capca4} & {\capca35.66} & {\capca4} \\\hline
{} & {DHI} & {\capca82.07} & {\capca2} & {\capca78.16} & {\capca2} & {\capca71.62} & {\capca4} & {\capca67.6} & {\capca4} & {\capca60.12} & {\capca4} & {\capca53.62} & {\capca4} & {\capca47.86} & {\capca4} & {\capca38.71} & {\capca4} \\\hline
{\datasetelnino} & {Lat} & {\capca15.96} & {\capca4} & {\capca15.96} & {\capca4} & {\capca15.82} & {\capca4} & {\capca15.11} & {\capca4} & {\capca12.34} & {\capca4} & {\capca9.89} & {\capca5} & {\capca8.61} & {\capca5} & {\capca5.76} & {\capca6} \\\hline
{} & {Long} & {\capca17.36} & {\capca3} & {\capca17.05} & {\capca4} & {\capca13.04} & {\capca4} & {\capca11.75} & {\capca5} & {\capca8.65} & {\capca6} & {\capca6.56} & {\capca6} & {\capca4.93} & {\capca7} & {\capca2.37} & {\capca8} \\\hline
{} & {Zonal Winds} & {\capca37.11} & {\capca2} & {\capca37.11} & {\capca2} & {\capca33.25} & {\capca2} & {\capca31.56} & {\capca2} & {\capca27.36} & {\capca2} & {\capca23.5} & {\capca2} & {\capca20.54} & {\capca2} & {\capca16.44} & {\capca3} \\\hline
{} & {Merid. Winds} & {\capca37.29} & {\capca2} & {\capca37.29} & {\capca2} & {\capca34.1} & {\capca2} & {\capca33.16} & {\capca2} & {\capca29.16} & {\capca2} & {\capca25.86} & {\capca2} & {\capca23.33} & {\capca2} & {\capca19.15} & {\capca2} \\\hline
{} & {Humidity} & {\capca26.39} & {\capca2} & {\capca26.29} & {\capca2} & {\capca25.38} & {\capca2} & {\capca23.42} & {\capca2} & {\capca20.51} & {\capca2} & {\capca18.14} & {\capca2} & {\capca16.01} & {\capca2} & {\capca12.94} & {\capca2} \\\hline
{} & {AirTemp} & {\capca36.2} & {\capca2} & {\capca34.96} & {\capca2} & {\capca30.33} & {\capca2} & {\capca27.39} & {\capca2} & {\capca22.42} & {\capca2} & {\capca19.24} & {\capca3} & {\capca16.76} & {\capca3} & {\capca13.31} & {\capca4} \\\hline
{} & {SST} & {\capca36.79} & {\capca2} & {\capca30.96} & {\capca2} & {\capca24.6} & {\capca2} & {\capca20.61} & {\capca2} & {\capca14.17} & {\capca3} & {\capca10.66} & {\capca4} & {\capca8.21} & {\capca4} & {\capca5.42} & {\capca5} \\\hline
{\datasethail} & {Lat} & {\capca\color{red}114.81} & {\capca2} & {\capca\color{red}102.05} & {\capca2} & {\capca89.83} & {\capca2} & {\capca82.62} & {\capca2} & {\capca71.49} & {\capca2} & {\capca64.62} & {\capca3} & {\capca57.49} & {\capca3} & {\capca46.75} & {\capca3} \\\hline
{} & {Long} & {\capca\color{red}114.14} & {\capca2} & {\capca\color{red}100.96} & {\capca2} & {\capca85.91} & {\capca2} & {\capca77.5} & {\capca2} & {\capca65.06} & {\capca2} & {\capca55.38} & {\capca3} & {\capca48.72} & {\capca3} & {\capca38.74} & {\capca4} \\\hline
{} & {Size} & {\capca80.61} & {\capca2} & {\capca80.59} & {\capca2} & {\capca80.59} & {\capca2} & {\capca80.58} & {\capca2} & {\capca80.56} & {\capca2} & {\capca80.53} & {\capca2} & {\capca80.52} & {\capca2} & {\capca64.35} & {\capca3} \\\hline
{\datasettornado} & {Lat} & {\capca\color{red}111.97} & {\capca2} & {\capca85.43} & {\capca2} & {\capca70.63} & {\capca2} & {\capca65.17} & {\capca2} & {\capca54.17} & {\capca3} & {\capca46.78} & {\capca3} & {\capca41.95} & {\capca4} & {\capca33.48} & {\capca4} \\\hline
{} & {Long} & {\capca\color{red}111.05} & {\capca2} & {\capca82.12} & {\capca2} & {\capca65.09} & {\capca2} & {\capca57.66} & {\capca3} & {\capca45.55} & {\capca3} & {\capca39.88} & {\capca4} & {\capca34.84} & {\capca4} & {\capca28.41} & {\capca4} \\\hline
{\datasetwind} & {Lat} & {\capca\color{red}113.34} & {\capca2} & {\capca\color{red}101.6} & {\capca2} & {\capca88.74} & {\capca2} & {\capca81.29} & {\capca2} & {\capca69.82} & {\capca2} & {\capca62.44} & {\capca3} & {\capca56.18} & {\capca3} & {\capca47.15} & {\capca3} \\\hline
{} & {Long} & {\capca\color{red}112.6} & {\capca2} & {\capca95.41} & {\capca2} & {\capca80.29} & {\capca2} & {\capca73.21} & {\capca2} & {\capca62.06} & {\capca3} & {\capca54.33} & {\capca3} & {\capca48.52} & {\capca3} & {\capca39.73} & {\capca4} \\\hline
{} & {Speed} & {\capca98.1} & {\capca2} & {\capca43.82} & {\capca3} & {\capca28.02} & {\capca4} & {\capca23.98} & {\capca4} & {\capca15.71} & {\capca5} & {\capca12.29} & {\capca6} & {\capca10.33} & {\capca6} & {\capca8.21} & {\capca6} \\\hline
\end{tabular}
\caption{\captionone}
\label{experiments:mask-results-overview1}
\end{sidewaystable}

\clearpage
\begin{sidewaystable}[ht]
\newcommand{\cpca}{\cellcolor{cyan!20}}
\newcommand{\capca}{\cellcolor{green!20}}
\newcommand{\cfr}{\cellcolor{yellow!25}}
\newcommand{\cgzip}{\cellcolor{orange!20}}
\centering
\legendsfive
\begin{tabular}{| l | l | c || c || c || c || c || c || c || c |}
\cline{3-10}
\multicolumn{1}{c}{}& \multicolumn{1}{c|}{} & \multicolumn{1}{c||}{e = 0} & \multicolumn{1}{c||}{e = 1} & \multicolumn{1}{c||}{e = 3} & \multicolumn{1}{c||}{e = 5} & \multicolumn{1}{c||}{e = 10} & \multicolumn{1}{c||}{e = 15} & \multicolumn{1}{c||}{e = 20} & \multicolumn{1}{c|}{e = 30} \\\hline
{Dataset} & {Data Type} & {\footnotesize RD} & {\footnotesize RD} & {\footnotesize RD} & {\footnotesize RD} & {\footnotesize RD} & {\footnotesize RD} & {\footnotesize RD} & {\footnotesize RD} \\\hline\hline
{\datasetirkis} & {VWC} & {\capca40.52} & {\capca42.28} & {\capca50.62} & {\capca59.86} & {\capca71.73} & {\capca75.33} & {\capca77.21} & {\capca80.28} \\\hline
{\datasetsst} & {SST} & {\cpca-8.64} & {\capca32.71} & {\capca49.42} & {\capca60.11} & {\capca68.29} & {\capca73.53} & {\capca76.95} & {\capca78.54} \\\hline
{\datasetadcp} & {Vel} & {\cpca-13.62} & {\cpca-9.22} & {\capca2.08} & {\capca10.48} & {\capca25.57} & {\capca31.35} & {\capca35.08} & {\capca40.41} \\\hline
{\datasetsolar} & {GHI} & {\cpca-6.42} & {\capca0.53} & {\capca5.62} & {\capca10.34} & {\capca19.4} & {\capca25.58} & {\capca31.11} & {\capca40.97} \\\hline
{} & {DNI} & {\cpca-3.92} & {\capca2.86} & {\capca9.62} & {\capca14.16} & {\capca22.65} & {\capca28.79} & {\capca34.44} & {\capca42.61} \\\hline
{} & {DHI} & {\cpca-5.68} & {\cpca-0.94} & {\capca5.9} & {\capca10.26} & {\capca18.74} & {\capca26.18} & {\capca32.34} & {\capca41.94} \\\hline
{\datasetelnino} & {Lat} & {\capca36.3} & {\capca36.3} & {\capca36.44} & {\capca37.28} & {\capca40.65} & {\capca45.2} & {\capca45.95} & {\capca52.65} \\\hline
{} & {Long} & {\capca36.28} & {\capca36.44} & {\capca40.39} & {\capca42.54} & {\capca47.32} & {\capca51.76} & {\capca56.53} & {\capca70.89} \\\hline
{} & {Zonal Winds} & {\cpca-17.98} & {\cpca-17.98} & {\cpca-5.69} & {\cpca-0.33} & {\capca13.02} & {\capca23.75} & {\capca30.33} & {\capca36.58} \\\hline
{} & {Merid. Winds} & {\cpca-18.54} & {\cpca-18.54} & {\cpca-8.42} & {\cpca-5.4} & {\capca7.32} & {\capca17.59} & {\capca23.75} & {\capca32.07} \\\hline
{} & {Humidity} & {\cpca-14.26} & {\cpca-13.8} & {\cpca-9.89} & {\cpca-1.37} & {\capca11.21} & {\capca20.82} & {\capca27.55} & {\capca34.9} \\\hline
{} & {AirTemp} & {\cpca-10.78} & {\cpca-6.97} & {\capca7.2} & {\capca14.24} & {\capca22.22} & {\capca25.69} & {\capca29.38} & {\capca35.61} \\\hline
{} & {SST} & {\cpca-11.79} & {\capca5.91} & {\capca20.8} & {\capca27.61} & {\capca38.16} & {\capca43.82} & {\capca49.02} & {\capca57.25} \\\hline
{\datasethail} & {Lat} & {\cpca-14.77} & {\cpca-2.01} & {\capca9.73} & {\capca15.05} & {\capca21.47} & {\capca24.54} & {\capca28.8} & {\capca34.4} \\\hline
{} & {Long} & {\cpca-14.11} & {\cpca-0.93} & {\capca13.12} & {\capca18.88} & {\capca25.0} & {\capca30.37} & {\capca33.5} & {\capca39.63} \\\hline
{} & {Size} & {\capca14.31} & {\capca14.32} & {\capca14.32} & {\capca14.32} & {\capca14.33} & {\capca14.35} & {\capca14.35} & {\capca26.22} \\\hline
{\datasettornado} & {Lat} & {\cpca-11.92} & {\capca14.36} & {\capca24.78} & {\capca26.95} & {\capca32.22} & {\capca36.77} & {\capca39.67} & {\capca45.69} \\\hline
{} & {Long} & {\cpca-10.92} & {\capca16.95} & {\capca27.03} & {\capca30.68} & {\capca37.96} & {\capca41.04} & {\capca43.82} & {\capca48.34} \\\hline
{\datasetwind} & {Lat} & {\cpca-13.31} & {\cpca-1.57} & {\capca10.63} & {\capca15.85} & {\capca22.19} & {\capca25.5} & {\capca29.32} & {\capca33.96} \\\hline
{} & {Long} & {\cpca-12.56} & {\capca4.62} & {\capca16.42} & {\capca19.94} & {\capca25.42} & {\capca29.79} & {\capca32.74} & {\capca37.96} \\\hline
{} & {Speed} & {\capca1.94} & {\capca35.31} & {\capca49.27} & {\capca45.76} & {\capca58.2} & {\capca64.93} & {\capca68.48} & {\capca73.05} \\\hline
\end{tabular}
\caption{\captionone}
\label{experiments:mask-results-overview3}
\end{sidewaystable}

\clearpage

TODO: Comparison with the gzip Algorithm. Table with data in next page.

\clearpage

\begin{table}[h]
\newcommand{\cpca}{\cellcolor{cyan!20}}
\newcommand{\capca}{\cellcolor{green!20}}
\newcommand{\cfr}{\cellcolor{yellow!25}}
\newcommand{\cgzip}{\cellcolor{orange!20}}
\centering
\legendstwo
\hspace*{-2.1cm}\begin{tabular}{| l | l | c | c || c | c || c | c || c | c || c | c || c | c || c | c || c | c |}
\cline{3-18}
\multicolumn{1}{c}{}& \multicolumn{1}{c|}{} & \multicolumn{2}{c||}{e = 0} & \multicolumn{2}{c||}{e = 1} & \multicolumn{2}{c||}{e = 3} & \multicolumn{2}{c||}{e = 5} & \multicolumn{2}{c||}{e = 10} & \multicolumn{2}{c||}{e = 15} & \multicolumn{2}{c||}{e = 20} & \multicolumn{2}{c|}{e = 30} \\\hline
{Dataset} & {Data Type} & {\footnotesize CR} & {\footnotesize w} & {\footnotesize CR} & {\footnotesize w} & {\footnotesize CR} & {\footnotesize w} & {\footnotesize CR} & {\footnotesize w} & {\footnotesize CR} & {\footnotesize w} & {\footnotesize CR} & {\footnotesize w} & {\footnotesize CR} & {\footnotesize w} & {\footnotesize CR} & {\footnotesize w} \\\hline\hline
{\datasetirkis} & {VWC} & {\cgzip0.13} & {\cgzip} & {\cgzip0.13} & {\cgzip} & {\capca0.12} & {\capca5} & {\capca0.07} & {\capca6} & {\capca0.03} & {\capca7} & {\capca0.02} & {\capca8} & {\capca0.02} & {\capca8} & {\capca0.01} & {\capca8} \\\hline
{\datasetsst} & {SST} & {\cgzip0.52} & {\cgzip} & {\capca0.28} & {\capca3} & {\capca0.14} & {\capca5} & {\capca0.09} & {\capca6} & {\capca0.05} & {\capca7} & {\capca0.03} & {\capca8} & {\capca0.02} & {\capca8} & {\capca0.02} & {\capca8} \\\hline
{\datasetadcp} & {Vel} & {\cgzip0.61} & {\cgzip} & {\cgzip0.61} & {\cgzip} & {\cgzip0.61} & {\cgzip} & {\capca0.61} & {\capca2} & {\capca0.48} & {\capca2} & {\capca0.41} & {\capca2} & {\capca0.35} & {\capca3} & {\capca0.26} & {\capca3} \\\hline
{\datasetsolar} & {GHI} & {\cgzip0.69} & {\cgzip} & {\cgzip0.69} & {\cgzip} & {\cgzip0.69} & {\cgzip} & {\capca0.67} & {\capca4} & {\capca0.59} & {\capca4} & {\capca0.52} & {\capca4} & {\capca0.47} & {\capca4} & {\capca0.38} & {\capca4} \\\hline
{} & {DNI} & {\cgzip0.67} & {\cgzip} & {\cgzip0.67} & {\cgzip} & {\capca0.66} & {\capca4} & {\capca0.61} & {\capca4} & {\capca0.54} & {\capca4} & {\capca0.49} & {\capca4} & {\capca0.43} & {\capca4} & {\capca0.36} & {\capca4} \\\hline
{} & {DHI} & {\cgzip0.61} & {\cgzip} & {\cgzip0.61} & {\cgzip} & {\cgzip0.61} & {\cgzip} & {\cgzip0.61} & {\cgzip} & {\capca0.60} & {\capca4} & {\capca0.54} & {\capca4} & {\capca0.48} & {\capca4} & {\capca0.39} & {\capca4} \\\hline
{\datasetelnino} & {Lat} & {\cgzip0.08} & {\cgzip} & {\cgzip0.08} & {\cgzip} & {\cgzip0.08} & {\cgzip} & {\cgzip0.08} & {\cgzip} & {\cgzip0.08} & {\cgzip} & {\cgzip0.08} & {\cgzip} & {\cgzip0.08} & {\cgzip} & {\capca0.06} & {\capca6} \\\hline
{} & {Long} & {\cgzip0.07} & {\cgzip} & {\cgzip0.07} & {\cgzip} & {\cgzip0.07} & {\cgzip} & {\cgzip0.07} & {\cgzip} & {\cgzip0.07} & {\cgzip} & {\capca0.07} & {\capca6} & {\capca0.05} & {\capca7} & {\capca0.02} & {\capca8} \\\hline
{} & {Zon.Wind} & {\cpca0.31} & {\cpca8} & {\cpca0.31} & {\cpca8} & {\cpca0.31} & {\cpca8} & {\cpca0.31} & {\cpca8} & {\capca0.27} & {\capca2} & {\capca0.24} & {\capca2} & {\capca0.21} & {\capca2} & {\capca0.16} & {\capca3} \\\hline
{} & {Mer.Wind} & {\cpca0.31} & {\cpca8} & {\cpca0.31} & {\cpca8} & {\cpca0.31} & {\cpca8} & {\cpca0.31} & {\cpca8} & {\capca0.29} & {\capca2} & {\capca0.26} & {\capca2} & {\capca0.23} & {\capca2} & {\capca0.19} & {\capca2} \\\hline
{} & {Humidity} & {\cpca0.23} & {\cpca8} & {\cpca0.23} & {\cpca8} & {\cpca0.23} & {\cpca8} & {\cpca0.23} & {\cpca8} & {\capca0.21} & {\capca2} & {\capca0.18} & {\capca2} & {\capca0.16} & {\capca2} & {\capca0.13} & {\capca2} \\\hline
{} & {Air Temp.} & {\cpca0.33} & {\cpca8} & {\cpca0.33} & {\cpca8} & {\capca0.30} & {\capca2} & {\capca0.27} & {\capca2} & {\capca0.22} & {\capca2} & {\capca0.19} & {\capca3} & {\capca0.17} & {\capca3} & {\capca0.13} & {\capca4} \\\hline
{} & {SST} & {\cgzip0.32} & {\cgzip} & {\capca0.31} & {\capca2} & {\capca0.25} & {\capca2} & {\capca0.21} & {\capca2} & {\capca0.14} & {\capca3} & {\capca0.11} & {\capca4} & {\capca0.08} & {\capca4} & {\capca0.05} & {\capca5} \\\hline
{\datasethail} & {Lat} & {\cpca1.00} & {\cpca8} & {\cpca1.00} & {\cpca8} & {\capca0.90} & {\capca2} & {\capca0.83} & {\capca2} & {\capca0.71} & {\capca2} & {\capca0.65} & {\capca3} & {\capca0.57} & {\capca3} & {\capca0.47} & {\capca3} \\\hline
{} & {Long} & {\cpca1.00} & {\cpca8} & {\cpca1.00} & {\cpca8} & {\capca0.86} & {\capca2} & {\capca0.78} & {\capca2} & {\capca0.65} & {\capca2} & {\capca0.55} & {\capca3} & {\capca0.49} & {\capca3} & {\capca0.39} & {\capca4} \\\hline
{} & {Size} & {\cgzip0.37} & {\cgzip} & {\cgzip0.37} & {\cgzip} & {\cgzip0.37} & {\cgzip} & {\cgzip0.37} & {\cgzip} & {\cgzip0.37} & {\cgzip} & {\cgzip0.37} & {\cgzip} & {\cgzip0.37} & {\cgzip} & {\cgzip0.37} & {\cgzip} \\\hline
{\datasettornado} & {Lat} & {\cpca1.00} & {\cpca8} & {\capca0.85} & {\capca2} & {\capca0.71} & {\capca2} & {\capca0.65} & {\capca2} & {\capca0.54} & {\capca3} & {\capca0.47} & {\capca3} & {\capca0.42} & {\capca4} & {\capca0.33} & {\capca4} \\\hline
{} & {Long} & {\cpca1.00} & {\cpca8} & {\capca0.82} & {\capca2} & {\capca0.65} & {\capca2} & {\capca0.58} & {\capca3} & {\capca0.46} & {\capca3} & {\capca0.40} & {\capca4} & {\capca0.35} & {\capca4} & {\capca0.28} & {\capca4} \\\hline
{\datasetwind} & {Lat} & {\cpca1.00} & {\cpca8} & {\cpca1.00} & {\cpca8} & {\capca0.89} & {\capca2} & {\capca0.81} & {\capca2} & {\capca0.70} & {\capca2} & {\capca0.62} & {\capca3} & {\capca0.56} & {\capca3} & {\capca0.47} & {\capca3} \\\hline
{} & {Long} & {\cpca1.00} & {\cpca8} & {\capca0.95} & {\capca2} & {\capca0.80} & {\capca2} & {\capca0.73} & {\capca2} & {\capca0.62} & {\capca3} & {\capca0.54} & {\capca3} & {\capca0.49} & {\capca3} & {\capca0.40} & {\capca4} \\\hline
{} & {Speed} & {\cfr0.65} & {\cfr4} & {\capca0.44} & {\capca3} & {\cfr0.26} & {\cfr6} & {\cfr0.17} & {\cfr7} & {\capca0.16} & {\capca5} & {\capca0.12} & {\capca6} & {\capca0.10} & {\capca6} & {\capca0.08} & {\capca6} \\\hline
\end{tabular}
\caption{\captiontwo}
\label{experiments:mask-results-overview2}
\end{table}

