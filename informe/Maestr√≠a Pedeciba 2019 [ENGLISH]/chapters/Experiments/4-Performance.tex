
\section{Algorithms Performance}
\label{secX:codersmask}


In this section we compare the compression performance of the coding algorithms presented in Chapter~\ref{algo}, by encoding the various datasets introduced in Chapter~\ref{datasets}. We begin by comparing the encoders among each other and later we compare them with gzip, a popular lossless compression algorithm. We analyze the performance of the algorithms on complete datasets (not individual files), so we always apply definitions~\ref{eq:coding-size-dataset}--\ref{def:relative-difference-dataset}. Following the results obtained in sections~\ref{secX:rendimiento-relativo} and~\ref{secX:windows}, we only consider the masking variants of the evaluated algorithms, and we always use the \owsns.


For each data type $z$ of each dataset $d$, and each coding algorithm $a \in A$ and threshold parameter $e \in E$, we calculate the CR of $c_{<a, w_{global}^{*}, e>}$, as defined in (\ref{eq:compression-rate-dataset}), where $w_{global}^{*}=\owsns(a, e, z, d)$. The following definition is useful for analyzing which CAI obtains the best compression result for a specific data type.


\begin{defcion}
\label{eq:bestcai}
Let $z$ be a data type of a certain dataset $d$, and let $e \in E$ be a threshold parameter. The \textit{best CAI} for $z, d, e$ is the CAI $c^{b}_{e}$ that minimizes the CR among all CAIs from~CI.
\end{defcion}


Our experiments include a total of 21 data types, in 8 datasets. As an example, in Figure~\ref{fig:algo-per-1} we show the CR and the OWS, as a function of the threshold, obtained for each algorithm for the data type ``SST" of the dataset \datasetelnino. For each threshold parameter $e \in E$, we use blue circles to highlight the markers for the minimum CR value and the best window size parameter (in the plots corresponding to the best algorithm). For instance, for $e=0$, the minimum CR is $0.33$, and the best algorithm is PCA with a window size equal to 256. For the remaining seven threshold parameters, the blue circles indicate that the best algorithm is always APCA with a window size ranging from 4 to 32.


\clearpage
\threefoursingle{5-ElNino-7}{\datasetelnino}{SST}{\ For each threshold parameter $e \in E$, we use blue circles to highlight the markers for the minimum CR value and the best window size parameter (in the plots corresponding to the best algorithm)}{\label{fig:algo-per-1}}
\clearpage


Table~\ref{experiments:mask-results-overview1} summarizes the compression performance results for each data type of each dataset. Each row contains information relative to certain data type. For example, the 13th row shows summarized results for the data type ``SST" of the dataset \datasetelnino, which were presented in more detail in Figure~\ref{fig:algo-per-1}. For each threshold, the first column shows the best (minimum) CR, and the second column shows the base-2 logarithm of the OWS for the best algorithm (the one that achieves this CR). The best algorithm itself is represented by a cell color code. We use red font to mark the cases in which the minimum CR is greater than $1$. In these cases, the best algorithm has a compression performance that is inferior to that of the trivial CAI, \textit{Base}.


We observe that there are three algorithms (PCA, APCA, and FR) that are best for at least one of the 168 possible data type and threshold parameter combinations. APCA is the best algorithm in exactly 134 combinations ($80\%$), including every case in which $e \geq 10$, and most of the cases in which $e \in [1, 3, 5]$. PCA is the best algorithm in 31 combinations ($18\%$), including most of the lossless cases, while FR is the best algorithm in only 3 combinations ($2\%$), all of them for data type ``Speed" of the dataset \datasetwind.


\newcommand{\legendsone}{
\begin{tabular}{| C{1.5cm} | C{1.5cm} | C{1.5cm} |}
\hline
  \multicolumn{1}{|>{\centering\arraybackslash}m{1.5cm}|}{\cpca PCA} 
& \multicolumn{1}{>{\centering\arraybackslash}m{1.5cm}|}{\capca APCA} 
& \multicolumn{1}{>{\centering\arraybackslash}m{1.5cm}|}{\cfr FR}\\
\toprule[0.1mm]
\end{tabular}
\vspace{+30pt}

}

\newcommand{\legendstwo}{
\begin{tabular}{| C{1.5cm} | C{1.5cm} | C{1.5cm} | C{1.5cm} |}
\hline
  \multicolumn{1}{|>{\centering\arraybackslash}m{1.5cm}|}{\cgzip GZIP} 
& \multicolumn{1}{>{\centering\arraybackslash}m{1.5cm}|}{\cpca PCA} 
& \multicolumn{1}{>{\centering\arraybackslash}m{1.5cm}|}{\capca APCA} 
& \multicolumn{1}{>{\centering\arraybackslash}m{1.5cm}|}{\cfr FR}\\
\toprule[0.1mm]
\end{tabular}
\vspace{+30pt}

}

\newcommand{\legendsfive}{
\begin{tabular}{| C{1.5cm} | C{1.5cm} |}
\hline
  \multicolumn{1}{|>{\centering\arraybackslash}m{1.5cm}|}{\cpca PCA} 
& \multicolumn{1}{>{\centering\arraybackslash}m{1.5cm}|}{\capca APCA}\\
\toprule[0.1mm]
\end{tabular}
\vspace{+30pt}

}





\newcommand{\captionone}{
Compression performance results for each data type of each dataset, regarding the best CAI obtained for each threshold parameter. Each row contains information pertaining a certain data type. For each threshold, the first column shows the minimum CR, the second column shows $\log_2$ of the best window parameter value, and the best algorithm is represented by a certain cell color. We use red font to mark the cases in which the minimum CR is greater than $100\%$.
}

\newcommand{\captiontwo}{
\captionone
}
\begin{sidewaystable}[ht]
\newcommand{\cpca}{\cellcolor{cyan!20}}
\newcommand{\capca}{\cellcolor{green!20}}
\newcommand{\cfr}{\cellcolor{yellow!25}}
\newcommand{\cgzip}{\cellcolor{orange!20}}
\newcommand{\cpwlhint}{\cellcolor{violet!25}}
\newcommand{\cpwlh}{\cellcolor{violet!50}}
\newcommand{\cca}{\cellcolor{brown!20}}
\centering
\legendsone
\begin{tabular}{| l | l | c | c || c | c || c | c || c | c || c | c || c | c || c | c || c | c |}
\cline{3-18}
\multicolumn{1}{c}{}& \multicolumn{1}{c|}{} & \multicolumn{2}{c||}{e = 0} & \multicolumn{2}{c||}{e = 1} & \multicolumn{2}{c||}{e = 3} & \multicolumn{2}{c||}{e = 5} & \multicolumn{2}{c||}{e = 10} & \multicolumn{2}{c||}{e = 15} & \multicolumn{2}{c||}{e = 20} & \multicolumn{2}{c|}{e = 30} \\\hline
{Dataset} & {Data Type} & {\footnotesize CR} & {\footnotesize w} & {\footnotesize CR} & {\footnotesize w} & {\footnotesize CR} & {\footnotesize w} & {\footnotesize CR} & {\footnotesize w} & {\footnotesize CR} & {\footnotesize w} & {\footnotesize CR} & {\footnotesize w} & {\footnotesize CR} & {\footnotesize w} & {\footnotesize CR} & {\footnotesize w} \\\hline\hline


\clearpage

TODO: Analize the RD between PCA and APCA (and FR?). Table with data in next page.
\clearpage
% \arrayrulecolor{blue}
\begin{sidewaystable}[ht]
\newcommand{\cpca}{\cellcolor{cyan!20}}
\newcommand{\capca}{\cellcolor{green!20}}
\newcommand{\cfr}{\cellcolor{yellow!25}}
\newcommand{\cgzip}{\cellcolor{orange!20}}
\newcommand{\cpwlhint}{\cellcolor{violet!25}}
\newcommand{\cpwlh}{\cellcolor{violet!50}}
\newcommand{\cca}{\cellcolor{brown!20}}
\centering
\legendstwelveone
{\color{blue}\begin{tabular}{| l | l | c | c || c | c || c | c || c | c || c | c || c | c || c | c || c | c |}
\cline{3-18}
\multicolumn{1}{c}{}& \multicolumn{1}{c|}{} & \multicolumn{2}{c||}{e = 0} & \multicolumn{2}{c||}{e = 1} & \multicolumn{2}{c||}{e = 3} & \multicolumn{2}{c||}{e = 5} & \multicolumn{2}{c||}{e = 10} & \multicolumn{2}{c||}{e = 15} & \multicolumn{2}{c||}{e = 20} & \multicolumn{2}{c|}{e = 30} \\\hline
{Dataset} & {Data Type} & {\footnotesize CR} & {\footnotesize w} & {\footnotesize CR} & {\footnotesize w} & {\footnotesize CR} & {\footnotesize w} & {\footnotesize CR} & {\footnotesize w} & {\footnotesize CR} & {\footnotesize w} & {\footnotesize CR} & {\footnotesize w} & {\footnotesize CR} & {\footnotesize w} & {\footnotesize CR} & {\footnotesize w} \\\hline\hline
{\datasetirkis} & {VWC} & {\capca20.9} & {\capca5} & {\capca18.67} & {\capca5} & {\capca12.71} & {\capca6} & {\capca6.9} & {\capca7} & {\capca3.09} & {\capca8} & {\capca2.29} & {\capca7} & {\capca1.84} & {\capca7} & {\capca1.41} & {\capca7} \\\hline
{\datasetsst} & {SST} & {\cpca60.85} & {\cpca7} & {\capca28.44} & {\capca4} & {\capca13.95} & {\capca4} & {\capca8.96} & {\capca5} & {\capca4.74} & {\capca6} & {\capca3.16} & {\capca7} & {\capca2.48} & {\capca7} & {\capca1.89} & {\capca7} \\\hline
{\datasetadcp} & {Vel} & {\cpca68.24} & {\cpca7} & {\cpca68.24} & {\cpca7} & {\cpca68.22} & {\cpca8} & {\capca65.23} & {\capca3} & {\capca50.8} & {\capca3} & {\capca41.49} & {\capca3} & {\capca35.9} & {\capca2} & {\capca26.27} & {\capca4} \\\hline
{\datasetsolar} & {GHI} & {\cpwlhint81.0} & {\cpwlhint4} & {\capca76.14} & {\capca4} & {\capca71.7} & {\capca3} & {\capca67.8} & {\capca3} & {\capca59.74} & {\capca3} & {\capca54.07} & {\capca3} & {\capca49.08} & {\capca3} & {\capca39.84} & {\capca5} \\\hline
{} & {DNI} & {\cpwlhint76.4} & {\cpwlhint4} & {\capca72.48} & {\capca3} & {\capca66.48} & {\capca3} & {\capca62.44} & {\capca3} & {\capca55.6} & {\capca3} & {\capca50.57} & {\capca3} & {\capca45.45} & {\capca5} & {\capca36.93} & {\capca5} \\\hline
{} & {DHI} & {\cpwlhint80.74} & {\cpwlhint4} & {\capca78.16} & {\capca2} & {\capca71.91} & {\capca3} & {\capca68.18} & {\capca3} & {\capca61.25} & {\capca3} & {\capca55.26} & {\capca3} & {\capca49.96} & {\capca3} & {\capca41.37} & {\capca5} \\\hline
{\datasetelnino} & {Lat} & {\capca16.12} & {\capca3} & {\capca16.12} & {\capca3} & {\capca16.01} & {\capca3} & {\capca15.42} & {\capca3} & {\capca12.43} & {\capca5} & {\capca10.08} & {\capca6} & {\capca8.66} & {\capca6} & {\capca5.82} & {\capca7} \\\hline
{} & {Long} & {\capca17.39} & {\capca4} & {\capca17.08} & {\capca3} & {\capca13.12} & {\capca5} & {\capca11.82} & {\capca4} & {\capca8.66} & {\capca5} & {\capca6.62} & {\capca7} & {\capca4.96} & {\capca6} & {\capca2.4} & {\capca7} \\\hline
{} & {Zonal Winds} & {\cpca31.47} & {\cpca7} & {\cpca31.47} & {\cpca7} & {\cpca31.47} & {\cpca7} & {\cpca31.47} & {\cpca7} & {\capca29.65} & {\capca3} & {\capca25.19} & {\capca3} & {\capca21.6} & {\capca3} & {\capca16.56} & {\capca2} \\\hline
{} & {Merid. Winds} & {\cpca31.47} & {\cpca7} & {\cpca31.47} & {\cpca7} & {\cpca31.47} & {\cpca7} & {\cpca31.47} & {\cpca7} & {\cpca31.46} & {\cpca8} & {\capca27.91} & {\capca3} & {\capca24.95} & {\capca3} & {\capca19.8} & {\capca3} \\\hline
{} & {Humidity} & {\cpca23.11} & {\cpca7} & {\cpca23.11} & {\cpca7} & {\cpca23.11} & {\cpca7} & {\cpca23.11} & {\cpca7} & {\capca22.1} & {\capca3} & {\capca19.4} & {\capca3} & {\capca16.89} & {\capca3} & {\capca13.05} & {\capca3} \\\hline
{} & {AirTemp} & {\cpca32.69} & {\cpca7} & {\cpca32.69} & {\cpca7} & {\capca32.34} & {\capca3} & {\capca28.97} & {\capca3} & {\capca22.97} & {\capca3} & {\capca19.51} & {\capca2} & {\capca17.24} & {\capca4} & {\capca13.42} & {\capca3} \\\hline
{} & {SST} & {\cpca32.92} & {\cpca7} & {\cpca32.91} & {\cpca8} & {\capca25.74} & {\capca3} & {\capca20.91} & {\capca3} & {\capca14.52} & {\capca4} & {\capca10.92} & {\capca3} & {\capca8.35} & {\capca5} & {\capca5.56} & {\capca6} \\\hline
{\datasethail} & {Lat} & {\cpca\color{red}100.07} & {\cpca7} & {\cpca\color{red}100.07} & {\cpca7} & {\capca95.57} & {\capca3} & {\capca87.08} & {\capca3} & {\capca73.35} & {\capca3} & {\capca64.91} & {\capca2} & {\capca59.08} & {\capca2} & {\capca47.11} & {\capca4} \\\hline
{} & {Long} & {\cpca\color{red}100.06} & {\cpca7} & {\cpca\color{red}100.06} & {\cpca7} & {\capca90.77} & {\capca3} & {\capca80.87} & {\capca3} & {\capca65.24} & {\capca3} & {\capca57.24} & {\capca4} & {\capca49.35} & {\capca4} & {\capca39.65} & {\capca3} \\\hline
{} & {Size} & {\capca83.97} & {\capca3} & {\capca83.95} & {\capca3} & {\capca83.95} & {\capca3} & {\capca83.94} & {\capca3} & {\capca83.91} & {\capca3} & {\capca83.88} & {\capca3} & {\capca83.87} & {\capca3} & {\capca64.6} & {\capca2} \\\hline
{\datasettornado} & {Lat} & {\cpca\color{red}100.08} & {\cpca7} & {\capca90.94} & {\capca3} & {\capca73.43} & {\capca3} & {\capca65.95} & {\capca3} & {\capca56.33} & {\capca2} & {\capca47.57} & {\capca4} & {\capca42.13} & {\capca3} & {\capca34.54} & {\capca5} \\\hline
{} & {Long} & {\cpca\color{red}100.14} & {\cpca7} & {\capca86.92} & {\capca3} & {\capca65.99} & {\capca3} & {\capca58.84} & {\capca2} & {\capca46.42} & {\capca4} & {\capca40.26} & {\capca3} & {\capca36.09} & {\capca5} & {\capca28.78} & {\capca5} \\\hline
{\datasetwind} & {Lat} & {\cpca\color{red}100.06} & {\cpca7} & {\cpca\color{red}100.06} & {\cpca7} & {\capca93.92} & {\capca3} & {\capca85.1} & {\capca3} & {\capca70.98} & {\capca3} & {\capca63.08} & {\capca2} & {\capca57.92} & {\capca4} & {\capca47.53} & {\capca4} \\\hline
{} & {Long} & {\cpca\color{red}100.06} & {\cpca7} & {\cpca\color{red}100.03} & {\cpca8} & {\capca83.85} & {\capca3} & {\capca75.03} & {\capca3} & {\capca63.08} & {\capca2} & {\capca55.61} & {\capca4} & {\capca48.84} & {\capca4} & {\capca40.81} & {\capca5} \\\hline
{} & {Speed} & {\cfr65.76} & {\cfr3} & {\capca45.45} & {\capca4} & {\cfr26.36} & {\cfr7} & {\cfr16.82} & {\cfr6} & {\capca16.1} & {\capca6} & {\capca12.39} & {\capca5} & {\capca10.63} & {\capca5} & {\capca8.39} & {\capca7} \\\hline
\end{tabular}}
\end{sidewaystable}

\clearpage
\begin{sidewaystable}[ht]
\newcommand{\cpca}{\cellcolor{cyan!20}}
\newcommand{\capca}{\cellcolor{green!20}}
\newcommand{\cfr}{\cellcolor{yellow!25}}
\newcommand{\cgzip}{\cellcolor{orange!20}}
\newcommand{\cpwlhint}{\cellcolor{violet!25}}
\newcommand{\cpwlh}{\cellcolor{violet!50}}
\newcommand{\cca}{\cellcolor{brown!20}}
\centering
\legendsfive
\begin{tabular}{| l | l | c | c || c | c || c | c || c | c || c | c || c | c || c | c || c | c |}
\cline{3-18}
\multicolumn{1}{c}{}& \multicolumn{1}{c|}{} & \multicolumn{2}{c||}{e = 0} & \multicolumn{2}{c||}{e = 1} & \multicolumn{2}{c||}{e = 3} & \multicolumn{2}{c||}{e = 5} & \multicolumn{2}{c||}{e = 10} & \multicolumn{2}{c||}{e = 15} & \multicolumn{2}{c||}{e = 20} & \multicolumn{2}{c|}{e = 30} \\\hline
{Dataset} & {Data Type} & {\footnotesize CR} & {\footnotesize w} & {\footnotesize CR} & {\footnotesize w} & {\footnotesize CR} & {\footnotesize w} & {\footnotesize CR} & {\footnotesize w} & {\footnotesize CR} & {\footnotesize w} & {\footnotesize CR} & {\footnotesize w} & {\footnotesize CR} & {\footnotesize w} & {\footnotesize CR} & {\footnotesize w} \\\hline\hline
{\datasetirkis} & {VWC} & {\cpwlhint27.81} & {\cpwlhint5} & {\cpwlhint27.81} & {\cpwlhint5} & {\cca21.21} & {\cca5} & {\cca12.99} & {\cca6} & {\cfr6.8} & {\cfr7} & {\cfr4.97} & {\cfr8} & {\cpwlhint3.44} & {\cpwlhint8} & {\cpwlhint1.63} & {\cpwlhint8} \\\hline
{\datasetsst} & {SST} & {\cpwlhint63.83} & {\cpwlhint2} & {\cpwlhint33.37} & {\cpwlhint4} & {\cpwlhint15.39} & {\cpwlhint6} & {\cpwlhint9.73} & {\cpwlhint7} & {\cpwlhint4.92} & {\cpwlhint8} & {\cpwlhint3.33} & {\cpwlhint8} & {\cpwlhint2.63} & {\cpwlhint8} & {\cpwlhint2.03} & {\cpwlhint8} \\\hline
{\datasetadcp} & {Vel} & {\cpwlhint73.3} & {\cpwlhint2} & {\cpwlhint72.88} & {\cpwlhint2} & {\cpca68.22} & {\cpca8} & {\cpwlhint66.83} & {\cpwlhint2} & {\cpwlhint59.28} & {\cpwlhint3} & {\cpwlhint51.1} & {\cpwlhint3} & {\cpwlhint44.7} & {\cpwlhint3} & {\cpwlhint34.02} & {\cpwlhint4} \\\hline
{\datasetsolar} & {GHI} & {\cpwlhint81.0} & {\cpwlhint4} & {\cpca76.51} & {\cpca2} & {\cpca75.65} & {\cpca2} & {\cpwlhint73.26} & {\cpwlhint4} & {\cpwlhint66.97} & {\cpwlhint4} & {\cpwlhint62.83} & {\cpwlhint4} & {\cpwlhint59.88} & {\cpwlhint4} & {\cca54.06} & {\cca4} \\\hline
{} & {DNI} & {\cpwlhint76.4} & {\cpwlhint4} & {\cpwlhint74.29} & {\cpwlhint4} & {\cpwlhint72.55} & {\cpwlhint4} & {\cpwlhint70.86} & {\cpwlhint4} & {\cpwlhint67.21} & {\cpwlhint4} & {\cpwlhint64.23} & {\cpwlhint4} & {\cpwlhint61.64} & {\cpwlhint4} & {\cpwlhint57.22} & {\cpwlhint4} \\\hline
{} & {DHI} & {\cpwlhint80.74} & {\cpwlhint4} & {\capca78.16} & {\capca2} & {\cpwlhint75.57} & {\cpwlhint4} & {\cpwlhint72.78} & {\cpwlhint4} & {\cpwlhint67.69} & {\cpwlhint4} & {\cpwlhint63.49} & {\cpwlhint4} & {\cpwlhint60.08} & {\cpwlhint4} & {\cpwlhint54.61} & {\cpwlhint4} \\\hline
{\datasetelnino} & {Lat} & {\cpwlhint22.54} & {\cpwlhint4} & {\cpwlhint22.54} & {\cpwlhint4} & {\cpwlhint22.54} & {\cpwlhint4} & {\cpwlhint22.54} & {\cpwlhint4} & {\cpca20.79} & {\cpca2} & {\cpca18.06} & {\cpca2} & {\cpca15.94} & {\cpca3} & {\cpca12.17} & {\cpca3} \\\hline
{} & {Long} & {\cpwlhint24.69} & {\cpwlhint4} & {\cpwlhint24.48} & {\cpwlhint4} & {\cca21.77} & {\cca4} & {\cca19.99} & {\cca4} & {\cca15.31} & {\cca5} & {\cca12.43} & {\cca6} & {\cca10.02} & {\cca6} & {\cpwlh5.8} & {\cpwlh8} \\\hline
{} & {Zonal Winds} & {\cpwlhint34.51} & {\cpwlhint2} & {\cpwlhint34.51} & {\cpwlhint2} & {\capca33.25} & {\capca2} & {\capca31.56} & {\capca2} & {\cpwlhint31.12} & {\cpwlhint2} & {\cpwlhint28.38} & {\cpwlhint2} & {\cpwlhint26.04} & {\cpwlhint3} & {\cpwlhint20.98} & {\cpwlhint3} \\\hline
{} & {Merid. Winds} & {\cpwlhint34.57} & {\cpwlhint2} & {\cpwlhint34.57} & {\cpwlhint2} & {\capca34.1} & {\capca2} & {\capca33.16} & {\capca2} & {\cpca31.46} & {\cpca8} & {\cpwlhint30.04} & {\cpwlhint2} & {\cpwlhint28.33} & {\cpwlhint2} & {\cpwlhint24.5} & {\cpwlhint3} \\\hline
{} & {Humidity} & {\cpwlhint25.12} & {\cpwlhint2} & {\cpwlhint25.12} & {\cpwlhint2} & {\cpwlhint24.98} & {\cpwlhint2} & {\capca23.42} & {\capca2} & {\cpwlhint22.38} & {\cpwlhint2} & {\cpwlhint20.88} & {\cpwlhint2} & {\cpwlhint19.34} & {\cpwlhint3} & {\cpwlhint15.85} & {\cpwlhint3} \\\hline
{} & {AirTemp} & {\cpwlhint34.95} & {\cpwlhint2} & {\cpwlhint34.71} & {\cpwlhint2} & {\cpwlhint32.44} & {\cpwlhint2} & {\cpwlhint30.31} & {\cpwlhint2} & {\cpwlhint25.93} & {\cpwlhint3} & {\cpwlhint22.76} & {\cpwlhint4} & {\cpwlhint20.04} & {\cpwlhint4} & {\cpwlhint16.43} & {\cpwlhint5} \\\hline
{} & {SST} & {\cpwlhint35.01} & {\cpwlhint2} & {\cpca32.91} & {\cpca8} & {\cpwlhint29.13} & {\cpwlhint3} & {\cpwlhint24.5} & {\cpwlhint3} & {\cpwlhint17.19} & {\cpwlhint4} & {\cpwlhint13.0} & {\cpwlhint5} & {\cpwlhint10.0} & {\cpwlhint5} & {\cpwlhint6.5} & {\cpwlhint6} \\\hline
{\datasethail} & {Lat} & {\cpwlhint\color{red}108.59} & {\cpwlhint2} & {\capca\color{red}102.05} & {\capca2} & {\cpwlhint99.1} & {\cpwlhint2} & {\cpwlhint95.09} & {\cpwlhint2} & {\cpwlhint86.27} & {\cpwlhint3} & {\cpwlhint79.57} & {\cpwlhint3} & {\cpwlhint73.8} & {\cpwlhint3} & {\cpwlhint63.72} & {\cpwlhint4} \\\hline
{} & {Long} & {\cpwlhint\color{red}107.97} & {\cpwlhint2} & {\capca\color{red}100.96} & {\capca2} & {\cpwlhint95.88} & {\cpwlhint2} & {\cpwlhint90.95} & {\cpwlhint3} & {\cpwlhint79.94} & {\cpwlhint3} & {\cpwlhint72.14} & {\cpwlhint3} & {\cpwlhint65.42} & {\cpwlhint4} & {\cpwlhint55.75} & {\cpwlhint4} \\\hline
{} & {Size} & {\cpca94.07} & {\cpca2} & {\cpca94.05} & {\cpca2} & {\cpca94.05} & {\cpca2} & {\cpca94.05} & {\cpca2} & {\cpca94.03} & {\cpca2} & {\cpca94.02} & {\cpca2} & {\cpwlhint92.76} & {\cpwlhint2} & {\cpwlhint85.3} & {\cpwlhint3} \\\hline
{\datasettornado} & {Lat} & {\cpwlhint\color{red}108.2} & {\cpwlhint2} & {\cpwlhint95.5} & {\cpwlhint2} & {\cpwlhint84.04} & {\cpwlhint3} & {\cpwlhint76.63} & {\cpwlhint3} & {\cpwlhint66.4} & {\cpwlhint4} & {\cpwlhint58.48} & {\cpwlhint4} & {\cpwlhint53.72} & {\cpwlhint4} & {\cpwlhint45.59} & {\cpwlhint4} \\\hline
{} & {Long} & {\cpwlhint\color{red}107.67} & {\cpwlhint2} & {\cpwlhint90.47} & {\cpwlhint2} & {\cpwlhint74.82} & {\cpwlhint3} & {\cpwlhint68.49} & {\cpwlhint3} & {\cpwlhint55.2} & {\cpwlhint4} & {\cpwlhint48.66} & {\cpwlhint4} & {\cpwlhint44.37} & {\cpwlhint5} & {\cpwlhint37.56} & {\cpwlhint5} \\\hline
{\datasetwind} & {Lat} & {\cpwlhint\color{red}108.36} & {\cpwlhint2} & {\capca\color{red}101.6} & {\capca2} & {\cpca99.29} & {\cpca2} & {\cpca96.6} & {\cpca2} & {\cpca89.73} & {\cpca2} & {\cpca83.81} & {\cpca2} & {\cpwlhint79.46} & {\cpwlhint3} & {\cpca71.4} & {\cpca2} \\\hline
{} & {Long} & {\cpwlhint\color{red}107.78} & {\cpwlhint2} & {\cpca\color{red}100.03} & {\cpca8} & {\cpwlhint95.67} & {\cpwlhint2} & {\cpca91.45} & {\cpca2} & {\cpca83.21} & {\cpca2} & {\cpca77.38} & {\cpca2} & {\cpca72.14} & {\cpca2} & {\cpca64.03} & {\cpca2} \\\hline
{} & {Speed} & {\cpwlhint68.31} & {\cpwlhint3} & {\cfr58.97} & {\cfr4} & {\capca28.02} & {\capca4} & {\capca23.98} & {\capca4} & {\cfr16.51} & {\cfr7} & {\cfr16.32} & {\cfr7} & {\cfr16.12} & {\cfr7} & {\cfr15.78} & {\cfr7} \\\hline
\end{tabular}
\caption{\captionone}
\label{experiments:mask-results-overview1}
\end{sidewaystable}

\clearpage
\begin{sidewaystable}[ht]
\newcommand{\cpca}{\cellcolor{cyan!20}}
\newcommand{\capca}{\cellcolor{green!20}}
\newcommand{\cfr}{\cellcolor{yellow!25}}
\newcommand{\cgzip}{\cellcolor{orange!20}}
\centering
\legendsone
\begin{tabular}{| l | l | c | c || c | c || c | c || c | c || c | c || c | c || c | c || c | c |}
\cline{3-18}
\multicolumn{1}{c}{}& \multicolumn{1}{c|}{} & \multicolumn{2}{c||}{e = 0} & \multicolumn{2}{c||}{e = 1} & \multicolumn{2}{c||}{e = 3} & \multicolumn{2}{c||}{e = 5} & \multicolumn{2}{c||}{e = 10} & \multicolumn{2}{c||}{e = 15} & \multicolumn{2}{c||}{e = 20} & \multicolumn{2}{c|}{e = 30} \\\hline
{Dataset} & {Data Type} & {\footnotesize CR} & {\footnotesize w} & {\footnotesize CR} & {\footnotesize w} & {\footnotesize CR} & {\footnotesize w} & {\footnotesize CR} & {\footnotesize w} & {\footnotesize CR} & {\footnotesize w} & {\footnotesize CR} & {\footnotesize w} & {\footnotesize CR} & {\footnotesize w} & {\footnotesize CR} & {\footnotesize w} \\\hline\hline
{\datasetirkis} & {VWC} & {\cpca34.15} & {\cpca2} & {\cpca31.79} & {\cpca2} & {\cpca25.05} & {\cpca2} & {\cpca16.87} & {\cpca3} & {\cpca10.87} & {\cpca3} & {\cpca8.99} & {\cpca4} & {\cpca7.49} & {\cpca4} & {\cpca6.15} & {\cpca4} \\\hline
{\datasetsst} & {SST} & {\cpca60.84} & {\cpca8} & {\cpca41.79} & {\cpca2} & {\cpca26.98} & {\cpca2} & {\cpca22.26} & {\cpca2} & {\cpca14.61} & {\cpca3} & {\cpca11.91} & {\cpca3} & {\cpca10.39} & {\cpca4} & {\cpca8.03} & {\cpca4} \\\hline
{\datasetadcp} & {Vel} & {\cpca68.22} & {\cpca8} & {\cpca68.22} & {\cpca8} & {\cpca68.22} & {\cpca8} & {\cpca68.22} & {\cpca8} & {\cpca65.08} & {\cpca2} & {\cpca59.58} & {\cpca2} & {\cpca53.76} & {\cpca2} & {\cpca43.52} & {\cpca2} \\\hline
{\datasetsolar} & {GHI} & {\cpca77.65} & {\cpca2} & {\cpca76.51} & {\cpca2} & {\cpca75.65} & {\cpca2} & {\cpca74.95} & {\cpca2} & {\cpca72.61} & {\cpca2} & {\cpca70.42} & {\cpca2} & {\cpca68.26} & {\cpca2} & {\cpca64.0} & {\cpca2} \\\hline
{} & {DNI} & {\cpca75.93} & {\cpca2} & {\cpca74.35} & {\cpca2} & {\cpca72.75} & {\cpca2} & {\cpca71.49} & {\cpca2} & {\cpca69.79} & {\cpca2} & {\cpca68.18} & {\cpca2} & {\cpca66.14} & {\cpca2} & {\cpca62.13} & {\cpca2} \\\hline
{} & {DHI} & {\cpca77.66} & {\cpca2} & {\cpca77.43} & {\cpca2} & {\cpca76.11} & {\cpca2} & {\cpca75.33} & {\cpca2} & {\cpca73.98} & {\cpca2} & {\cpca72.64} & {\cpca2} & {\cpca70.74} & {\cpca2} & {\cpca66.68} & {\cpca2} \\\hline
{\datasetelnino} & {Lat} & {\cpca25.05} & {\cpca2} & {\cpca25.05} & {\cpca2} & {\cpca24.89} & {\cpca2} & {\cpca24.09} & {\cpca2} & {\cpca20.79} & {\cpca2} & {\cpca18.06} & {\cpca2} & {\cpca15.94} & {\cpca3} & {\cpca12.17} & {\cpca3} \\\hline
{} & {Long} & {\cpca27.24} & {\cpca2} & {\cpca26.83} & {\cpca2} & {\cpca21.87} & {\cpca2} & {\cpca20.44} & {\cpca2} & {\cpca16.42} & {\cpca3} & {\cpca13.59} & {\cpca3} & {\cpca11.33} & {\cpca3} & {\cpca8.13} & {\cpca3} \\\hline
{} & {Zonal Winds} & {\cpca31.46} & {\cpca8} & {\cpca31.46} & {\cpca8} & {\cpca31.46} & {\cpca8} & {\cpca31.46} & {\cpca8} & {\cpca31.46} & {\cpca8} & {\cpca30.82} & {\cpca2} & {\cpca29.48} & {\cpca2} & {\cpca25.92} & {\cpca2} \\\hline
{} & {Merid. Winds} & {\cpca31.46} & {\cpca8} & {\cpca31.46} & {\cpca8} & {\cpca31.46} & {\cpca8} & {\cpca31.46} & {\cpca8} & {\cpca31.46} & {\cpca8} & {\cpca31.38} & {\cpca2} & {\cpca30.6} & {\cpca2} & {\cpca28.19} & {\cpca2} \\\hline
{} & {Humidity} & {\cpca23.1} & {\cpca8} & {\cpca23.1} & {\cpca8} & {\cpca23.1} & {\cpca8} & {\cpca23.1} & {\cpca8} & {\cpca23.1} & {\cpca8} & {\cpca22.91} & {\cpca2} & {\cpca22.09} & {\cpca2} & {\cpca19.88} & {\cpca2} \\\hline
{} & {AirTemp} & {\cpca32.68} & {\cpca8} & {\cpca32.68} & {\cpca8} & {\cpca32.68} & {\cpca8} & {\cpca31.93} & {\cpca2} & {\cpca28.83} & {\cpca2} & {\cpca25.89} & {\cpca2} & {\cpca23.73} & {\cpca2} & {\cpca20.67} & {\cpca2} \\\hline
{} & {SST} & {\cpca32.91} & {\cpca8} & {\cpca32.91} & {\cpca8} & {\cpca31.05} & {\cpca2} & {\cpca28.46} & {\cpca2} & {\cpca22.91} & {\cpca2} & {\cpca18.98} & {\cpca2} & {\cpca16.1} & {\cpca2} & {\cpca12.68} & {\cpca2} \\\hline
{\datasethail} & {Lat} & {\cpca\color{red}100.04} & {\cpca8} & {\cpca\color{red}100.04} & {\cpca8} & {\cpca99.52} & {\cpca2} & {\cpca97.25} & {\cpca2} & {\cpca91.04} & {\cpca2} & {\cpca85.63} & {\cpca2} & {\cpca80.74} & {\cpca2} & {\cpca71.26} & {\cpca2} \\\hline
{} & {Long} & {\cpca\color{red}100.03} & {\cpca8} & {\cpca\color{red}100.03} & {\cpca8} & {\cpca98.88} & {\cpca2} & {\cpca95.54} & {\cpca2} & {\cpca86.75} & {\cpca2} & {\cpca79.54} & {\cpca2} & {\cpca73.26} & {\cpca2} & {\cpca64.17} & {\cpca2} \\\hline
{} & {Size} & {\cpca94.07} & {\cpca2} & {\cpca94.05} & {\cpca2} & {\cpca94.05} & {\cpca2} & {\cpca94.05} & {\cpca2} & {\cpca94.03} & {\cpca2} & {\cpca94.02} & {\cpca2} & {\cpca94.02} & {\cpca2} & {\cpca87.23} & {\cpca2} \\\hline
{\datasettornado} & {Lat} & {\cpca\color{red}100.05} & {\cpca8} & {\cpca99.76} & {\cpca2} & {\cpca93.9} & {\cpca2} & {\cpca89.21} & {\cpca2} & {\cpca79.92} & {\cpca2} & {\cpca73.98} & {\cpca2} & {\cpca69.53} & {\cpca2} & {\cpca61.64} & {\cpca2} \\\hline
{} & {Long} & {\cpca\color{red}100.11} & {\cpca8} & {\cpca98.88} & {\cpca2} & {\cpca89.2} & {\cpca2} & {\cpca83.18} & {\cpca2} & {\cpca73.42} & {\cpca2} & {\cpca67.64} & {\cpca2} & {\cpca62.02} & {\cpca2} & {\cpca54.99} & {\cpca2} \\\hline
{\datasetwind} & {Lat} & {\cpca\color{red}100.03} & {\cpca8} & {\cpca\color{red}100.03} & {\cpca8} & {\cpca99.29} & {\cpca2} & {\cpca96.6} & {\cpca2} & {\cpca89.73} & {\cpca2} & {\cpca83.81} & {\cpca2} & {\cpca79.48} & {\cpca2} & {\cpca71.4} & {\cpca2} \\\hline
{} & {Long} & {\cpca\color{red}100.03} & {\cpca8} & {\cpca\color{red}100.03} & {\cpca8} & {\cpca96.07} & {\cpca2} & {\cpca91.45} & {\cpca2} & {\cpca83.21} & {\cpca2} & {\cpca77.38} & {\cpca2} & {\cpca72.14} & {\cpca2} & {\cpca64.03} & {\cpca2} \\\hline
{} & {Speed} & {\cpca\color{red}100.04} & {\cpca8} & {\cpca67.73} & {\cpca2} & {\cpca55.23} & {\cpca2} & {\cpca44.21} & {\cpca2} & {\cpca37.59} & {\cpca2} & {\cpca35.05} & {\cpca3} & {\cpca32.79} & {\cpca3} & {\cpca30.46} & {\cpca3} \\\hline
\end{tabular}
\caption{\captionone}
\label{experiments:mask-results-overview1}
\end{sidewaystable}

\clearpage
\begin{sidewaystable}[ht]
\newcommand{\cpca}{\cellcolor{cyan!20}}
\newcommand{\capca}{\cellcolor{green!20}}
\newcommand{\cfr}{\cellcolor{yellow!25}}
\newcommand{\cgzip}{\cellcolor{orange!20}}
\centering
\legendsone
\begin{tabular}{| l | l | c | c || c | c || c | c || c | c || c | c || c | c || c | c || c | c |}
\cline{3-18}
\multicolumn{1}{c}{}& \multicolumn{1}{c|}{} & \multicolumn{2}{c||}{e = 0} & \multicolumn{2}{c||}{e = 1} & \multicolumn{2}{c||}{e = 3} & \multicolumn{2}{c||}{e = 5} & \multicolumn{2}{c||}{e = 10} & \multicolumn{2}{c||}{e = 15} & \multicolumn{2}{c||}{e = 20} & \multicolumn{2}{c|}{e = 30} \\\hline
{Dataset} & {Data Type} & {\footnotesize CR} & {\footnotesize w} & {\footnotesize CR} & {\footnotesize w} & {\footnotesize CR} & {\footnotesize w} & {\footnotesize CR} & {\footnotesize w} & {\footnotesize CR} & {\footnotesize w} & {\footnotesize CR} & {\footnotesize w} & {\footnotesize CR} & {\footnotesize w} & {\footnotesize CR} & {\footnotesize w} \\\hline\hline
{\datasetirkis} & {VWC} & {\capca20.32} & {\capca4} & {\capca18.35} & {\capca4} & {\capca12.37} & {\capca5} & {\capca6.77} & {\capca6} & {\capca3.07} & {\capca7} & {\capca2.22} & {\capca8} & {\capca1.71} & {\capca8} & {\capca1.21} & {\capca8} \\\hline
{\datasetsst} & {SST} & {\capca66.1} & {\capca2} & {\capca28.12} & {\capca3} & {\capca13.64} & {\capca5} & {\capca8.88} & {\capca6} & {\capca4.63} & {\capca7} & {\capca3.15} & {\capca8} & {\capca2.39} & {\capca8} & {\capca1.72} & {\capca8} \\\hline
{\datasetadcp} & {Vel} & {\capca77.52} & {\capca2} & {\capca74.51} & {\capca2} & {\capca66.8} & {\capca2} & {\capca61.07} & {\capca2} & {\capca48.44} & {\capca2} & {\capca40.9} & {\capca2} & {\capca34.9} & {\capca3} & {\capca25.93} & {\capca3} \\\hline
{\datasetsolar} & {GHI} & {\capca82.64} & {\capca2} & {\capca76.1} & {\capca3} & {\capca71.39} & {\capca4} & {\capca67.2} & {\capca4} & {\capca58.52} & {\capca4} & {\capca52.41} & {\capca4} & {\capca47.03} & {\capca4} & {\capca37.78} & {\capca4} \\\hline
{} & {DNI} & {\capca78.91} & {\capca2} & {\capca72.22} & {\capca4} & {\capca65.75} & {\capca4} & {\capca61.37} & {\capca4} & {\capca53.98} & {\capca4} & {\capca48.55} & {\capca4} & {\capca43.36} & {\capca4} & {\capca35.66} & {\capca4} \\\hline
{} & {DHI} & {\capca82.07} & {\capca2} & {\capca78.16} & {\capca2} & {\capca71.62} & {\capca4} & {\capca67.6} & {\capca4} & {\capca60.12} & {\capca4} & {\capca53.62} & {\capca4} & {\capca47.86} & {\capca4} & {\capca38.71} & {\capca4} \\\hline
{\datasetelnino} & {Lat} & {\capca15.96} & {\capca4} & {\capca15.96} & {\capca4} & {\capca15.82} & {\capca4} & {\capca15.11} & {\capca4} & {\capca12.34} & {\capca4} & {\capca9.89} & {\capca5} & {\capca8.61} & {\capca5} & {\capca5.76} & {\capca6} \\\hline
{} & {Long} & {\capca17.36} & {\capca3} & {\capca17.05} & {\capca4} & {\capca13.04} & {\capca4} & {\capca11.75} & {\capca5} & {\capca8.65} & {\capca6} & {\capca6.56} & {\capca6} & {\capca4.93} & {\capca7} & {\capca2.37} & {\capca8} \\\hline
{} & {Zonal Winds} & {\capca37.11} & {\capca2} & {\capca37.11} & {\capca2} & {\capca33.25} & {\capca2} & {\capca31.56} & {\capca2} & {\capca27.36} & {\capca2} & {\capca23.5} & {\capca2} & {\capca20.54} & {\capca2} & {\capca16.44} & {\capca3} \\\hline
{} & {Merid. Winds} & {\capca37.29} & {\capca2} & {\capca37.29} & {\capca2} & {\capca34.1} & {\capca2} & {\capca33.16} & {\capca2} & {\capca29.16} & {\capca2} & {\capca25.86} & {\capca2} & {\capca23.33} & {\capca2} & {\capca19.15} & {\capca2} \\\hline
{} & {Humidity} & {\capca26.39} & {\capca2} & {\capca26.29} & {\capca2} & {\capca25.38} & {\capca2} & {\capca23.42} & {\capca2} & {\capca20.51} & {\capca2} & {\capca18.14} & {\capca2} & {\capca16.01} & {\capca2} & {\capca12.94} & {\capca2} \\\hline
{} & {AirTemp} & {\capca36.2} & {\capca2} & {\capca34.96} & {\capca2} & {\capca30.33} & {\capca2} & {\capca27.39} & {\capca2} & {\capca22.42} & {\capca2} & {\capca19.24} & {\capca3} & {\capca16.76} & {\capca3} & {\capca13.31} & {\capca4} \\\hline
{} & {SST} & {\capca36.79} & {\capca2} & {\capca30.96} & {\capca2} & {\capca24.6} & {\capca2} & {\capca20.61} & {\capca2} & {\capca14.17} & {\capca3} & {\capca10.66} & {\capca4} & {\capca8.21} & {\capca4} & {\capca5.42} & {\capca5} \\\hline
{\datasethail} & {Lat} & {\capca\color{red}114.81} & {\capca2} & {\capca\color{red}102.05} & {\capca2} & {\capca89.83} & {\capca2} & {\capca82.62} & {\capca2} & {\capca71.49} & {\capca2} & {\capca64.62} & {\capca3} & {\capca57.49} & {\capca3} & {\capca46.75} & {\capca3} \\\hline
{} & {Long} & {\capca\color{red}114.14} & {\capca2} & {\capca\color{red}100.96} & {\capca2} & {\capca85.91} & {\capca2} & {\capca77.5} & {\capca2} & {\capca65.06} & {\capca2} & {\capca55.38} & {\capca3} & {\capca48.72} & {\capca3} & {\capca38.74} & {\capca4} \\\hline
{} & {Size} & {\capca80.61} & {\capca2} & {\capca80.59} & {\capca2} & {\capca80.59} & {\capca2} & {\capca80.58} & {\capca2} & {\capca80.56} & {\capca2} & {\capca80.53} & {\capca2} & {\capca80.52} & {\capca2} & {\capca64.35} & {\capca3} \\\hline
{\datasettornado} & {Lat} & {\capca\color{red}111.97} & {\capca2} & {\capca85.43} & {\capca2} & {\capca70.63} & {\capca2} & {\capca65.17} & {\capca2} & {\capca54.17} & {\capca3} & {\capca46.78} & {\capca3} & {\capca41.95} & {\capca4} & {\capca33.48} & {\capca4} \\\hline
{} & {Long} & {\capca\color{red}111.05} & {\capca2} & {\capca82.12} & {\capca2} & {\capca65.09} & {\capca2} & {\capca57.66} & {\capca3} & {\capca45.55} & {\capca3} & {\capca39.88} & {\capca4} & {\capca34.84} & {\capca4} & {\capca28.41} & {\capca4} \\\hline
{\datasetwind} & {Lat} & {\capca\color{red}113.34} & {\capca2} & {\capca\color{red}101.6} & {\capca2} & {\capca88.74} & {\capca2} & {\capca81.29} & {\capca2} & {\capca69.82} & {\capca2} & {\capca62.44} & {\capca3} & {\capca56.18} & {\capca3} & {\capca47.15} & {\capca3} \\\hline
{} & {Long} & {\capca\color{red}112.6} & {\capca2} & {\capca95.41} & {\capca2} & {\capca80.29} & {\capca2} & {\capca73.21} & {\capca2} & {\capca62.06} & {\capca3} & {\capca54.33} & {\capca3} & {\capca48.52} & {\capca3} & {\capca39.73} & {\capca4} \\\hline
{} & {Speed} & {\capca98.1} & {\capca2} & {\capca43.82} & {\capca3} & {\capca28.02} & {\capca4} & {\capca23.98} & {\capca4} & {\capca15.71} & {\capca5} & {\capca12.29} & {\capca6} & {\capca10.33} & {\capca6} & {\capca8.21} & {\capca6} \\\hline
\end{tabular}
\caption{\captionone}
\label{experiments:mask-results-overview1}
\end{sidewaystable}

\clearpage
\begin{sidewaystable}[ht]
\newcommand{\cpca}{\cellcolor{cyan!20}}
\newcommand{\capca}{\cellcolor{green!20}}
\newcommand{\cfr}{\cellcolor{yellow!25}}
\newcommand{\cgzip}{\cellcolor{orange!20}}
\centering
\legendsfive
\begin{tabular}{| l | l | c || c || c || c || c || c || c || c |}
\cline{3-10}
\multicolumn{1}{c}{}& \multicolumn{1}{c|}{} & \multicolumn{1}{c||}{e = 0} & \multicolumn{1}{c||}{e = 1} & \multicolumn{1}{c||}{e = 3} & \multicolumn{1}{c||}{e = 5} & \multicolumn{1}{c||}{e = 10} & \multicolumn{1}{c||}{e = 15} & \multicolumn{1}{c||}{e = 20} & \multicolumn{1}{c|}{e = 30} \\\hline
{Dataset} & {Data Type} & {\footnotesize RD} & {\footnotesize RD} & {\footnotesize RD} & {\footnotesize RD} & {\footnotesize RD} & {\footnotesize RD} & {\footnotesize RD} & {\footnotesize RD} \\\hline\hline
{\datasetirkis} & {VWC} & {\capca40.52} & {\capca42.28} & {\capca50.62} & {\capca59.86} & {\capca71.73} & {\capca75.33} & {\capca77.21} & {\capca80.28} \\\hline
{\datasetsst} & {SST} & {\cpca-8.64} & {\capca32.71} & {\capca49.42} & {\capca60.11} & {\capca68.29} & {\capca73.53} & {\capca76.95} & {\capca78.54} \\\hline
{\datasetadcp} & {Vel} & {\cpca-13.62} & {\cpca-9.22} & {\capca2.08} & {\capca10.48} & {\capca25.57} & {\capca31.35} & {\capca35.08} & {\capca40.41} \\\hline
{\datasetsolar} & {GHI} & {\cpca-6.42} & {\capca0.53} & {\capca5.62} & {\capca10.34} & {\capca19.4} & {\capca25.58} & {\capca31.11} & {\capca40.97} \\\hline
{} & {DNI} & {\cpca-3.92} & {\capca2.86} & {\capca9.62} & {\capca14.16} & {\capca22.65} & {\capca28.79} & {\capca34.44} & {\capca42.61} \\\hline
{} & {DHI} & {\cpca-5.68} & {\cpca-0.94} & {\capca5.9} & {\capca10.26} & {\capca18.74} & {\capca26.18} & {\capca32.34} & {\capca41.94} \\\hline
{\datasetelnino} & {Lat} & {\capca36.3} & {\capca36.3} & {\capca36.44} & {\capca37.28} & {\capca40.65} & {\capca45.2} & {\capca45.95} & {\capca52.65} \\\hline
{} & {Long} & {\capca36.28} & {\capca36.44} & {\capca40.39} & {\capca42.54} & {\capca47.32} & {\capca51.76} & {\capca56.53} & {\capca70.89} \\\hline
{} & {Zonal Winds} & {\cpca-17.98} & {\cpca-17.98} & {\cpca-5.69} & {\cpca-0.33} & {\capca13.02} & {\capca23.75} & {\capca30.33} & {\capca36.58} \\\hline
{} & {Merid. Winds} & {\cpca-18.54} & {\cpca-18.54} & {\cpca-8.42} & {\cpca-5.4} & {\capca7.32} & {\capca17.59} & {\capca23.75} & {\capca32.07} \\\hline
{} & {Humidity} & {\cpca-14.26} & {\cpca-13.8} & {\cpca-9.89} & {\cpca-1.37} & {\capca11.21} & {\capca20.82} & {\capca27.55} & {\capca34.9} \\\hline
{} & {AirTemp} & {\cpca-10.78} & {\cpca-6.97} & {\capca7.2} & {\capca14.24} & {\capca22.22} & {\capca25.69} & {\capca29.38} & {\capca35.61} \\\hline
{} & {SST} & {\cpca-11.79} & {\capca5.91} & {\capca20.8} & {\capca27.61} & {\capca38.16} & {\capca43.82} & {\capca49.02} & {\capca57.25} \\\hline
{\datasethail} & {Lat} & {\cpca-14.77} & {\cpca-2.01} & {\capca9.73} & {\capca15.05} & {\capca21.47} & {\capca24.54} & {\capca28.8} & {\capca34.4} \\\hline
{} & {Long} & {\cpca-14.11} & {\cpca-0.93} & {\capca13.12} & {\capca18.88} & {\capca25.0} & {\capca30.37} & {\capca33.5} & {\capca39.63} \\\hline
{} & {Size} & {\capca14.31} & {\capca14.32} & {\capca14.32} & {\capca14.32} & {\capca14.33} & {\capca14.35} & {\capca14.35} & {\capca26.22} \\\hline
{\datasettornado} & {Lat} & {\cpca-11.92} & {\capca14.36} & {\capca24.78} & {\capca26.95} & {\capca32.22} & {\capca36.77} & {\capca39.67} & {\capca45.69} \\\hline
{} & {Long} & {\cpca-10.92} & {\capca16.95} & {\capca27.03} & {\capca30.68} & {\capca37.96} & {\capca41.04} & {\capca43.82} & {\capca48.34} \\\hline
{\datasetwind} & {Lat} & {\cpca-13.31} & {\cpca-1.57} & {\capca10.63} & {\capca15.85} & {\capca22.19} & {\capca25.5} & {\capca29.32} & {\capca33.96} \\\hline
{} & {Long} & {\cpca-12.56} & {\capca4.62} & {\capca16.42} & {\capca19.94} & {\capca25.42} & {\capca29.79} & {\capca32.74} & {\capca37.96} \\\hline
{} & {Speed} & {\capca1.94} & {\capca35.31} & {\capca49.27} & {\capca45.76} & {\capca58.2} & {\capca64.93} & {\capca68.48} & {\capca73.05} \\\hline
\end{tabular}
\caption{\captionone}
\label{experiments:mask-results-overview3}
\end{sidewaystable}

\clearpage
\begin{table}
\newcommand{\cpca}{\cellcolor{cyan!20}}
\newcommand{\capca}{\cellcolor{green!20}}
\newcommand{\cfr}{\cellcolor{yellow!25}}
\newcommand{\cgzip}{\cellcolor{orange!20}}
\newcommand{\cpwlhint}{\cellcolor{violet!25}}
\newcommand{\cpwlh}{\cellcolor{violet!50}}
\newcommand{\cca}{\cellcolor{brown!20}}
\centering
\legendsone
\hspace*{-2.1cm}\begin{tabular}{| l | l | c | c || c | c || c | c || c | c || c | c || c | c || c | c || c | c |}
\cline{3-18}
\multicolumn{1}{c}{}& \multicolumn{1}{c|}{} & \multicolumn{2}{c||}{e = 0} & \multicolumn{2}{c||}{e = 1} & \multicolumn{2}{c||}{e = 3} & \multicolumn{2}{c||}{e = 5} & \multicolumn{2}{c||}{e = 10} & \multicolumn{2}{c||}{e = 15} & \multicolumn{2}{c||}{e = 20} & \multicolumn{2}{c|}{e = 30} \\\hline
{Dataset} & {Data Type} & {\footnotesize CR} & {\footnotesize RD} & {\footnotesize CR} & {\footnotesize RD} & {\footnotesize CR} & {\footnotesize RD} & {\footnotesize CR} & {\footnotesize RD} & {\footnotesize CR} & {\footnotesize RD} & {\footnotesize CR} & {\footnotesize RD} & {\footnotesize CR} & {\footnotesize RD} & {\footnotesize CR} & {\footnotesize RD} \\\hline\hline
{\datasetirkis} & {VWC} & {\capca0.28} & {\capca26.94} & {\capca0.28} & {\capca34.0} & {\capca0.25} & {\capca49.94} & {\capca0.22} & {\capca68.95} & {\capca0.13} & {\capca76.68} & {\capca0.07} & {\capca67.29} & {\capca0.03} & {\capca50.37} & {\capca0.02} & {\capca25.64} \\\hline
{\datasetsst} & {SST} & {\cpca0.64} & {\cpca4.69} & {\capca0.33} & {\capca15.74} & {\capca0.15} & {\capca11.32} & {\capca0.1} & {\capca8.69} & {\capca0.05} & {\capca5.74} & {\capca0.03} & {\capca5.23} & {\capca0.03} & {\capca8.88} & {\capca0.02} & {\capca15.12} \\\hline
{\datasetadcp} & {Vel} & {\cpca0.73} & {\cpca6.92} & {\cpca0.73} & {\cpca6.39} & {\capca0.7} & {\capca4.34} & {\capca0.67} & {\capca8.61} & {\capca0.59} & {\capca18.28} & {\capca0.51} & {\capca19.95} & {\capca0.45} & {\capca21.92} & {\capca0.34} & {\capca23.78} \\\hline
{\datasetsolar} & {GHI} & {\cpca0.81} & {\cpca4.13} & {\capca0.78} & {\capca2.77} & {\capca0.76} & {\capca5.78} & {\capca0.73} & {\capca8.27} & {\capca0.67} & {\capca12.62} & {\capca0.63} & {\capca16.59} & {\capca0.6} & {\capca21.46} & {\capca0.55} & {\capca31.77} \\\hline
{} & {DNI} & {\cpca0.76} & {\cpca0.61} & {\capca0.74} & {\capca2.79} & {\capca0.73} & {\capca9.37} & {\capca0.71} & {\capca13.39} & {\capca0.67} & {\capca19.68} & {\capca0.64} & {\capca24.42} & {\capca0.62} & {\capca29.65} & {\capca0.57} & {\capca37.69} \\\hline
{} & {DHI} & {\cpca0.81} & {\cpca3.82} & {\cpca0.81} & {\cpca4.1} & {\capca0.76} & {\capca5.22} & {\capca0.73} & {\capca7.11} & {\capca0.68} & {\capca11.19} & {\capca0.63} & {\capca15.55} & {\capca0.6} & {\capca20.34} & {\capca0.55} & {\capca29.11} \\\hline
{\datasetelnino} & {Lat} & {\capca0.23} & {\capca29.23} & {\capca0.23} & {\capca29.23} & {\capca0.23} & {\capca29.83} & {\capca0.23} & {\capca33.0} & {\capca0.22} & {\capca44.82} & {\capca0.21} & {\capca53.84} & {\capca0.2} & {\capca55.95} & {\capca0.17} & {\capca66.05} \\\hline
{} & {Long} & {\capca0.25} & {\capca29.72} & {\capca0.24} & {\capca30.34} & {\capca0.24} & {\capca46.18} & {\capca0.22} & {\capca47.63} & {\capca0.18} & {\capca51.24} & {\capca0.16} & {\capca58.83} & {\capca0.14} & {\capca65.11} & {\capca0.11} & {\capca79.27} \\\hline
{} & {Z. Wind} & {\cpca0.35} & {\cpca8.85} & {\cpca0.35} & {\cpca8.85} & {\cpca0.35} & {\cpca8.85} & {\cpca0.34} & {\cpca6.81} & {\capca0.31} & {\capca12.08} & {\capca0.28} & {\capca17.18} & {\capca0.26} & {\capca21.15} & {\capca0.21} & {\capca21.62} \\\hline
{} & {M. Wind} & {\cpca0.35} & {\cpca9.0} & {\cpca0.35} & {\cpca9.0} & {\cpca0.35} & {\cpca9.0} & {\cpca0.34} & {\cpca8.32} & {\capca0.32} & {\capca9.27} & {\capca0.3} & {\capca13.92} & {\capca0.28} & {\capca17.64} & {\capca0.25} & {\capca21.85} \\\hline
{} & {Humidity} & {\cpca0.25} & {\cpca8.03} & {\cpca0.25} & {\cpca8.03} & {\cpca0.25} & {\cpca7.53} & {\cpca0.24} & {\cpca4.83} & {\capca0.22} & {\capca8.35} & {\capca0.21} & {\capca13.1} & {\capca0.19} & {\capca17.24} & {\capca0.16} & {\capca18.35} \\\hline
{} & {AirTemp} & {\cpca0.35} & {\cpca6.48} & {\cpca0.35} & {\cpca5.85} & {\capca0.32} & {\capca6.52} & {\capca0.3} & {\capca9.63} & {\capca0.26} & {\capca13.52} & {\capca0.23} & {\capca15.48} & {\capca0.2} & {\capca16.37} & {\capca0.16} & {\capca18.99} \\\hline
{} & {SST} & {\cpca0.35} & {\cpca6.0} & {\capca0.34} & {\capca9.42} & {\capca0.29} & {\capca15.56} & {\capca0.24} & {\capca15.88} & {\capca0.17} & {\capca17.57} & {\capca0.13} & {\capca17.94} & {\capca0.1} & {\capca17.94} & {\capca0.06} & {\capca16.62} \\\hline
{\datasethail} & {Lat} & {\cpca1.09} & {\cpca7.88} & {\cpca1.04} & {\cpca4.23} & {\capca0.99} & {\capca9.35} & {\capca0.95} & {\capca13.11} & {\capca0.86} & {\capca17.13} & {\capca0.8} & {\capca18.79} & {\capca0.74} & {\capca22.1} & {\capca0.64} & {\capca26.64} \\\hline
{} & {Long} & {\cpca1.08} & {\cpca7.35} & {\cpca1.03} & {\cpca2.89} & {\capca0.96} & {\capca10.39} & {\capca0.91} & {\capca14.78} & {\capca0.8} & {\capca18.61} & {\capca0.72} & {\capca23.22} & {\capca0.65} & {\capca25.53} & {\capca0.56} & {\capca30.52} \\\hline
{} & {Size} & {\capca1.0} & {\capca19.31} & {\capca1.0} & {\capca19.33} & {\capca0.99} & {\capca18.94} & {\capca0.99} & {\capca18.39} & {\capca0.97} & {\capca17.07} & {\capca0.94} & {\capca14.71} & {\capca0.93} & {\capca13.19} & {\capca0.85} & {\capca24.56} \\\hline
{\datasettornado} & {Lat} & {\cpca1.08} & {\cpca7.54} & {\capca0.95} & {\capca10.55} & {\capca0.84} & {\capca15.95} & {\capca0.77} & {\capca14.96} & {\capca0.66} & {\capca18.43} & {\capca0.58} & {\capca20.01} & {\capca0.54} & {\capca21.91} & {\capca0.46} & {\capca26.57} \\\hline
{} & {Long} & {\cpca1.08} & {\cpca7.02} & {\capca0.9} & {\capca9.23} & {\capca0.75} & {\capca13.01} & {\capca0.68} & {\capca15.81} & {\capca0.55} & {\capca17.48} & {\capca0.49} & {\capca18.04} & {\capca0.44} & {\capca21.47} & {\capca0.38} & {\capca24.37} \\\hline
{\datasetwind} & {Lat} & {\cpca1.08} & {\cpca7.69} & {\cpca1.05} & {\cpca4.92} & {\capca1.0} & {\capca11.36} & {\capca0.97} & {\capca15.86} & {\capca0.9} & {\capca22.49} & {\capca0.84} & {\capca25.71} & {\capca0.79} & {\capca29.3} & {\capca0.73} & {\capca34.98} \\\hline
{} & {Long} & {\cpca1.08} & {\cpca7.19} & {\capca1.02} & {\capca6.41} & {\capca0.96} & {\capca16.07} & {\capca0.92} & {\capca20.15} & {\capca0.83} & {\capca25.56} & {\capca0.78} & {\capca30.33} & {\capca0.74} & {\capca34.07} & {\capca0.66} & {\capca40.07} \\\hline
\end{tabular}
\caption{CoderPWLHInt vs. BEST}
\label{experiments:mask-results-overview1}
\end{table}

\clearpage
\begin{table}
\newcommand{\cpca}{\cellcolor{cyan!20}}
\newcommand{\capca}{\cellcolor{green!20}}
\newcommand{\cfr}{\cellcolor{yellow!25}}
\newcommand{\cgzip}{\cellcolor{orange!20}}
\newcommand{\cpwlhint}{\cellcolor{violet!25}}
\newcommand{\cpwlh}{\cellcolor{violet!50}}
\newcommand{\cca}{\cellcolor{brown!20}}
\centering
\legendsone
\hspace*{-2.1cm}\begin{tabular}{| l | l | c | c || c | c || c | c || c | c || c | c || c | c || c | c || c | c |}
\cline{3-18}
\multicolumn{1}{c}{}& \multicolumn{1}{c|}{} & \multicolumn{2}{c||}{e = 0} & \multicolumn{2}{c||}{e = 1} & \multicolumn{2}{c||}{e = 3} & \multicolumn{2}{c||}{e = 5} & \multicolumn{2}{c||}{e = 10} & \multicolumn{2}{c||}{e = 15} & \multicolumn{2}{c||}{e = 20} & \multicolumn{2}{c|}{e = 30} \\\hline
{Dataset} & {Data Type} & {\footnotesize CR} & {\footnotesize RD} & {\footnotesize CR} & {\footnotesize RD} & {\footnotesize CR} & {\footnotesize RD} & {\footnotesize CR} & {\footnotesize RD} & {\footnotesize CR} & {\footnotesize RD} & {\footnotesize CR} & {\footnotesize RD} & {\footnotesize CR} & {\footnotesize RD} & {\footnotesize CR} & {\footnotesize RD} \\\hline\hline
{\datasetirkis} & {VWC} & {\capca0.2} & {\capca0} & {\capca0.18} & {\capca0} & {\capca0.12} & {\capca0} & {\capca0.07} & {\capca0} & {\capca0.03} & {\capca0} & {\capca0.02} & {\capca0} & {\capca0.02} & {\capca0} & {\capca0.01} & {\capca0} \\\hline
{\datasetsst} & {SST} & {\cpca0.66} & {\cpca7.96} & {\capca0.28} & {\capca0} & {\capca0.14} & {\capca0} & {\capca0.09} & {\capca0} & {\capca0.05} & {\capca0} & {\capca0.03} & {\capca0} & {\capca0.02} & {\capca0} & {\capca0.02} & {\capca0} \\\hline
{\datasetadcp} & {Vel} & {\cpca0.78} & {\cpca11.99} & {\cpca0.75} & {\cpca8.44} & {\capca0.67} & {\capca0} & {\capca0.61} & {\capca0} & {\capca0.48} & {\capca0} & {\capca0.41} & {\capca0} & {\capca0.35} & {\capca0} & {\capca0.26} & {\capca0} \\\hline
{\datasetsolar} & {GHI} & {\cpca0.83} & {\cpca6.03} & {\capca0.76} & {\capca0} & {\capca0.71} & {\capca0} & {\capca0.67} & {\capca0} & {\capca0.59} & {\capca0} & {\capca0.52} & {\capca0} & {\capca0.47} & {\capca0} & {\capca0.38} & {\capca0} \\\hline
{} & {DNI} & {\cpca0.79} & {\cpca3.78} & {\capca0.72} & {\capca0} & {\capca0.66} & {\capca0} & {\capca0.61} & {\capca0} & {\capca0.54} & {\capca0} & {\capca0.49} & {\capca0} & {\capca0.43} & {\capca0} & {\capca0.36} & {\capca0} \\\hline
{} & {DHI} & {\cpca0.82} & {\cpca5.37} & {\cpca0.78} & {\cpca0.94} & {\capca0.72} & {\capca0} & {\capca0.68} & {\capca0} & {\capca0.6} & {\capca0} & {\capca0.54} & {\capca0} & {\capca0.48} & {\capca0} & {\capca0.39} & {\capca0} \\\hline
{\datasetelnino} & {Lat} & {\capca0.16} & {\capca0} & {\capca0.16} & {\capca0} & {\capca0.16} & {\capca0} & {\capca0.15} & {\capca0} & {\capca0.12} & {\capca0} & {\capca0.1} & {\capca0} & {\capca0.09} & {\capca0} & {\capca0.06} & {\capca0} \\\hline
{} & {Long} & {\capca0.17} & {\capca0} & {\capca0.17} & {\capca0} & {\capca0.13} & {\capca0} & {\capca0.12} & {\capca0} & {\capca0.09} & {\capca0} & {\capca0.07} & {\capca0} & {\capca0.05} & {\capca0} & {\capca0.02} & {\capca0} \\\hline
{} & {Z. Wind} & {\cpca0.37} & {\cpca15.24} & {\cpca0.37} & {\cpca15.24} & {\cpca0.33} & {\cpca5.38} & {\cpca0.32} & {\cpca0.33} & {\capca0.27} & {\capca0} & {\capca0.24} & {\capca0} & {\capca0.21} & {\capca0} & {\capca0.16} & {\capca0} \\\hline
{} & {M. Wind} & {\cpca0.37} & {\cpca15.64} & {\cpca0.37} & {\cpca15.64} & {\cpca0.34} & {\cpca7.76} & {\cpca0.33} & {\cpca5.13} & {\capca0.29} & {\capca0} & {\capca0.26} & {\capca0} & {\capca0.23} & {\capca0} & {\capca0.19} & {\capca0} \\\hline
{} & {Humidity} & {\cpca0.26} & {\cpca12.48} & {\cpca0.26} & {\cpca12.13} & {\cpca0.25} & {\cpca9.0} & {\cpca0.23} & {\cpca1.35} & {\capca0.21} & {\capca0} & {\capca0.18} & {\capca0} & {\capca0.16} & {\capca0} & {\capca0.13} & {\capca0} \\\hline
{} & {AirTemp} & {\cpca0.36} & {\cpca9.73} & {\cpca0.35} & {\cpca6.51} & {\capca0.3} & {\capca0} & {\capca0.27} & {\capca0} & {\capca0.22} & {\capca0} & {\capca0.19} & {\capca0} & {\capca0.17} & {\capca0} & {\capca0.13} & {\capca0} \\\hline
{} & {SST} & {\cpca0.37} & {\cpca10.55} & {\capca0.31} & {\capca0} & {\capca0.25} & {\capca0} & {\capca0.21} & {\capca0} & {\capca0.14} & {\capca0} & {\capca0.11} & {\capca0} & {\capca0.08} & {\capca0} & {\capca0.05} & {\capca0} \\\hline
{\datasethail} & {Lat} & {\cpca1.15} & {\cpca12.87} & {\cpca1.02} & {\cpca1.97} & {\capca0.9} & {\capca0} & {\capca0.83} & {\capca0} & {\capca0.71} & {\capca0} & {\capca0.65} & {\capca0} & {\capca0.57} & {\capca0} & {\capca0.47} & {\capca0} \\\hline
{} & {Long} & {\cpca1.14} & {\cpca12.36} & {\cpca1.01} & {\cpca0.92} & {\capca0.86} & {\capca0} & {\capca0.78} & {\capca0} & {\capca0.65} & {\capca0} & {\capca0.55} & {\capca0} & {\capca0.49} & {\capca0} & {\capca0.39} & {\capca0} \\\hline
{} & {Size} & {\capca0.81} & {\capca0} & {\capca0.81} & {\capca0} & {\capca0.81} & {\capca0} & {\capca0.81} & {\capca0} & {\capca0.81} & {\capca0} & {\capca0.81} & {\capca0} & {\capca0.81} & {\capca0} & {\capca0.64} & {\capca0} \\\hline
{\datasettornado} & {Lat} & {\cpca1.12} & {\cpca10.65} & {\capca0.85} & {\capca0} & {\capca0.71} & {\capca0} & {\capca0.65} & {\capca0} & {\capca0.54} & {\capca0} & {\capca0.47} & {\capca0} & {\capca0.42} & {\capca0} & {\capca0.33} & {\capca0} \\\hline
{} & {Long} & {\cpca1.11} & {\cpca9.85} & {\capca0.82} & {\capca0} & {\capca0.65} & {\capca0} & {\capca0.58} & {\capca0} & {\capca0.46} & {\capca0} & {\capca0.4} & {\capca0} & {\capca0.35} & {\capca0} & {\capca0.28} & {\capca0} \\\hline
{\datasetwind} & {Lat} & {\cpca1.13} & {\cpca11.74} & {\cpca1.02} & {\cpca1.55} & {\capca0.89} & {\capca0} & {\capca0.81} & {\capca0} & {\capca0.7} & {\capca0} & {\capca0.62} & {\capca0} & {\capca0.56} & {\capca0} & {\capca0.47} & {\capca0} \\\hline
{} & {Long} & {\cpca1.13} & {\cpca11.16} & {\capca0.95} & {\capca0} & {\capca0.8} & {\capca0} & {\capca0.73} & {\capca0} & {\capca0.62} & {\capca0} & {\capca0.54} & {\capca0} & {\capca0.49} & {\capca0} & {\capca0.4} & {\capca0} \\\hline
{} & {Speed} & {\cfr0.98} & {\cfr33.25} & {\capca0.44} & {\capca0} & {\cfr0.28} & {\cfr7.57} & {\cfr0.24} & {\cfr29.96} & {\capca0.16} & {\capca0} & {\capca0.12} & {\capca0} & {\capca0.1} & {\capca0} & {\capca0.08} & {\capca0} \\\hline
\end{tabular}
\caption{CoderAPCA vs. BEST}
\label{experiments:mask-results-overview1}
\end{table}

\clearpage
\begin{table}
\newcommand{\cpca}{\cellcolor{cyan!20}}
\newcommand{\capca}{\cellcolor{green!20}}
\newcommand{\cfr}{\cellcolor{yellow!25}}
\newcommand{\cgzip}{\cellcolor{orange!20}}
\newcommand{\cpwlhint}{\cellcolor{violet!25}}
\newcommand{\cpwlh}{\cellcolor{violet!50}}
\newcommand{\cca}{\cellcolor{brown!20}}
\centering
\legendsone
\hspace*{-2.1cm}\begin{tabular}{| l | l | c | c || c | c || c | c || c | c || c | c || c | c || c | c || c | c |}
\cline{3-18}
\multicolumn{1}{c}{}& \multicolumn{1}{c|}{} & \multicolumn{2}{c||}{e = 0} & \multicolumn{2}{c||}{e = 1} & \multicolumn{2}{c||}{e = 3} & \multicolumn{2}{c||}{e = 5} & \multicolumn{2}{c||}{e = 10} & \multicolumn{2}{c||}{e = 15} & \multicolumn{2}{c||}{e = 20} & \multicolumn{2}{c|}{e = 30} \\\hline
{Dataset} & {Data Type} & {\footnotesize CR} & {\footnotesize RD} & {\footnotesize CR} & {\footnotesize RD} & {\footnotesize CR} & {\footnotesize RD} & {\footnotesize CR} & {\footnotesize RD} & {\footnotesize CR} & {\footnotesize RD} & {\footnotesize CR} & {\footnotesize RD} & {\footnotesize CR} & {\footnotesize RD} & {\footnotesize CR} & {\footnotesize RD} \\\hline\hline
{\datasetirkis} & {VWC} & {\capca0.34} & {\capca40.52} & {\capca0.32} & {\capca42.28} & {\capca0.25} & {\capca50.62} & {\capca0.17} & {\capca59.86} & {\capca0.11} & {\capca71.73} & {\capca0.09} & {\capca75.33} & {\capca0.07} & {\capca77.21} & {\capca0.06} & {\capca80.28} \\\hline
{\datasetsst} & {SST} & {\cpca0.61} & {\cpca0} & {\capca0.42} & {\capca32.71} & {\capca0.27} & {\capca49.42} & {\capca0.22} & {\capca60.11} & {\capca0.15} & {\capca68.29} & {\capca0.12} & {\capca73.53} & {\capca0.1} & {\capca76.95} & {\capca0.08} & {\capca78.54} \\\hline
{\datasetadcp} & {Vel} & {\cpca0.68} & {\cpca0} & {\cpca0.68} & {\cpca0} & {\capca0.68} & {\capca2.08} & {\capca0.68} & {\capca10.48} & {\capca0.65} & {\capca25.57} & {\capca0.6} & {\capca31.35} & {\capca0.54} & {\capca35.08} & {\capca0.44} & {\capca40.41} \\\hline
{\datasetsolar} & {GHI} & {\cpca0.78} & {\cpca0} & {\capca0.77} & {\capca0.53} & {\capca0.76} & {\capca5.62} & {\capca0.75} & {\capca10.34} & {\capca0.73} & {\capca19.4} & {\capca0.7} & {\capca25.58} & {\capca0.68} & {\capca31.11} & {\capca0.64} & {\capca40.97} \\\hline
{} & {DNI} & {\cpca0.76} & {\cpca0} & {\capca0.74} & {\capca2.86} & {\capca0.73} & {\capca9.62} & {\capca0.71} & {\capca14.16} & {\capca0.7} & {\capca22.65} & {\capca0.68} & {\capca28.79} & {\capca0.66} & {\capca34.44} & {\capca0.62} & {\capca42.61} \\\hline
{} & {DHI} & {\cpca0.78} & {\cpca0} & {\cpca0.77} & {\cpca0} & {\capca0.76} & {\capca5.9} & {\capca0.75} & {\capca10.26} & {\capca0.74} & {\capca18.74} & {\capca0.73} & {\capca26.18} & {\capca0.71} & {\capca32.34} & {\capca0.67} & {\capca41.94} \\\hline
{\datasetelnino} & {Lat} & {\capca0.25} & {\capca36.3} & {\capca0.25} & {\capca36.3} & {\capca0.25} & {\capca36.44} & {\capca0.24} & {\capca37.28} & {\capca0.21} & {\capca40.65} & {\capca0.18} & {\capca45.2} & {\capca0.16} & {\capca45.95} & {\capca0.12} & {\capca52.65} \\\hline
{} & {Long} & {\capca0.27} & {\capca36.28} & {\capca0.27} & {\capca36.44} & {\capca0.22} & {\capca40.39} & {\capca0.2} & {\capca42.54} & {\capca0.16} & {\capca47.32} & {\capca0.14} & {\capca51.76} & {\capca0.11} & {\capca56.53} & {\capca0.08} & {\capca70.89} \\\hline
{} & {Z. Wind} & {\cpca0.31} & {\cpca0} & {\cpca0.31} & {\cpca0} & {\cpca0.31} & {\cpca0} & {\cpca0.31} & {\cpca0} & {\capca0.31} & {\capca13.02} & {\capca0.31} & {\capca23.75} & {\capca0.29} & {\capca30.33} & {\capca0.26} & {\capca36.58} \\\hline
{} & {M. Wind} & {\cpca0.31} & {\cpca0} & {\cpca0.31} & {\cpca0} & {\cpca0.31} & {\cpca0} & {\cpca0.31} & {\cpca0} & {\capca0.31} & {\capca7.32} & {\capca0.31} & {\capca17.59} & {\capca0.31} & {\capca23.75} & {\capca0.28} & {\capca32.07} \\\hline
{} & {Humidity} & {\cpca0.23} & {\cpca0} & {\cpca0.23} & {\cpca0} & {\cpca0.23} & {\cpca0} & {\cpca0.23} & {\cpca0} & {\capca0.23} & {\capca11.21} & {\capca0.23} & {\capca20.82} & {\capca0.22} & {\capca27.55} & {\capca0.2} & {\capca34.9} \\\hline
{} & {AirTemp} & {\cpca0.33} & {\cpca0} & {\cpca0.33} & {\cpca0} & {\capca0.33} & {\capca7.2} & {\capca0.32} & {\capca14.24} & {\capca0.29} & {\capca22.22} & {\capca0.26} & {\capca25.69} & {\capca0.24} & {\capca29.38} & {\capca0.21} & {\capca35.61} \\\hline
{} & {SST} & {\cpca0.33} & {\cpca0} & {\capca0.33} & {\capca5.91} & {\capca0.31} & {\capca20.8} & {\capca0.28} & {\capca27.61} & {\capca0.23} & {\capca38.16} & {\capca0.19} & {\capca43.82} & {\capca0.16} & {\capca49.02} & {\capca0.13} & {\capca57.25} \\\hline
{\datasethail} & {Lat} & {\cpca1.0} & {\cpca0} & {\cpca1.0} & {\cpca0} & {\capca1.0} & {\capca9.73} & {\capca0.97} & {\capca15.05} & {\capca0.91} & {\capca21.47} & {\capca0.86} & {\capca24.54} & {\capca0.81} & {\capca28.8} & {\capca0.71} & {\capca34.4} \\\hline
{} & {Long} & {\cpca1.0} & {\cpca0} & {\cpca1.0} & {\cpca0} & {\capca0.99} & {\capca13.12} & {\capca0.96} & {\capca18.88} & {\capca0.87} & {\capca25.0} & {\capca0.8} & {\capca30.37} & {\capca0.73} & {\capca33.5} & {\capca0.64} & {\capca39.63} \\\hline
{} & {Size} & {\capca0.94} & {\capca14.31} & {\capca0.94} & {\capca14.32} & {\capca0.94} & {\capca14.32} & {\capca0.94} & {\capca14.32} & {\capca0.94} & {\capca14.33} & {\capca0.94} & {\capca14.35} & {\capca0.94} & {\capca14.35} & {\capca0.87} & {\capca26.22} \\\hline
{\datasettornado} & {Lat} & {\cpca1.0} & {\cpca0} & {\capca1.0} & {\capca14.36} & {\capca0.94} & {\capca24.78} & {\capca0.89} & {\capca26.95} & {\capca0.8} & {\capca32.22} & {\capca0.74} & {\capca36.77} & {\capca0.7} & {\capca39.67} & {\capca0.62} & {\capca45.69} \\\hline
{} & {Long} & {\cpca1.0} & {\cpca0} & {\capca0.99} & {\capca16.95} & {\capca0.89} & {\capca27.03} & {\capca0.83} & {\capca30.68} & {\capca0.73} & {\capca37.96} & {\capca0.68} & {\capca41.04} & {\capca0.62} & {\capca43.82} & {\capca0.55} & {\capca48.34} \\\hline
{\datasetwind} & {Lat} & {\cpca1.0} & {\cpca0} & {\cpca1.0} & {\cpca0} & {\capca0.99} & {\capca10.63} & {\capca0.97} & {\capca15.85} & {\capca0.9} & {\capca22.19} & {\capca0.84} & {\capca25.5} & {\capca0.79} & {\capca29.32} & {\capca0.71} & {\capca33.96} \\\hline
{} & {Long} & {\cpca1.0} & {\cpca0} & {\capca1.0} & {\capca4.62} & {\capca0.96} & {\capca16.42} & {\capca0.91} & {\capca19.94} & {\capca0.83} & {\capca25.42} & {\capca0.77} & {\capca29.79} & {\capca0.72} & {\capca32.74} & {\capca0.64} & {\capca37.96} \\\hline
{} & {Speed} & {\cfr1.0} & {\cfr34.54} & {\capca0.68} & {\capca35.31} & {\cfr0.55} & {\cfr53.11} & {\cfr0.44} & {\cfr62.01} & {\capca0.38} & {\capca58.2} & {\capca0.35} & {\capca64.93} & {\capca0.33} & {\capca68.48} & {\capca0.3} & {\capca73.05} \\\hline
\end{tabular}
\caption{CoderPCA vs. BEST}
\label{experiments:mask-results-overview1}
\end{table}

\clearpage

TODO: Comparison with the algorithm gzip. Table with data in next page.

\clearpage

\begin{table}[h]
\newcommand{\cpca}{\cellcolor{cyan!20}}
\newcommand{\capca}{\cellcolor{green!20}}
\newcommand{\cfr}{\cellcolor{yellow!25}}
\newcommand{\cgzip}{\cellcolor{orange!20}}
\centering
\legendstwo
\hspace*{-2.1cm}\begin{tabular}{| l | l | c | c || c | c || c | c || c | c || c | c || c | c || c | c || c | c |}
\cline{3-18}
\multicolumn{1}{c}{}& \multicolumn{1}{c|}{} & \multicolumn{2}{c||}{e = 0} & \multicolumn{2}{c||}{e = 1} & \multicolumn{2}{c||}{e = 3} & \multicolumn{2}{c||}{e = 5} & \multicolumn{2}{c||}{e = 10} & \multicolumn{2}{c||}{e = 15} & \multicolumn{2}{c||}{e = 20} & \multicolumn{2}{c|}{e = 30} \\\hline
{Dataset} & {Data Type} & {\footnotesize CR} & {\footnotesize w} & {\footnotesize CR} & {\footnotesize w} & {\footnotesize CR} & {\footnotesize w} & {\footnotesize CR} & {\footnotesize w} & {\footnotesize CR} & {\footnotesize w} & {\footnotesize CR} & {\footnotesize w} & {\footnotesize CR} & {\footnotesize w} & {\footnotesize CR} & {\footnotesize w} \\\hline\hline
{\datasetirkis} & {VWC} & {\cgzip0.13} & {\cgzip} & {\cgzip0.13} & {\cgzip} & {\capca0.12} & {\capca5} & {\capca0.07} & {\capca6} & {\capca0.03} & {\capca7} & {\capca0.02} & {\capca8} & {\capca0.02} & {\capca8} & {\capca0.01} & {\capca8} \\\hline
{\datasetsst} & {SST} & {\cgzip0.52} & {\cgzip} & {\capca0.28} & {\capca3} & {\capca0.14} & {\capca5} & {\capca0.09} & {\capca6} & {\capca0.05} & {\capca7} & {\capca0.03} & {\capca8} & {\capca0.02} & {\capca8} & {\capca0.02} & {\capca8} \\\hline
{\datasetadcp} & {Vel} & {\cgzip0.61} & {\cgzip} & {\cgzip0.61} & {\cgzip} & {\cgzip0.61} & {\cgzip} & {\capca0.61} & {\capca2} & {\capca0.48} & {\capca2} & {\capca0.41} & {\capca2} & {\capca0.35} & {\capca3} & {\capca0.26} & {\capca3} \\\hline
{\datasetsolar} & {GHI} & {\cgzip0.69} & {\cgzip} & {\cgzip0.69} & {\cgzip} & {\cgzip0.69} & {\cgzip} & {\capca0.67} & {\capca4} & {\capca0.59} & {\capca4} & {\capca0.52} & {\capca4} & {\capca0.47} & {\capca4} & {\capca0.38} & {\capca4} \\\hline
{} & {DNI} & {\cgzip0.67} & {\cgzip} & {\cgzip0.67} & {\cgzip} & {\capca0.66} & {\capca4} & {\capca0.61} & {\capca4} & {\capca0.54} & {\capca4} & {\capca0.49} & {\capca4} & {\capca0.43} & {\capca4} & {\capca0.36} & {\capca4} \\\hline
{} & {DHI} & {\cgzip0.61} & {\cgzip} & {\cgzip0.61} & {\cgzip} & {\cgzip0.61} & {\cgzip} & {\cgzip0.61} & {\cgzip} & {\capca0.60} & {\capca4} & {\capca0.54} & {\capca4} & {\capca0.48} & {\capca4} & {\capca0.39} & {\capca4} \\\hline
{\datasetelnino} & {Lat} & {\cgzip0.08} & {\cgzip} & {\cgzip0.08} & {\cgzip} & {\cgzip0.08} & {\cgzip} & {\cgzip0.08} & {\cgzip} & {\cgzip0.08} & {\cgzip} & {\cgzip0.08} & {\cgzip} & {\cgzip0.08} & {\cgzip} & {\capca0.06} & {\capca6} \\\hline
{} & {Long} & {\cgzip0.07} & {\cgzip} & {\cgzip0.07} & {\cgzip} & {\cgzip0.07} & {\cgzip} & {\cgzip0.07} & {\cgzip} & {\cgzip0.07} & {\cgzip} & {\capca0.07} & {\capca6} & {\capca0.05} & {\capca7} & {\capca0.02} & {\capca8} \\\hline
{} & {Zon.Wind} & {\cpca0.31} & {\cpca8} & {\cpca0.31} & {\cpca8} & {\cpca0.31} & {\cpca8} & {\cpca0.31} & {\cpca8} & {\capca0.27} & {\capca2} & {\capca0.24} & {\capca2} & {\capca0.21} & {\capca2} & {\capca0.16} & {\capca3} \\\hline
{} & {Mer.Wind} & {\cpca0.31} & {\cpca8} & {\cpca0.31} & {\cpca8} & {\cpca0.31} & {\cpca8} & {\cpca0.31} & {\cpca8} & {\capca0.29} & {\capca2} & {\capca0.26} & {\capca2} & {\capca0.23} & {\capca2} & {\capca0.19} & {\capca2} \\\hline
{} & {Humidity} & {\cpca0.23} & {\cpca8} & {\cpca0.23} & {\cpca8} & {\cpca0.23} & {\cpca8} & {\cpca0.23} & {\cpca8} & {\capca0.21} & {\capca2} & {\capca0.18} & {\capca2} & {\capca0.16} & {\capca2} & {\capca0.13} & {\capca2} \\\hline
{} & {Air Temp.} & {\cpca0.33} & {\cpca8} & {\cpca0.33} & {\cpca8} & {\capca0.30} & {\capca2} & {\capca0.27} & {\capca2} & {\capca0.22} & {\capca2} & {\capca0.19} & {\capca3} & {\capca0.17} & {\capca3} & {\capca0.13} & {\capca4} \\\hline
{} & {SST} & {\cgzip0.32} & {\cgzip} & {\capca0.31} & {\capca2} & {\capca0.25} & {\capca2} & {\capca0.21} & {\capca2} & {\capca0.14} & {\capca3} & {\capca0.11} & {\capca4} & {\capca0.08} & {\capca4} & {\capca0.05} & {\capca5} \\\hline
{\datasethail} & {Lat} & {\cpca1.00} & {\cpca8} & {\cpca1.00} & {\cpca8} & {\capca0.90} & {\capca2} & {\capca0.83} & {\capca2} & {\capca0.71} & {\capca2} & {\capca0.65} & {\capca3} & {\capca0.57} & {\capca3} & {\capca0.47} & {\capca3} \\\hline
{} & {Long} & {\cpca1.00} & {\cpca8} & {\cpca1.00} & {\cpca8} & {\capca0.86} & {\capca2} & {\capca0.78} & {\capca2} & {\capca0.65} & {\capca2} & {\capca0.55} & {\capca3} & {\capca0.49} & {\capca3} & {\capca0.39} & {\capca4} \\\hline
{} & {Size} & {\cgzip0.37} & {\cgzip} & {\cgzip0.37} & {\cgzip} & {\cgzip0.37} & {\cgzip} & {\cgzip0.37} & {\cgzip} & {\cgzip0.37} & {\cgzip} & {\cgzip0.37} & {\cgzip} & {\cgzip0.37} & {\cgzip} & {\cgzip0.37} & {\cgzip} \\\hline
{\datasettornado} & {Lat} & {\cpca1.00} & {\cpca8} & {\capca0.85} & {\capca2} & {\capca0.71} & {\capca2} & {\capca0.65} & {\capca2} & {\capca0.54} & {\capca3} & {\capca0.47} & {\capca3} & {\capca0.42} & {\capca4} & {\capca0.33} & {\capca4} \\\hline
{} & {Long} & {\cpca1.00} & {\cpca8} & {\capca0.82} & {\capca2} & {\capca0.65} & {\capca2} & {\capca0.58} & {\capca3} & {\capca0.46} & {\capca3} & {\capca0.40} & {\capca4} & {\capca0.35} & {\capca4} & {\capca0.28} & {\capca4} \\\hline
{\datasetwind} & {Lat} & {\cpca1.00} & {\cpca8} & {\cpca1.00} & {\cpca8} & {\capca0.89} & {\capca2} & {\capca0.81} & {\capca2} & {\capca0.70} & {\capca2} & {\capca0.62} & {\capca3} & {\capca0.56} & {\capca3} & {\capca0.47} & {\capca3} \\\hline
{} & {Long} & {\cpca1.00} & {\cpca8} & {\capca0.95} & {\capca2} & {\capca0.80} & {\capca2} & {\capca0.73} & {\capca2} & {\capca0.62} & {\capca3} & {\capca0.54} & {\capca3} & {\capca0.49} & {\capca3} & {\capca0.40} & {\capca4} \\\hline
{} & {Speed} & {\cfr0.65} & {\cfr4} & {\capca0.44} & {\capca3} & {\cfr0.26} & {\cfr6} & {\cfr0.17} & {\cfr7} & {\capca0.16} & {\capca5} & {\capca0.12} & {\capca6} & {\capca0.10} & {\capca6} & {\capca0.08} & {\capca6} \\\hline
\end{tabular}
\caption{\captiontwo}
\label{experiments:mask-results-overview2}
\end{table}

