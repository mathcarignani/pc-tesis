\clearpage
Un punteo de los cambios más importantes respecto a la versión anterior del informe. \textcolor{blue}{Los cambios de la entrega del 12.03.2020 están en azul.}


\vspace{+5pt}
\textbf{Cambios generales}:
\vspace{-10pt}
\begin{itemize}
    \item Sustituyo todos los “CoderABC” por “ABC” (en el texto y en las gráficas)
\end{itemize}


\vspace{+5pt}
\textbf{Cambios en las gráficas}:
\vspace{-10pt}
\begin{itemize}
    \item Agrego (\%) después de CR y RD
    \item Cambié los markers en las gráficas de CR (Section 3.2) 
    \item Cambié los markers y los colores en las gráficas de Window parameter (Section 3.3) 
\end{itemize}


\vspace{+5pt}
\textbf{Introduction}:
\vspace{-10pt}
\begin{itemize}
    \item Agrego una frase de conclusión para las sections 3.2 y 3.3
\end{itemize}


\vspace{+5pt}
\textbf{Section 3.2}:
\vspace{-10pt}
\begin{itemize}
    \item Sustituyo todos los “mode” por “variant” / En las definiciones cambio “||” por “|”
    \item \textcolor{blue}{Mejoré explicación de la Table 3.1.}
    \item Reescribí la descripción abajo de las gráficas
    \item El párrafo anterior a Table 3.1 lo moví a la misma página de la tabla. Por un lado me parece mejor que este párrafo esté en la misma página de la tabla (sino quedan las dos páginas de gráficas en el medio). Por otro lado no queda muy bien tanto espacio en blanco en la página anterior a las gráficas.
\end{itemize}


\vspace{+5pt}
\textbf{Section 3.3}:
\vspace{-10pt}
\begin{itemize}
    \item Las referencias $a \in A$ las debería cambiar. En vez de A (definido al comienzo de la Section 3.2) debería definir otro conjunto que incluye solamente las variantes con máscaras de A y el algoritmo FR (que solo tiene variante con máscara y que no había tenido en cuenta en las comparaciones de la Section 3.2).
\end{itemize}

\vspace{+5pt}
\textbf{Section 3.4}:
\vspace{-10pt}
\begin{itemize}
    \item \textcolor{blue}{Empiezo a escribir la sección. Agrego las tablas.}
\end{itemize}



% Chapter ?
\chapter{Experimental Results} % Main chapter title

\label{experiments} % For referencing the chapter elsewhere, use \ref{Chapter1} 

\lhead{Chapter 3. \emph{Experimental Results}} % This is for the header on each page - perhaps a shortened title

\newcommand{\maskalgo}{\textit{M}}
\newcommand{\NOmaskalgo}{\textit{NM}}
\newcommand{\coder}{\textit{c}}
\newcommand{\difrelativa}{\textit{RD}}
\newcommand{\tasacompresion}{\textit{CR}}
\newcommand{\nmbits}{\NOmaskalgo_{\textit{S}}}
\newcommand{\mbits}{\maskalgo_\textit{S}}
\newcommand{\cmaskalgo}{$c_\maskalgo$}
\newcommand{\cNOmaskalgo}{$c_\NOmaskalgo$}
\newcommand{\ca}{\textit{CI}}
\newcommand{\algo}{\textit{c}}



In this chapter we present our experimental results. The main goal of our experiments is to analyze the performance of each of the coding algorithms presented in Chapter~\ref{algo}, by encoding the various datasets introduced in Chapter~\ref{datasets}. In Section~\ref{experiments:experiments} we describe our experimental setting, defining the evaluated combinations of algorithms and parameter values, and the figures of merit used for comparison. In Section~\ref{secX:rendimiento-relativo} we compare the compression performance of the masking and non-masking variants for each coding algorithm. The results suggest that the masking variant is more robust and performs better in general. In Section~\ref{secX:windows} we analyze the extent to which the window size parameter impacts on the performance of the algorithms. The results indicate that the impact of using the global window instead of the local window on the compression performance of the coding algorithms is rather small. In Section~\ref{secX:codersmask} we compare the performance of the different algorithms among each other and with the general purpose compression algorithm gzip. ...


\section{Experimental Setting}
\label{experiments:experiments}


We evaluate the compression performance of all the coding algorithms presented in Chapter~\ref{coders} on the datasets described in Chapter~\ref{datasets}. For each algorithm we test both the masking and the non-masking modes (except for $\coderBase$, \textit{CoderFR} and \textit{CoderSF}, which only operate in non-masking mode).


We also test several combinations of algorithm parameters. Specifically, for the algorithms that admit a window size parameter $w$ (every algorithm except $\coderBase$ and \textit{CoderSF}), we test all the values of $w$ in the set $W = \{4, 8, 16, 32, 64, 128, 256\}$. For the encoders that admit a lossy compression mode with a threshold parameter $e$ (every encoder except $\coderBase$), we test all the values of $e$ in the set $E= \{1, 3, 5, 10, 15, 20, 30\}$, where each threshold is expressed as a percentage fraction of the standard deviation of the data type being coded. For example, for a certain data type with a standard deviation of 20, taking $e=10$ implies that the lossy compression allows for a maximum sample distortion of 2 sampling units.


\vspace{+5pt}
\begin{defcion}
We refer to a specific combination of a coding algorithm and its parameter values as a \textit{coding algorithm instance (CAI)}. We define \textit{CI} as the set of all the CAIs obtained by combining each of the algorithms presented in Chapter~\ref{coders} with the parameter values (from $W$ and $E$) which are suitable for that algorithm. We denote by $c_{<a, w, e>}$ the CAI obtained by setting a window size equal to $w$ and a threshold parameter equal to $e$ on the algorithm $a$.
\end{defcion}


We assess the compression performance of a CAI mainly through the compression ratio, which we define next. For that definition, we recall that $\coderBase$ is a trivial encoder that serves as a base ground for compression performance comparison.


\clearpage


\begin{defcion}
Let $f$ be a file and let $z$ be a data type of a certain dataset. We define $f_z$ as the subset of data of type $z$ from file $f$.
\end{defcion}


\vspace{+2pt}
\begin{defcion}
The \textit{compression ratio (CR)} of a CAI $\algo \in \ca$ for the data of type $z$ of a certain file $f$ is given by
\vspace{-5pt}
\begin{equation}
\label{eq:compression-rate}
\tasacompresion(\algo, f_z) = 100\times\frac{|\algo(f_z)|}{|\coderBase(f_z)|},
\end{equation}
where $|\algo(f_z)|$ and $|\coderBase(f_z)|$ are the sizes of the resulting files obtained when coding $f_z$ with $\algo$ and $\coderBase$, respectively.
\end{defcion}


The performance of $\algo$ improves as $|\algo(f_z)|$ decreases. Thus, our main goals are to analyze which CAIs minimize (\ref{eq:compression-rate}) for the different data types, and to study how the CR depends on the different coding algorithms and parameter values.


To compare the compression performance between a pair of CAIs we calculate the relative difference, which we define next. In general, it only makes sense to compare CAIs that have the same threshold parameter $e$.


\vspace{+5pt}
\begin{defcion}
The \textit{relative difference (RD)} between a pair of CAIs $\algo_1, \algo_2 \in {\ca}$ for the data of type $z$ of a certain file $f$ is given by
\vspace{-5pt}
\begin{equation}
\label{eq:relative-difference}
\difrelativa(\algo_1, \algo_2, f_z)  =
100\times\frac{|\algo_2 (f_z)| - |\algo_1 (f_z)|}{ |\algo_2 (f_z)| },
\end{equation}
where $|\algo_1(f_z)|$ and $|\algo_2(f_z)|$ are the sizes of the resulting files obtained when coding $f_z$ with $\algo_1$ and $\algo_2$, respectively. Notice that $\algo_1$ has a better performance than $\algo_2$ when (\ref{eq:relative-difference}) is positive.
\end{defcion}



\clearpage
\section{Comparison of Masking and Non-Masking Variants}
\label{secX:rendimiento-relativo}


In this section, we compare the compression performance of the masking and non-masking variants of every coding algorithm in $A_M$ (recall this definition from the first paragraph in Section~\ref{experiments:experiments}). Specifically, we compare:


\vspace{-6pt}
\newcommand{\against}[1]{$\text{{#1}}_\textit{M}$ against $\text{{#1}}_\textit{NM}$}
\begin{itemize}
    \item \against{PCA}
    \item \against{APCA}
    \item \against{CA}
    \item \against{PWLH}
    \item \against{PWLHInt}
    \item \against{GAMPS}
\end{itemize}


\vspace{+3pt}
For each algorithm $a \in A_M$, and each error parameter $e \in E$, we compare the performance of $a_\maskalgo$ and $a_\NOmaskalgo$. For the purpose of this comparison, we choose the most favorable window size for each variant $a_v$, in the sense of the following definition.


\vspace{+5pt}
\begin{defcion}
\label{def:ows}
The \textit{optimal window size (\owsit)} of a coding algorithm variant $a_v \in V$, and an error parameter $e \in E$, for the data type $z$ of a certain dataset $d$, is given by
\begin{equation}
\label{eq:ows}
\ows(a_v, e, z, d) = \argmin_{w\ \in \ W} \biggl\{ \CR(c_{<a_v, w, e>}, z, d) \biggr\},
\end{equation}
where we break ties in favor of the smallest window size.
\end{defcion}


For each data type $z$ of each dataset $d$, and each coding algorithm $a \in A_M$ and error parameter $e \in E$, we calculate the RD between $c_{<a_\maskalgo, w_\maskalgo^{*}, e>}$ and $c_{<a_\NOmaskalgo, w_\NOmaskalgo^{*}, e>}$, as defined in~(\ref{eq:relative-difference-dataset}), where $w_\maskalgo^{*}=\owsns(a_\maskalgo, e, z, d)$ and $w_\NOmaskalgo^{*}=\owsns(a_\NOmaskalgo, e, z, d)$.


\vspace{+2pt}
As an example, in figures~\ref{fig:diff-sst} and~\ref{fig:diff-tornado} we show the CR and the RD, as a function of the error parameter, obtained for two data types of two different datasets. Figure~\ref{fig:diff-sst} shows the results for the data type $z=$~``SST" of the dataset $d=$~\datasetsst, presented in Section~\ref{datasets:sst}, and Figure~\ref{fig:diff-tornado} shows the results for the data type $z=$~``Longitude" of the dataset $d=$~\datasettornado, presented in Section~\ref{datasets:tornado}. In Figure~\ref{fig:diff-sst} we observe a large RD favoring the masking variant for all tested algorithms. On the other hand, in Figure~\ref{fig:diff-tornado} we observe that the non-masking variant outperforms the masking variant for all algorithms. We notice, however, that the RD is very small in the latter case.


\clearpage

%%%%%%%%%%%%%%%%%%%%%%%%%%%%%%%%%%%%%
%%%%%%%%%%%%%%%% 3.2 %%%%%%%%%%%%%%%%
%%%%%%%%%%%%%%%%%%%%%%%%%%%%%%%%%%%%%
\newcommand{\threetwocommon}[4]{
In the relative difference plot for algorithm {#1} we made a {#2} circle around the marker with the {#3} value obtained for all of the tested CAIs ({#4}).
}
\newcommand{\threetwomost}{\threetwocommon{PCA}{red}{maximum}{50.60\%}}
\newcommand{\threetwoleast}{\threetwocommon{APCA}{blue}{minimum}{-0.29\%}}

\newcommand{\threetwosingle}[5]{
    \clearpage
    \begin{figure}
    \hspace{-90pt} % trim=left bottom right top
    \includegraphics[clip,trim=0 2.9cm 0 3.5cm,height=23.5cm]{appendices/1pdfs/{#1}}
    \hspace{+5pt}
    \caption{Compression ratio and relative difference plots for every pair of algorithm variants $a_\maskalgo, a_\NOmaskalgo \in A$, for the data type ``{#3}" of the dataset {#2}.{#4}}
    {#5}
    \end{figure}
}

%%%%%%%%%%%%%%%%%%%%%%%%%%%%%%%%%%%%%
%%%%%%%%%%%%%%%% 3.3 %%%%%%%%%%%%%%%%
%%%%%%%%%%%%%%%%%%%%%%%%%%%%%%%%%%%%%

\newcommand{\threethreecommon}[4]{
In the relative difference plot for algorithm {#1} we made a {#2} circle around the marker with the {#3} value obtained for all the tested CAIs ({#4}).
}

\newcommand{\threethreemost}{\threethreecommon{PCA}{red}{maximum}{10.68\%}}

\newcommand{\threethreesingle}[6]{
    \clearpage
    \begin{figure}
    \hspace{-100pt} % trim=left bottom right top
    \includegraphics[clip,trim=0 1.8cm 0 2.5cm,height=19.5cm]{appendices/2pdfs/{#1}}
    \hspace{+5pt}
    \caption{Global and local window sizes, and relative difference plots for every algorithm, for the data type ``{#2}" of the file ``{#3}" of the dataset {#4}.{#6}}
    {#5}
    \end{figure}
}

%%%%%%%%%%%%%%%%%%%%%%%%%%%%%%%%%%%%%
%%%%%%%%%%%%%%%% 3.4 %%%%%%%%%%%%%%%%
%%%%%%%%%%%%%%%%%%%%%%%%%%%%%%%%%%%%%

\newcommand{\threefoursingle}[5]{
    \clearpage
    \begin{figure}
    \hspace{-100pt} % trim=left bottom right top
    \includegraphics[clip,trim=0 1.8cm 0 2.5cm,height=19.5cm]{appendices/3pdfs/{#1}}
    \hspace{+5pt}
    \caption{Compression ratio and window parameter plots for every algorithm, for the data type ``{#3}" of the dataset {#2}.{#4}}
    {#5}
    \end{figure}
}

\threetwosingle{2-NOAA-SST-1}{\datasetsst}{SST}{\threetwomost}{\label{fig:diff-sst}}
\threetwosingle{7-NOAA-SPC-tornado-2}{\datasettornado}{Longitude}{\threetwoleast}{\label{fig:diff-tornado}}

\clearpage


\newcommand{\errParVal}{8 different error parameter values}
We analyze the experimental results to compare the performance of the masking and non-masking variants of each algorithm. For each data type, we iterate through each algorithm $a \in A_M$, and each error parameter $e \in E$, and we calculate the RD between the CAIs $c_{<a_\maskalgo, w_\maskalgo^{*}, e>}$ and $c_{<a_\NOmaskalgo, w_\NOmaskalgo^{*}, e>}$, obtained by setting the OWS for the masking variant $a_\maskalgo$ and the non-masking variant $a_\NOmaskalgo$, respectively. Since we consider \errParVal\ and there are 6 algorithms in $A_M$, for each data type we compare a total of 48 pairs of CAIs. Table~\ref{tabla:rendimiento-relativ-NM-M} summarizes the results of these comparisons, aggregated by dataset. The number of pairs of CAIs evaluated for each dataset depends on the number of different data types it contains.


\vspace{+5pt}

\begin{table}[h]
\begin{center}
    \begin{tabular}{| C{2.2cm} || C{2.5cm} | C{4.4cm} | C{3.0cm} |}
    \hline
      \multicolumn{1}{|>{\centering\arraybackslash}m{2.2cm}||}{\textbf{Dataset}} 
    & \multicolumn{1}{>{\centering\arraybackslash}m{2.5cm}|}{\textbf{Dataset Characterstic}} 
    & \multicolumn{1}{>{\centering\arraybackslash}m{4.4cm}|}{\textbf{Cases where masking outperforms non-masking variant (\%)}}
    & \multicolumn{1}{>{\centering\arraybackslash}m{3.0cm}|}{\textbf{RD (\%) Range}}\\
    \hline
    \datasetirkis   & Many gaps     & 100 & (0; 36.88]                    \\\hline
    \datasetsst     & Many gaps     & 100 & (0; \textcolor{red}{50.60}]  \\\hline
    \datasetadcp    & Many gaps     & 100 & (0; 17.35]                    \\\hline
    \datasetelnino  & Many gaps     & 100 & (0; 50.52]                    \\\hline
    \datasetsolar   & Few gaps      & 51  & [-0.25; 1.77]                 \\\hline
    \datasethail    & No gaps       & 0   & [-0.04; 0)                    \\\hline
    \datasettornado & No gaps       & 0   & [\textcolor{blue}{-0.29}; 0)   \\\hline
    \datasetwind    & No gaps       & 0   & [-0.12; 0)                    \\\hline
    \toprule[0.1mm]
    \end{tabular}
    \caption{Relative difference between the masking and non-masking variants of each algorithm. In the last column we highlight the maximum (red) and minimum (blue) values taken by RD.}
    \label{tabla:rendimiento-relativ-NM-M}
\end{center}
\end{table}

\vspace{-5pt}


Consider, for example, the results for the dataset Wind, in the last row. The second column shows that there are no gaps in any of the data types of the dataset (recall the dataset information from Table~\ref{datasets:table:overview}). Since the dataset has three data types, we compare a total of $3\times48=144$ pairs of CAIs. The third column reveals that in none of these comparisons the masking variant $a_\maskalgo$ outperforms the non-masking variant $a_\NOmaskalgo$, i.e. the RD is always negative. The last column shows the range for the values attained by the RD for those tested CAIs.


Observing the last column of Table~\ref{tabla:rendimiento-relativ-NM-M}, we notice that, in every case in which the non-masking variant performs best, the RD is close to zero. The minimum value it takes is -0.29\%, which is obtained for the data type ``Longitude" of the dataset \datasettornado, with algorithm APCA, and error parameter $e=30$. In Figure~\ref{fig:diff-tornado} we highlight the marker associated to this minimum with a blue circle. On the other hand, we observe that, for the datasets in which the masking variant performs best, the RD reaches high absolute values. The maximum (50.78\%) is obtained for the data type ``VWC" of the dataset \datasetsst, with algorithm PCA, and error parameter $e=30$, which is highlighted in Figure~\ref{fig:diff-sst} with a red circle.


\newcommand{\vaster}{V^*}
The experimental results presented in this section suggest that if we were interested in compressing a dataset with many gaps, we would benefit from using the masking variant of an algorithm, $a_\maskalgo$. However, even if the dataset didn't have any gaps, the performance would not be significantly worse than that obtained by using the non-masking variant of the algorithm, $a_\NOmaskalgo$. Therefore, since masking variants are, in general, more robust in this sense, in the sequel we focus on the set of variants $\vaster$ that we define next.


\vspace{+5pt}
\begin{defcion}
\label{defcion:vaster}
We denote by $\vaster$ the set of all the masking algorithm variants $a_\maskalgo$ for $a \in A$.
\end{defcion}


Notice that $\vaster$ includes a single variant for each algorithm. Therefore, in what follows we sometimes refer to the elements of $\vaster$ simply as algorithms.



\clearpage
\section{Window Size Parameter}
\label{secX:windows}

In this section, we analyze the extent to which the window size parameter impacts the performance of the coding algorithms. We only consider the four datasets that consist of multiple files, i.e. \datasetirkis, \datasetsst, \datasetadcp \ and \datasetsolar. For each file, we compare the compression performance when using the optimal window size for the dataset, as defined in (\ref{eq:ows}), and the local optimal window size, defined next.


\newcommand{\lows}{\textit{LOWS}}
\begin{defcion}
The \textit{local optimal window size (\lows)} of a coding algorithm $a \in A$ and a threshold parameter $e \in E$, for the data type $z$ of a certain file $f$ is given by
\begin{equation}
\lows(a, e, z, f) = \argmin_{w\ \in \ W} \biggl\{ \tasacompresion(c_{<a, w, e>}, z, f) \biggr\},
\end{equation}
where we break ties in favor of the smallest window size.
\end{defcion}


For each data type $z$ of each dataset $d$, and each file $f \in F(d, z)$, coding algorithm $a \in A$ and threshold parameter $e \in E$, we calculate the relative difference between $c_{<a, w_{global}^{*}, e>}$ and $c_{<a, w_{local}^{*}, e>}$, as defined in~(\ref{eq:relative-difference}), where $w_{global}^{*}=OWS(a, e, z, d)$ and $w_{local}^{*}=LOWS(a, e, z, f)$. In what follows, we refer to $w_{global}^{*}$ and $w_{local}^{*}$ as the global and local window size, respectively.


As an example, in Figures~\ref{fig:window-compare-1202} and~\ref{fig:window-compare-1203} we display the global and local window sizes and the relative difference, as a function of the threshold, obtained for the data type ``VWC", for two different files of the dataset \datasetirkis. Figure~\ref{fig:window-compare-1202} shows the results for the file ``vwc\_1202.dat.csv", while Figure~\ref{fig:window-compare-1203} shows the results for ``vwc\_1203.dat.csv". Observe that the global window sizes are repeated for every matching plot of both figures, which is expected, since both figures consider the same data type of the same dataset.


In Figure~\ref{fig:window-compare-1202} we notice, for instance, that in the APCA algorithm case both window sizes match for every threshold parameter $e$, except 3 and 10. The global window is larger than the local window when $e=3$, but it is smaller when $e=10$. In those two cases the relative difference values are 1.52 and 1.76, respectively. We observe that the relative difference is non-negative in every plot, which makes sense, since the compression ratio obtained when using the global window cannot be lower than the compression ratio obtained when using the local window.


\clearpage

\threethreesingle{1-IRKIS-1-1}{VWC}{vwc\_1202.dat.csv}{\datasetirkis}{\label{fig:window-compare-1202}}{}
\threethreesingle{1-IRKIS-2-1}{VWC}{vwc\_1203.dat.csv}{\datasetirkis}{\label{fig:window-compare-1203}}{\threethreemost}

\clearpage


NO LEER ESTA PARTE, TODAVÍA NO TERMINÉ.

TODO:
\vspace{-10pt}
\begin{itemize}
    \item Reescribir el párrafo que presenta a la tabla
    \item Analizar los datos de la tabla
    \item Escribir un párrafo final con conclusiones
\end{itemize}

Table~\ref{tabla:windows-comparison} summarizes the results obtained for each combination of algorithm, error threshold parameter, data type and file. In 88.7\% of the cases both window sizes match, and so the relative difference is 0. In the remaining cases, most of the times the relative difference is smaller than 1, and there are only six cases in which the relative difference is larger than 5.


\vspace{+5pt}




\begin{table}[h]

\begin{center}

    \begin{tabular}{| C{2.5cm} || C{2.2cm} | C{1.5cm} | C{1.5cm} | C{1.5cm} | C{1.5cm} |}

    \hline

    \multicolumn{1}{|>{\centering\arraybackslash}m{2.5cm}||}{}

    & \multicolumn{5}{>{\centering\arraybackslash}m{9cm}|}{RD (\%) Range}\\

    \hline

      \multicolumn{1}{|>{\centering\arraybackslash}m{2.5cm}||}{\textbf{Algorithm}}

    & \multicolumn{1}{>{\centering\arraybackslash}m{2.2cm}|}{\textbf{0}}

    & \multicolumn{1}{>{\centering\arraybackslash}m{1.5cm}|}{\textbf{(0,1]}}

    & \multicolumn{1}{>{\centering\arraybackslash}m{1.5cm}|}{\textbf{(1,2]}}

    & \multicolumn{1}{>{\centering\arraybackslash}m{1.5cm}|}{\textbf{(2,5]}}

    & \multicolumn{1}{>{\centering\arraybackslash}m{1.5cm}|}{\textbf{(5,11]}}\\

    \hline\hline

    PCA & 186 (93\%) & 3 (1.5\%) & 4 (2\%) & 2 (1\%) & 5 (2.5\%) \\\hline
    APCA & 174 (87\%) & 13 (6.5\%) & 7 (3.5\%) & 6 (3\%) & 0 \\\hline
    CA & 172 (86\%) & 16 (8\%) & 6 (3\%) & 6 (3\%) & 0 \\\hline
    FR & 171 (85.5\%) & 14 (7\%) & 8 (4\%) & 7 (3.5\%) & 0 \\\hline
    PWLH & 184 (92\%) & 13 (6.5\%) & 3 (1.5\%) & 0 & 0 \\\hline
    PWLHInt & 180 (90\%) & 8 (4\%) & 9 (4.5\%) & 3 (1.5\%) & 0 \\\hline
    GAMPS & 167 (83.5\%) & 15 (7.5\%) & 11 (5.5\%) & 3 (1.5\%) & 4 (2\%) \\\hline
    SF & 199 (99.5\%) & 1 (0.5\%) & 0 & 0 & 0 \\\hline\hline
    Total & 1,433 (89.5\%) & 83 (5.2\%) & 48 (3\%) & 27 (1.7\%) & 9 (0.6\%) \\\hline
    \toprule[0.1mm]

    \end{tabular}

    \caption{RD between the \ows and \lows variants of each CAI.\\The results are aggregated by algorithm and the range to which the RD belongs.}

    \label{tabla:windows-comparison}

\end{center}

\end{table}


\vspace{-5pt}


In Figure~\ref{fig:window-compare-1203} we display the plots obtained for the ``VWC" data type of the ``vwc\_1203.dat.csv" file. For CoderPCA and $e=15$ the relative difference is 10.68, which is the largest value obtained for all of the combinations. The next four largest relative differences (9.79, 9.22, 7.20, and 5.51) are also obtained with the CoderPCA algorithm. These results support the idea that the performance of the CoderPCA algorithm is more sensible to the window size parameter than the rest of the algorithms.

\section{Algorithms Performance}
\label{secX:codersmask}


In this section, we compare the compression performance of the coding algorithms presented in Chapter~\ref{algo}, by encoding the various datasets introduced in Chapter~\ref{datasets}. We begin by comparing the algorithms among each other and later we compare them with gzip, a popular lossless compression algorithm. We analyze the performance of the algorithms on complete datasets (not individual files), so we always apply definitions~\ref{eq:coding-size-dataset}--\ref{def:relative-difference-dataset}. Following the results obtained in sections~\ref{secX:rendimiento-relativo} and~\ref{secX:windows}, we only consider the masking variants of the evaluated algorithms (i.e. set $V^{*}$), and we always set the window size parameter to the \owsns\ (recall Definition \ref{def:ows}).


For each data type $z$ of each dataset $d$, and each coding algorithm variant $a_v \in V^{*}$ and threshold parameter $e \in E$, we calculate the CR of $c_{<a_v, w_{global}^{*}, e>}$, as defined in (\ref{eq:compression-rate-dataset}), where \WGlobal\~$=\owsns(a_v, e, z, d)$. The following definition is useful for analyzing which CAI obtains the best compression result for a specific data type.


\begin{defcion}
\label{def:bestcai}
Let $z$ be a data type of a certain dataset $d$, and let $e \in E$ be a threshold parameter. We denote by $c^{b}(z, d, e)$ the \textit{best CAI} for $z, d, e$, and define it as the CAI that minimizes the CR among all the CAIs in CI, i.e.,
\vspace{-3pt}
\begin{equation}
\label{eq:eqbestcai}
c^{b}(z, d, e) = \argmin_{\algo \ \in \ca} \biggl\{ \tasacompresion(c_{<a_v, w_{global}^{*}, e>}, z, d) \biggr\}.
\end{equation}
When $c^{b}(z, d, e) = c_{<a^{b}_v, w_{global}^{b*}, e>}$, we refer to $a^{b}$ and $w_{global}^{b*}$ as the \textit{best coding algorithm} and the \textit{best window size} for $z, d, e$, respectively.
\end{defcion}


Our experiments include a total of 21 data types, in 8 datasets. As an example, in Figure~\ref{fig:algo-per-1} we show the CR and the window size parameter \WGlobal, as a function of the threshold parameter, obtained for each algorithm, for the data type ``SST" of the dataset \datasetelnino. For each threshold parameter $e \in E$, we use blue circles to highlight the markers for the minimum CR value and the best window parameter (in the respective plots corresponding to the best algorithm). For instance, for $e=0$, the best CAI achieves a CR equal to $0.33$ using algorithm PCA with a window size of 256. So in this case, algorithm PCA is the best coding algorithm, and 256 is the best window size. For the remaining seven values for the threshold parameter, the blue circles indicate that in every case the best algorithm is APCA, and the best window size ranges from 4 up to 32.


\clearpage
\threefoursingle{5-ElNino-7}{\datasetelnino}{SST}{\ For each threshold parameter $e \in E$, we use blue circles to highlight the markers for the minimum CR value and the best window size parameter (in the respective plots corresponding to the best algorithm)}{\label{fig:algo-per-1}}
\clearpage


Table~\ref{experiments:mask-results-overview1} summarizes the compression performance results obtained by the evaluated coding algorithms, for each data type of each dataset. Each row contains information relative to certain data type. For example, the 13th row shows summarized results for the data type ``SST" of the dataset \datasetelnino, which are presented in more detail in Figure~\ref{fig:algo-per-1}. For each threshold, the first column shows the CR obtained by the best CAI, the second column shows the base-2 logarithm of its window size parameter, and the cell color identifies the best algorithm.
\vspace{+5pt}


\newcommand{\legendsone}{
\begin{tabular}{| C{1.5cm} | C{1.5cm} | C{1.5cm} | C{1.5cm} |}
\hline
  \multicolumn{1}{|>{\centering\arraybackslash}m{1.5cm}|}{\cpca PCA} 
& \multicolumn{1}{>{\centering\arraybackslash}m{1.5cm}|}{\capca APCA} 
& \multicolumn{1}{>{\centering\arraybackslash}m{1.5cm}|}{\cca CA} 
& \multicolumn{1}{>{\centering\arraybackslash}m{1.5cm}|}{\cfr FR}\\
\toprule[0.1mm]
\end{tabular}
\vspace{+10pt}

}

\newcommand{\legendstwo}{
\begin{tabular}{| C{1.5cm} | C{1.5cm} | C{1.5cm} | C{1.5cm} |}
% \begin{tabular}{| C{1.5cm} | C{1.5cm} | C{1.5cm} | C{1.5cm} | C{1.5cm} |}
\hline
  \multicolumn{1}{|>{\centering\arraybackslash}m{1.5cm}|}{\cgzip GZIP} 
& \multicolumn{1}{>{\centering\arraybackslash}m{1.5cm}|}{\cpca PCA} 
& \multicolumn{1}{>{\centering\arraybackslash}m{1.5cm}|}{\capca APCA} 
% & \multicolumn{1}{>{\centering\arraybackslash}m{1.5cm}|}{\cca CA} 
& \multicolumn{1}{>{\centering\arraybackslash}m{1.5cm}|}{\cfr FR}\\
\toprule[0.1mm]
\end{tabular}
\vspace{+10pt}

}

\newcommand{\FirstSentence}{Compression performance of the best evaluated coding algorithm, for various error values on each data type of each dataset}
\newcommand{\SecondSentence}{Each row contains information relative to certain data type. For each error parameter value, the first column shows the minimum CR, and the second column shows the base-2 logarithm of the best window size for the best algorithm (the one that achieves the minimum CR), which is identified by a certain cell color described in the legend above the table}

\newcommand{\captionzero}{Lossless compression performance of algorithm variants \NonMaskVar{PCA} and \NonMaskVar{APCA}, for the timestamp column of each dataset. For each variant, the first column shows the minimum CR, and the second column shows the base-2 logarithm of the best window size (the window size that achieves the minimum CR). For each dataset, the cells corresponding to the algorithm that obtains the best result are colored, and the last column shows the RD between the corresponding CAIs, as defined in (\ref{eq:relative-difference-dataset}).
}

\newcommand{\captionone}{
\FirstSentence. \SecondSentence.
}

\newcommand{\captiontwo}{
\FirstSentence, including the results obtained by gzip. Each row contains information relative to certain data type. \SecondSentence. Algorithm gzip doesn't have a window size parameter, so the cell is left blank in these cases.
}

\newcommand{\captionminmaxone}{
$\text{\maxRD}(a, e)$ obtained for every pair of coding algorithm variant $a_v \in V^*$ and error parameter $e \in E$. For each $e$, the cell corresponding to the $\text{\minmaxRD}(a)$ value is highlighted.  
}

\newcommand{\captionminmaxtwo}{
$\text{\maxRD}(a, e)$ obtained for every pair of coding algorithm variant $a_v \in V^* \cup \{\text{gzip}\}$ and error parameter $e \in E$. For each $e$, the cell corresponding to the $\text{\minmaxRD}(a)$ value is highlighted.  
}


\begin{sidewaystable}[ht]
\newcommand{\cpca}{\cellcolor{cyan!20}}
\newcommand{\capca}{\cellcolor{green!20}}
\newcommand{\cfr}{\cellcolor{yellow!25}}
\newcommand{\cgzip}{\cellcolor{orange!20}}
\newcommand{\cpwlhint}{\cellcolor{violet!25}}
\newcommand{\cpwlh}{\cellcolor{violet!50}}
\newcommand{\cca}{\cellcolor{brown!20}}
\centering
\legendsone
\begin{tabular}{| l | l | c | c || c | c || c | c || c | c || c | c || c | c || c | c || c | c |}
\cline{3-18}
\multicolumn{1}{c}{}& \multicolumn{1}{c|}{} & \multicolumn{2}{c||}{e = 0} & \multicolumn{2}{c||}{e = 1} & \multicolumn{2}{c||}{e = 3} & \multicolumn{2}{c||}{e = 5} & \multicolumn{2}{c||}{e = 10} & \multicolumn{2}{c||}{e = 15} & \multicolumn{2}{c||}{e = 20} & \multicolumn{2}{c|}{e = 30} \\\hline
{Dataset} & {Data Type} & {\footnotesize CR} & {\footnotesize w} & {\footnotesize CR} & {\footnotesize w} & {\footnotesize CR} & {\footnotesize w} & {\footnotesize CR} & {\footnotesize w} & {\footnotesize CR} & {\footnotesize w} & {\footnotesize CR} & {\footnotesize w} & {\footnotesize CR} & {\footnotesize w} & {\footnotesize CR} & {\footnotesize w} \\\hline\hline



We observe that there are only three algorithms (PCA, APCA, and FR) which are used by the best CAI for at least one of the 168 possible data type and threshold parameter combinations. Algorithm APCA is used in exactly 134 combinations ($80\%$), including every case in which $e \geq 10$, and most of the cases in which $e \in [1, 3, 5]$. PCA is used in 31 combinations ($18\%$), including most of the lossless cases, while FR is the best algorithm in only 3 combinations ($2\%$), all of them for data type ``Speed" of the dataset \datasetwind.


Since there is not a single algorithm that obtains the best compression performance for every data type, it is useful to analyze how much is the RD between the best algorithm and the rest, for every experimental combination. With that in mind, next we define a pair of metrics.


\clearpage


\newcommand{\maxRD}{maxRD}
\newcommand{\maxRDit}{\textit{maxRD}}
\begin{defcion}
\label{eq:maxRD}
The \textit{maximum RD} (\maxRDit) of a coding algorithm $a\in A$ for certain threshold parameter $e\in E$ is given by
\vspace{-4pt}
\begin{equation}
\maxRDit(a, e) = \maxi_{z, d} \ \biggl\{ \difrelativa(c^{b}(z,d,e), c_{<a_v, w_{global}^{*}, e>}) \biggr\},
\end{equation}
where the maximum is taken over all the combinations of data type $z$ and dataset $d$, and we recall that $c^{b}(z,d,e)$ is the best CAI for $z, d, e$.
\end{defcion}


The \maxRD\ metric is useful for assessing the compression performance of a coding algorithm $a$ on the set of data types as a whole. Notice that \maxRD\ is always non-negative. A satisfactory result (i.e. close to zero) can only be obtained when $a$ achieves a good compression performance \textit{for every data type}. In other words, bad compression performance \textit{on a single data type} yields a poor result for the \maxRD\ metric altogether. When \maxRD\ is equal to zero, $a$ achieves the best compression performance for every combination. Analyzing the results in Table~\ref{experiments:mask-results-overview1}, we observe that $\text{\maxRD}(\text{APCA}, e)=0$ for every $e \geq 10$. Since the best algorithm is unique for every combination (i.e. exactly one algorithm obtains the minimum CR in every case), it is also true that, when $a \neq \text{APCA}$, $\text{\maxRD}(a, e)>0$ for every $e \geq 10$.


\vspace{+10pt}
\newcommand{\minmaxRD}{minmaxRD}
\newcommand{\minmaxRDit}{\textit{minmaxRD}}
\newcommand{\minmaxca}{minmax coding algorithm}
\begin{defcion}
\label{eq:minmaxRD}
The \textit{minmax RD} (\minmaxRDit) for certain threshold parameter $e\in E$ is given~by
\vspace{-4pt}
\begin{equation}
\minmaxRDit(e) = \mini_{a \in A} \ \biggl\{ \maxRDit(a, e) \biggr\},
\end{equation}
and we refer to $\argmin_{a \in A}$ as the \textit{\minmaxca} for $e$.
\end{defcion}


\vspace{-3pt}
Again, \minmaxRD\ is always always non-negative. Notice that $\text{\minmaxRD}(e)=0$ for certain $e$, if and only if there exists a \minmaxca\ $a$ such that $\text{\maxRD}(a, e)=0$. Continuing the analysis from the previous paragraph, it should be clear that APCA is the \minmaxca\ for every $e \geq 10$, since $\text{\maxRD}(\text{APCA}, e)=0$ for every $e \geq 10$.


Table~\ref{experiments:minmaxone} shows the $\text{\maxRD}(a, e)$ obtained for every pair of coding algorithm variant $a_v \in V^*$ and threshold parameter $e \in E$. For each $e$, the cell corresponding to the $\text{\minmaxRD}(e)$ value (i.e. the minimum value in the column) is highlighted.
\vspace{+3pt}


\begin{table}[h]
\newcommand{\cpca}{\cellcolor{cyan!20}}
\newcommand{\capca}{\cellcolor{green!20}}
\newcommand{\cfr}{\cellcolor{yellow!25}}
\newcommand{\cgzip}{\cellcolor{orange!20}}
\newcommand{\best}{\cellcolor{gray!30}}
\centering\hspace*{0cm}\begin{tabular}{| l | c | c | c | c | c | c | c | c |}\cline{2-9}\multicolumn{1}{c|}{}& \multicolumn{8}{c|}{maxRD (\%)}\\\hline
{Algorithm} & {e = 0} & {e = 1} & {e = 3} & {e = 5} & {e = 10} & {e = 15} & {e = 20} & {e = 30} \\\hline
{PCA\cpca} & {40.51} & {42.27} & {53.11} & {62.01} & {71.71} & {75.28} & {77.15} & {80.21} \\\hline
{APCA\capca} & {33.25} & {\best15.64} & {\best9.00} & {\best29.96} & {\best0} & {\best0} & {\best0} & {\best0} \\\hline
{CA} & {38.28} & {38.28} & {54.68} & {63.12} & {65.44} & {72.94} & {77.21} & {81.84} \\\hline
{PWLH} & {73.46} & {72.93} & {72.52} & {82.14} & {83.24} & {86.86} & {88.94} & {91.19} \\\hline
{PWLHInt} & {\best29.72} & {34.00} & {49.94} & {68.95} & {76.68} & {69.96} & {74.72} & {79.89} \\\hline
{FR\cfr} & {48.75} & {49.85} & {52.21} & {52.70} & {54.82} & {55.35} & {54.48} & {64.72} \\\hline
{SF} & {85.84} & {85.77} & {85.26} & {85.17} & {84.36} & {83.64} & {83.17} & {82.88} \\\hline
{GAMPS} & {49.62} & {57.46} & {79.36} & {86.57} & {92.99} & {95.23} & {96.38} & {97.39} \\\hline
\end{tabular}
\caption{\captionminmaxone}
\label{experiments:minmaxone}
\end{table}



% \vspace{+3pt}
In the lossless case, PWLHInt is the minmax coding algorithm, with \minmaxRD \ being equal to $29.72\%$. This value is rather high, which means that none of the considered algorithms achieves a CR that is close to the minimum simultaneously \textit{for every data type}. 
% Algorithms APCA ($33.25\%$) and CA ($38.28\%$) obtain the second and third best \maxRD\ values, respectively. 
Recalling the results from Table~\ref{experiments:mask-results-overview1} we notice that $e=0$ is the only threshold parameter value for which the minmax coding algorithm doesn't obtain the minimum CR in any combination. In other words, when $e=0$, PWLHInt is the algorithm that minimizes the RD with the best algorithm among every data type, even though it itself is not the best algorithm for any data type.


When $e \in [1, 3, 5]$, the minmax coding algorithm is always APCA, and the \minmaxRD \ values are $15.64\%$, $9.00\%$ and $29.96\%$, respectively. Again, these values are fairly high, so we would select the most convenient algorithm depending on the data type we want to compress. 
%The second best \maxRD\ values are obtained by algorithms PWLHInt ($34.00\%$ and $49.94\%$) and FR ($52.70\%$), respectively. 
Notice that in the closest case (algorithm FR for $e=5$), the second best \maxRD\ ($52.70\%$) is about $75\%$ larger than the \minmaxRD, which is a much bigger difference than in the lossless case.


When $e \geq 10$, the minmax coding algorithm is also always APCA, but in these cases the \minmaxRD \ values are always 0. In the closest case (algorithm FR for $e=20$) the second best \maxRD\ is $54.48\%$. If we wanted to compress any data type with any of these threshold parameter values, we would pick algorithm APCA, since according to our experimental results, it always obtains the best compression results with a significant difference over the remaining algorithms.


