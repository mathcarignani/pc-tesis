
\section{Experimental Setting}
\label{experiments:experiments}


We evaluate the compression performance of all the coding algorithms presented in Chapter~\ref{algo} on the datasets described in Chapter~\ref{datasets}. For each algorithm we test both the masking and the non-masking variants (except for \textit{Base}, \textit{FR} and \textit{SF}, which do not admit a masking variant).


We also test several combinations of algorithm parameters. Specifically, for the algorithms that admit a window size parameter $w$ (every algorithm except \textit{Base} and \textit{SF}), we test all the values of $w$ in the set $W = \{4, 8, 16, 32, 64, 128, 256\}$. For the encoders that admit a lossy compression mode with a threshold parameter $e$ (every encoder except \textit{Base}), we test all the values of $e$ in the set $E= \{1, 3, 5, 10, 15, 20, 30\}$, where each threshold is expressed as a percentage fraction of the standard deviation of the data being coded. For example, for certain data with a standard deviation of 20, taking $e=10$ implies that the lossy compression allows for a maximum sample distortion of 2 sampling units.


\vspace{+5pt}
\begin{defcion}
We refer to a specific combination of a coding algorithm and its parameter values as a \textit{coding algorithm instance (CAI)}. We define \textit{CI} as the set of all the CAIs obtained by combining each of the algorithms presented in Chapter~\ref{algo} with the parameter values (from $W$ and $E$) that are suitable for that algorithm. We denote by $c_{<a, w, e>}$ the CAI obtained by setting a window size parameter equal to $w$ and a threshold parameter equal to $e$ on algorithm $a$.
\end{defcion}


\clearpage


We assess the compression performance of a CAI mainly through the compression ratio, which we define next. For this definition, we regard \textit{Base} as a trivial CAI that serves as a base ground for compression performance comparison (recall the definition of algorithm \textit{Base} from Section~\ref{algo:base}).


\vspace{+5pt}
\begin{defcion}
Let $f$ be a file and $z$ a data type of a certain dataset. We define $f_z$ as the subset of data of type $z$ from file $f$.
\end{defcion}


\vspace{+2pt}
\begin{defcion}
\label{eq:coding-size}
Let $f$ be a file and $z$ a data type of a certain dataset. Let $\algo \in \ca$ be a CAI. We define $|\algo(z, f)|$ as the size of the resulting file obtained when using coding $f_z$ with $\algo$.
\end{defcion}


\vspace{+2pt}
\begin{defcion}
The \textit{compression ratio (CR)} of a CAI $\algo \in \ca$ \ for the data type $z$ of a certain file $f$ is the percentual fraction of $|\algo(z, f)|$ with respect to $|\coderBase(z, f)|$, i.e.
\vspace{-5pt}
\begin{equation}
\label{eq:compression-rate}
\tasacompresion(\algo, z, f) = 100\times\frac{|\algo(z, f)|}{|\coderBase(z, f)|}.
\end{equation}
\end{defcion}


Notice that smaller values of CR imply better performance. Thus, our main goals are to analyze which CAIs minimize (\ref{eq:compression-rate}) for the different data types, and to study how the CR depends on the different algorithms and parameter values.


To compare the compression performance between a pair of CAIs we calculate the relative difference, which we define next. In general, it only makes sense to compare CAIs which have the same threshold parameter $e$.


\vspace{+5pt}
\begin{defcion}
\label{relative-difference}
The \textit{relative difference (RD)} between a pair of CAIs $\algo_1, \algo_2 \in {\ca}$ \ for the data type $z$ of a certain file $f$ is given by
\vspace{-5pt}
\begin{equation}
\label{eq:relative-difference}
\difrelativa(\algo_1, \algo_2, z, f)  =
100\times\frac{|\algo_2 (z, f)| - |\algo_1 (z, f)|}{ |\algo_2 (z, f)| }.
\end{equation}
\end{defcion}


Notice that $\algo_1$ has a better performance than $\algo_2$ when (\ref{eq:relative-difference}) is positive.


\vspace{+3pt}
In some of our experiments we consider the performance of algorithms on datasets as a whole, rather than individual files. With that in mind, we extend the definitions~\ref{eq:coding-size}-\ref{relative-difference} to datasets as follows.


\vspace{+5pt}
\begin{defcion}
\label{eq:coding-size-dataset}
Let $z$ be a data type of a certain dataset $d$. We define $F(d, z)$ as the set of files $f$ from dataset $d$ for which $f_z$ is not empty.
\end{defcion}


\begin{defcion}
Let $z$ be a data type of a certain dataset $d$. Let $\algo \in \ca$ \ be a CAI. We define $|\algo(z, d)|$ as follows
\vspace{-5pt}
\begin{equation}
\label{eq:dataset-size}
|\algo(z, d)|  = \sum_{f \in F(d, z)}^{} |\algo(z, f)|.
\end{equation}
\end{defcion}


\vspace{+3pt}
\begin{defcion}
The \textit{compression ratio (CR)} of a CAI $\algo \in \ca$ \ for the data type $z$ of a certain dataset $d$ is given by
\vspace{-5pt}
\begin{equation}
\label{eq:compression-rate-dataset}
\tasacompresion(\algo, z, d) = 100\times\frac{|\algo(z, d)|}{|\coderBase(z, d)|}.
\end{equation}
\end{defcion}


\vspace{+3pt}
\begin{defcion}
\label{def:relative-difference-dataset}
The \textit{relative difference (RD)} between a pair of CAIs $\algo_1, \algo_2 \in {\ca}$ \ for the data type $z$ of a certain dataset $d$ is given by
\vspace{-5pt}
\begin{equation}
\label{eq:relative-difference-dataset}
\difrelativa(\algo_1, \algo_2, z, d)  =
100\times\frac{|\algo_2 (z, d)| - |\algo_1 (z, d)|}{ |\algo_2 (z, d)| }.
\end{equation}
\end{defcion}


\clearpage