\clearpage

HECHO INFORME:

- Elegir nomenclatura para los dos distintos modos de ejecución => CoderPCA-NM (sin máscara) y CoderPCA-M (con máscara).

- Realizar un análisis cuantitativo para saber qué tanto mejor comprime el modo MM=0 en los pocos casos en los que funciona mejor que el modo MM=3. Vimos que esos casos se dan en los datasets con pocos o ningún gap, y la diferencia en las tasas de compresión es mínima. En cambio, cuando hay gaps en los datasets, la diferencia relativa de rendimiento a favor del modo MM=3 es mayor. Escribir un párrafo con dicho análisis, incluyendo alguna gráfica como ejemplo.

- Agregar tabla con resumen de los datasets - ver AVANCES / DUDAS (13)

- Poner las gráficas horizontales, 3 arriba y 3 abajo.

- Mencionar que CoderSlideFilter no tiene en cuenta el parámetro con el tamaño máximo de la ventana.

% \clearpage
\vspace{+20pt}

TODO INFORME:

- Vimos que en los datasets sin gaps, en general para todas las combinaciones <tipo de dato, algoritmo> la diferencia relativa no crece al aumentar el umbral de error. Escribir un párrafo explicando el por qué de este comportamiento.

- Mencionar experimentos ventana local vs ventana global. (ver minuta de la reunión del lunes 10/06/2019).

- Mencionar relación de compromiso entre el umbral y la tasa de compresión: al aumentar el umbral mejor la tasa de compresión (lógico).



- Agregar tabla con resumen de los algoritmos.

- Subir todo el material complementario en un link (después referirlo en el informe)

\clearpage

TODO CÓDIGO:

- Para los experimentos sin máscara no se están considerando los datos para los algoritmos CoderFractalRestampling y CoderSlideFilter.

- Agregar tests para MM=3.

- Al ejecutar los algoritmos GAMPS/GAMPSLimit sobre el dataset de "El Niño" (546 columnas) tengo problemas de memoria en Ubuntu, pero no en la Mac.

- Universalizar algoritmo

- Modificar GAMPS/GAMPSLimit para que utilice floats (4 bytes) en vez de doubles (8 bytes). De todas maneras, no creo que esto cambie los resultados de manera significativa, ya que aun si la cantidad de bits utilizados al codificar con GAMPS/GAMPSLimit fuera la mitad, en ningún caso superaría la tasa obtenida con el mejor codificador.