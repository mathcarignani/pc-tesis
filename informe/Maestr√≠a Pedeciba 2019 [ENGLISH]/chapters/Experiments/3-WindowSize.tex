
\clearpage
\section{Window Size Parameter}
\label{secX:windows}

In this section, we analyze the extent to which the window size parameter impacts the performance of the coding algorithms. We only consider the four datasets that consist of multiple files, i.e. \datasetirkis, \datasetsst, \datasetadcp \ and \datasetsolar. For each file, we compare the compression performance when using the optimal window size for the dataset, as defined in (\ref{eq:ows}), and the local optimal window size, defined next.


\newcommand{\lows}{\textit{LOWS}}
\begin{defcion}
The \textit{local optimal window size (\lows)} of a coding algorithm $a \in A$ and a threshold parameter $e \in E$, for the data type $z$ of a certain file $f$ is given by
\begin{equation}
\lows(a, e, z, f) = \argmin_{w\ \in \ W} \biggl\{ \tasacompresion(c_{<a, w, e>}, z, f) \biggr\},
\end{equation}
where we break ties in favor of the smallest window size.
\end{defcion}


For each data type $z$ of each dataset $d$, and each file $f \in F(d, z)$, coding algorithm $a \in A$ and threshold parameter $e \in E$, we calculate the relative difference between $c_{<a, w_{global}^{*}, e>}$ and $c_{<a, w_{local}^{*}, e>}$, as defined in~(\ref{eq:relative-difference}), where $w_{global}^{*}=OWS(a, e, z, d)$ and $w_{local}^{*}=LOWS(a, e, z, f)$. In what follows, we refer to $w_{global}^{*}$ and $w_{local}^{*}$ as the global and local window size, respectively.


As an example, in Figures~\ref{fig:window-compare-1202} and~\ref{fig:window-compare-1203} we display the global and local window sizes and the relative difference, as a function of the threshold, obtained for the data type ``VWC", for two different files of the dataset \datasetirkis. Figure~\ref{fig:window-compare-1202} shows the results for the file ``vwc\_1202.dat.csv", while Figure~\ref{fig:window-compare-1203} shows the results for ``vwc\_1203.dat.csv". Observe that the global window sizes are repeated for every matching plot of both figures, which is expected, since both figures consider the same data type of the same dataset.


In Figure~\ref{fig:window-compare-1202} we notice, for instance, that in the APCA algorithm case both window sizes match for every threshold parameter $e$, except 3 and 10. The global window is larger than the local window when $e=3$, but it is smaller when $e=10$. In those two cases the relative difference values are 1.52 and 1.76, respectively. We observe that the relative difference is non-negative in every plot, which makes sense, since the compression ratio obtained when using the global window cannot be lower than the compression ratio obtained when using the local window.


\clearpage

\threethreesingle{1-IRKIS-1-1}{VWC}{vwc\_1202.dat.csv}{\datasetirkis}{\label{fig:window-compare-1202}}{}
\threethreesingle{1-IRKIS-2-1}{VWC}{vwc\_1203.dat.csv}{\datasetirkis}{\label{fig:window-compare-1203}}{\threethreemost}

\clearpage


We analyze the experimental results to evaluate how much does using the global instead of the local window reduces the compression performance of the coding algorithms. The experiments consider 1,400 pairs of CAIs, since there are 200 pairs for each of the seven algorithms. In 1,242 (88.7\%) cases the global and local windows match, so the compression performance is not affected, i.e. RD is~0. In the remaining 158 (11.3\%) cases both windows do not match, so the CAI with the local window outperforms the CAI with the global window and RD is greater than 0. These results are summarized in Table~\ref{tabla:windows-comparison}, in which we aggregate them by algorithm and according to the range to which values taken by RD belong.


\vspace{+5pt}




\begin{table}[h]

\begin{center}

    \begin{tabular}{| C{2.5cm} || C{2.2cm} | C{1.5cm} | C{1.5cm} | C{1.5cm} | C{1.5cm} |}

    \hline

    \multicolumn{1}{|>{\centering\arraybackslash}m{2.5cm}||}{}

    & \multicolumn{5}{>{\centering\arraybackslash}m{9cm}|}{RD (\%) Range}\\

    \hline

      \multicolumn{1}{|>{\centering\arraybackslash}m{2.5cm}||}{\textbf{Algorithm}}

    & \multicolumn{1}{>{\centering\arraybackslash}m{2.2cm}|}{\textbf{0}}

    & \multicolumn{1}{>{\centering\arraybackslash}m{1.5cm}|}{\textbf{(0,1]}}

    & \multicolumn{1}{>{\centering\arraybackslash}m{1.5cm}|}{\textbf{(1,2]}}

    & \multicolumn{1}{>{\centering\arraybackslash}m{1.5cm}|}{\textbf{(2,5]}}

    & \multicolumn{1}{>{\centering\arraybackslash}m{1.5cm}|}{\textbf{(5,11]}}\\

    \hline\hline

    PCA & 186 (93\%) & 3 (1.5\%) & 4 (2\%) & 2 (1\%) & 5 (2.5\%) \\\hline
    APCA & 174 (87\%) & 13 (6.5\%) & 7 (3.5\%) & 6 (3\%) & 0 \\\hline
    CA & 172 (86\%) & 16 (8\%) & 6 (3\%) & 6 (3\%) & 0 \\\hline
    FR & 171 (85.5\%) & 14 (7\%) & 8 (4\%) & 7 (3.5\%) & 0 \\\hline
    PWLH & 184 (92\%) & 13 (6.5\%) & 3 (1.5\%) & 0 & 0 \\\hline
    PWLHInt & 180 (90\%) & 8 (4\%) & 9 (4.5\%) & 3 (1.5\%) & 0 \\\hline
    GAMPS & 167 (83.5\%) & 15 (7.5\%) & 11 (5.5\%) & 3 (1.5\%) & 4 (2\%) \\\hline
    SF & 199 (99.5\%) & 1 (0.5\%) & 0 & 0 & 0 \\\hline\hline
    Total & 1,433 (89.5\%) & 83 (5.2\%) & 48 (3\%) & 27 (1.7\%) & 9 (0.6\%) \\\hline
    \toprule[0.1mm]

    \end{tabular}

    \caption{RD between the \ows and \lows variants of each CAI.\\The results are aggregated by algorithm and the range to which the RD belongs.}

    \label{tabla:windows-comparison}

\end{center}

\end{table}


\vspace{-5pt}


Observing Table~\ref{tabla:windows-comparison} we notice that, not only in 88.7\% of the cases RD is equal to 0, but also, in 97.8\% of the cases RD is less than or equal to 2. This means that, for the vast majority of CAI pairs, either the global and local windows match or the compression performance is not affected considerably when they don't. This result suggests that we could always set the global window as the window size parameter without this having a significant negative impact on the compression performance of the coding algorithm. This is meaningful, since calculating the LOWS for a file is both a time and computationally expensive process. Having a predefined global windows table would improve the compression technique.


Furthermore, we notice that there are only 6 (0.4\%) cases in which the values taken by RD fall in the (5, 11] range. These are the cases in which the CAIs with the global window achieve the worst relative compression. The maximum value taken by RD (10.68\%) is obtained for the data type ``VWC" of the file ``vwc\_1203.dat.csv" of the dataset \datasetsst, with the PCA algorithm and error parameter $e=15$. In Figure~\ref{fig:window-compare-1203} we made a red circle around the marker associated to that maximum value. Note that in that case the global window is 16 and the local window is 8. The next four largest values taken by RD (9.79\%, 9.22\%, 7.20\% and 5.51\%) are also cases in which the algorithm is PCA. These results support the notion that the performance of the PCA algorithm is more sensible to the window size parameter than the rest of the algorithms.


The experimental results presented in this section suggest that, in general, using the global instead of the local window affects the compression performance of the coding algorithm in a negligible way. In the following section, in which we compare the algorithms performance, we always use the global window. However, we must keep in mind that the window parameter appears to degrade the compression performance of the PCA algorithm to a higher degree than the rest of the algorithms.

