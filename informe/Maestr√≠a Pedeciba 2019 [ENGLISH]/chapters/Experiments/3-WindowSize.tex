
\clearpage
\section{Window Size Parameter}
\label{secX:windows}


In this section, we analyze the extent to which the window size parameter impacts the performance of the coding algorithms. We only consider the four datasets that consist of multiple files, i.e. \datasetirkis, \datasetsst, \datasetadcp \ and \datasetsolar. For each file, we compare the compression performance when using the optimal window size for the dataset, as defined in (\ref{eq:ows}), and the local optimal window size, defined next.


\newcommand{\lows}{\textit{LOWS}}
\begin{defcion}
The \textit{local optimal window size (\lows)} of a coding algorithm $a \in A$ and a threshold parameter $e \in E$, for the data type $z$ of a certain file $f$ is given by
\begin{equation}
\lows(a, e, z, f) = \argmin_{w\ \in \ W} \biggl\{ \tasacompresion(c_{<a, w, e>}, z, f) \biggr\},
\end{equation}
where we break ties in favor of the smallest window size.
\end{defcion}


For each data type $z$ of each dataset $d$, and each file $f \in F(d, z)$, coding algorithm $a \in A$ and threshold parameter $e \in E$, we calculate the relative difference between $c_{<a, w_{global}^{*}, e>}$ and $c_{<a, w_{local}^{*}, e>}$, as defined in~(\ref{eq:relative-difference}), where $w_{global}^{*}=OWS(a, e, z, d)$ and $w_{local}^{*}=LOWS(a, e, z, f)$. In what follows, we refer to $w_{global}^{*}$ and $w_{local}^{*}$ as the global and local window size, respectively.


As an example, in Figures~\ref{fig:window-compare-1202} and~\ref{fig:window-compare-1203} we show the global and local window sizes and the relative difference, as a function of the threshold, obtained for the data type ``VWC", for two different files of the dataset \datasetirkis. Figure~\ref{fig:window-compare-1202} shows the results for the file ``vwc\_1202.dat.csv", and Figure~\ref{fig:window-compare-1203} shows the results for ``vwc\_1203.dat.csv". Observe that the global window sizes are the same for both figures, which is expected, since both are obtained from the same data type of the same dataset.


In Figure~\ref{fig:window-compare-1202} we notice, for instance, that in the algorithm APCA both window sizes match for every threshold parameter $e$, except 3 and 10. The global window is larger than the local window when $e=3$, but it is smaller when $e=10$. In these two cases, the relative difference values are 1.52 and 1.76, respectively. We observe that the relative difference is non-negative in every plot, which makes sense, since the compression ratio obtained when using the global window cannot be lower than the compression ratio obtained when using the local window.


\clearpage

\threethreesingle{1-IRKIS-1-1}{VWC}{vwc\_1202.dat.csv}{\datasetirkis}{\label{fig:window-compare-1202}}{}
\threethreesingle{1-IRKIS-2-1}{VWC}{vwc\_1203.dat.csv}{\datasetirkis}{\label{fig:window-compare-1203}}{\threethreemost}

\clearpage


We analyze the experimental results to evaluate the impact of using a global window instead of a local window on the compression performance of the different coding algorithms. For each algorithm, we iterate through each threshold parameter, and each data type of each file, and calculate the relative difference between the CAI with the global window and the CAI with the local window. Since we consider 8 threshold parameters and there are 13 files with a single data type and 4 files with 3 different data types each, for each algorithm we compare a total of $8 \times (13 + 4\times3) = 200$ pairs of CAIs. Table~\ref{tabla:windows-comparison} summarizes the results of these comparisons, aggregated by algorithm and according to the range to which values taken by the relative difference belong. 


\vspace{+5pt}




\begin{table}[h]

\begin{center}

    \begin{tabular}{| C{2.5cm} || C{2.2cm} | C{1.5cm} | C{1.5cm} | C{1.5cm} | C{1.5cm} |}

    \hline

    \multicolumn{1}{|>{\centering\arraybackslash}m{2.5cm}||}{}

    & \multicolumn{5}{>{\centering\arraybackslash}m{9cm}|}{RD (\%) Range}\\

    \hline

      \multicolumn{1}{|>{\centering\arraybackslash}m{2.5cm}||}{\textbf{Algorithm}}

    & \multicolumn{1}{>{\centering\arraybackslash}m{2.2cm}|}{\textbf{0}}

    & \multicolumn{1}{>{\centering\arraybackslash}m{1.5cm}|}{\textbf{(0,1]}}

    & \multicolumn{1}{>{\centering\arraybackslash}m{1.5cm}|}{\textbf{(1,2]}}

    & \multicolumn{1}{>{\centering\arraybackslash}m{1.5cm}|}{\textbf{(2,5]}}

    & \multicolumn{1}{>{\centering\arraybackslash}m{1.5cm}|}{\textbf{(5,11]}}\\

    \hline\hline

    PCA & 186 (93\%) & 3 (1.5\%) & 4 (2\%) & 2 (1\%) & 5 (2.5\%) \\\hline
    APCA & 174 (87\%) & 13 (6.5\%) & 7 (3.5\%) & 6 (3\%) & 0 \\\hline
    CA & 172 (86\%) & 16 (8\%) & 6 (3\%) & 6 (3\%) & 0 \\\hline
    FR & 171 (85.5\%) & 14 (7\%) & 8 (4\%) & 7 (3.5\%) & 0 \\\hline
    PWLH & 184 (92\%) & 13 (6.5\%) & 3 (1.5\%) & 0 & 0 \\\hline
    PWLHInt & 180 (90\%) & 8 (4\%) & 9 (4.5\%) & 3 (1.5\%) & 0 \\\hline
    GAMPS & 167 (83.5\%) & 15 (7.5\%) & 11 (5.5\%) & 3 (1.5\%) & 4 (2\%) \\\hline
    SF & 199 (99.5\%) & 1 (0.5\%) & 0 & 0 & 0 \\\hline\hline
    Total & 1,433 (89.5\%) & 83 (5.2\%) & 48 (3\%) & 27 (1.7\%) & 9 (0.6\%) \\\hline
    \toprule[0.1mm]

    \end{tabular}

    \caption{RD between the \ows and \lows variants of each CAI.\\The results are aggregated by algorithm and the range to which the RD belongs.}

    \label{tabla:windows-comparison}

\end{center}

\end{table}


\vspace{-5pt}


For example, let's analyze the results of algorithm CA, in the third row. The values in the first column indicate that the RD is equal to 0 between exactly 172 pairs (86\%) of the 200 evaluated pairs of CAIs for algorithm CA. The second column reveals that between 16 pairs of CAIs (8\%) the RD takes values greater than 0 and less than or equal to 1\%. The remaining three columns contemplate other ranges for the values taken by the relative difference. Notice that adding all of the values in the row gives a total of 200. This is true for every row except the last one, since we compare exactly 200 pairs of CAIs for each algorithm.


The values in the last row of Table~\ref{tabla:windows-comparison} consider the results for every pair of CAIs and are obtained by adding the values of every one of the previous rows. We notice that in 88.7\% of the total number of evaluated pairs of CAIs, the RD is equal to 0. In this cases, in fact, the local and global windows coincide. In 97.8\% of the cases the RD is less than or equal to 2\%. This means that, for the vast majority of CAI pairs, either the global and local windows match or they yield roughly the same compression performance. This result suggests that we could fix the global window as the window size parameter without compromising the performance of the coding algorithm. This is relevant, since calculating the LOWS for a file is, in general, a computationally expensive process. Having a predefined global windows table would improve the compression technique.


We notice that there are only 6 cases (0.4\%) in which the RD falls in the range (5, 11], most of which (5 cases) involve the algorithm PCA. The maximum value taken by RD (10.68\%) is obtained for the data type ``VWC" of the file ``vwc\_1203.dat.csv" of the dataset \datasetsst, with the PCA algorithm and error parameter $e=15$. In Figure~\ref{fig:window-compare-1203} we made a red circle around the marker associated to that maximum value. In this case, the global window is 16 and the local window is 8. This suggests that the performance of the PCA algorithm is more sensible to the window size parameter than the rest of the algorithms.


The experimental results presented in this section suggest that, in general, the impact of using the global window instead of the local window on the compression performance of coding algorithms is rather small. In the following section, in which we compare the algorithms performance, we always use the global window.

