
\section{Comparison of Masking and Non-Masking Variants}
\label{secX:rendimiento-relativo}


In this section, we compare the compression performance of the masking and non-masking variants of each of the evaluated algorithms that admit both; we denote by $A$ this set of algorithm variants. Specifically, we compare:


\vspace{-5pt}
\begin{itemize}
    \item \textit{PCA-M} against \textit{PCA-NM}
    \item \textit{APCA-M} against \textit{APCA-NM}
    \item \textit{CA-M} against \textit{CA-NM}
    \item \textit{PWLH-M} against \textit{PWLH-NM}
    \item \textit{PWLHInt-M} against \textit{PWLHInt-NM} 
    \item \textit{GAMPSLimit-M} against \textit{GAMPSLimit-NM}.
\end{itemize}


\vspace{+3pt}
\begin{notation}
We denote by $a_\maskalgo$ and $a_\NOmaskalgo$ the masking and non-masking variants of an algorithm.
\end{notation}


For each algorithm and each threshold parameter, we compare the performance of the masking and non-masking variants of the algorithm. For the purpose of this comparison, we choose the most favorable window size for each variant, in the sense of the following definition.


\newcommand{\ows}{\textit{OWS}}
\begin{defcion}
The \textit{optimal window size (\ows)} of a coding algorithm $a \in A$ and a threshold parameter $e \in E$, for the data type $z$ of a certain dataset $d$ is given by
\begin{equation}
\label{eq:ows}
\ows(a, e, z, d) = \argmin_{w\ \in \ W} \biggl\{ \tasacompresion(c_{<a, w, e>}, z, d) \biggr\},
\end{equation}
where we break ties in favor of the smallest window size.
\end{defcion}


For each data type $z$ of each dataset $d$, and each coding algorithm $a \in A$ and threshold parameter $e \in E$, we calculate the relative difference between $c_{<a_\maskalgo, w_\maskalgo^{*}, e>}$ and $c_{<a_\NOmaskalgo, w_\NOmaskalgo^{*}, e>}$, as defined in~(\ref{eq:relative-difference-dataset}), where $w_\maskalgo^{*}=OWS(a_\maskalgo, e, z, d)$ and $w_\NOmaskalgo^{*}=OWS(a_\NOmaskalgo, e, z, d)$.


\vspace{+2pt}
As an example, in Figures~\ref{fig:diff-sst} and~\ref{fig:diff-tornado} we show the compression ratio and relative difference, as a function of the error threshold, obtained for two data types of two different datasets. Figure~\ref{fig:diff-sst} shows the results for the data type ``VWC" of the dataset \datasetsst, and Figure~\ref{fig:diff-tornado} shows the results for the data type ``Longitude" of the dataset \datasettornado. In Figure~\ref{fig:diff-sst} we observe a large relative difference favoring the masking variant for all tested algorithms. On the other hand, in Figure~\ref{fig:diff-tornado} we observe that the non-masking variant outperforms the masking variant for all algorithms. We notice, however, that the relative difference is very small in the latter case.


\clearpage

%%%%%%%%%%%%%%%%%%%%%%%%%%%%%%%%%%%%%
%%%%%%%%%%%%%%%% 3.2 %%%%%%%%%%%%%%%%
%%%%%%%%%%%%%%%%%%%%%%%%%%%%%%%%%%%%%
\newcommand{\threetwocommon}[4]{
In the relative difference plot for algorithm {#1} we made a {#2} circle around the marker with the {#3} value obtained for all of the tested CAIs ({#4}).
}
\newcommand{\threetwomost}{\threetwocommon{PCA}{red}{maximum}{50.60\%}}
\newcommand{\threetwoleast}{\threetwocommon{APCA}{blue}{minimum}{-0.29\%}}

\newcommand{\threetwosingle}[5]{
    \clearpage
    \begin{figure}
    \hspace{-90pt} % trim=left bottom right top
    \includegraphics[clip,trim=0 2.9cm 0 3.5cm,height=23.5cm]{appendices/1pdfs/{#1}}
    \hspace{+5pt}
    \caption{Compression ratio and relative difference plots for every pair of algorithm variants $a_\maskalgo, a_\NOmaskalgo \in A$, for the data type ``{#3}" of the dataset {#2}.{#4}}
    {#5}
    \end{figure}
}

%%%%%%%%%%%%%%%%%%%%%%%%%%%%%%%%%%%%%
%%%%%%%%%%%%%%%% 3.3 %%%%%%%%%%%%%%%%
%%%%%%%%%%%%%%%%%%%%%%%%%%%%%%%%%%%%%

\newcommand{\threethreecommon}[4]{
In the relative difference plot for algorithm {#1} we made a {#2} circle around the marker with the {#3} value obtained for all the tested CAIs ({#4}).
}

\newcommand{\threethreemost}{\threethreecommon{PCA}{red}{maximum}{10.68\%}}

\newcommand{\threethreesingle}[6]{
    \clearpage
    \begin{figure}
    \hspace{-100pt} % trim=left bottom right top
    \includegraphics[clip,trim=0 1.8cm 0 2.5cm,height=19.5cm]{appendices/2pdfs/{#1}}
    \hspace{+5pt}
    \caption{Global and local window sizes, and relative difference plots for every algorithm, for the data type ``{#2}" of the file ``{#3}" of the dataset {#4}.{#6}}
    {#5}
    \end{figure}
}

%%%%%%%%%%%%%%%%%%%%%%%%%%%%%%%%%%%%%
%%%%%%%%%%%%%%%% 3.4 %%%%%%%%%%%%%%%%
%%%%%%%%%%%%%%%%%%%%%%%%%%%%%%%%%%%%%

\newcommand{\threefoursingle}[5]{
    \clearpage
    \begin{figure}
    \hspace{-100pt} % trim=left bottom right top
    \includegraphics[clip,trim=0 1.8cm 0 2.5cm,height=19.5cm]{appendices/3pdfs/{#1}}
    \hspace{+5pt}
    \caption{Compression ratio and window parameter plots for every algorithm, for the data type ``{#3}" of the dataset {#2}.{#4}}
    {#5}
    \end{figure}
}

\threetwosingle{2-NOAA-SST-1}{\datasetsst}{VWC}{\threetwomost}{\label{fig:diff-sst}}
\threetwosingle{7-NOAA-SPC-tornado-2}{\datasettornado}{Longitude}{\threetwoleast}{\label{fig:diff-tornado}}

\clearpage

TODO: HACER LO MISMO QUE EN LA TABLE~\ref{tabla:windows-comparison}\\
Table~\ref{tabla:rendimiento-relativ-NM-M} summarizes the results obtained for each dataset when comparing the relative difference between every pair of algorithm variants $a_\maskalgo, a_\NOmaskalgo \in A$. The second column outlines the amount of gaps in each dataset. The third column displays the percentage of tested CAIs in which $a_\maskalgo$ performs better than $a_\NOmaskalgo$. The last column shows the range for the values taken by the relative difference for those tested CAIs.


\vspace{+5pt}

\begin{table}[h]
\begin{center}
    \begin{tabular}{| C{2.2cm} || C{2.5cm} | C{4.4cm} | C{3.0cm} |}
    \hline
      \multicolumn{1}{|>{\centering\arraybackslash}m{2.2cm}||}{\textbf{Dataset}} 
    & \multicolumn{1}{>{\centering\arraybackslash}m{2.5cm}|}{\textbf{Dataset Characterstic}} 
    & \multicolumn{1}{>{\centering\arraybackslash}m{4.4cm}|}{\textbf{Cases where masking outperforms non-masking variant (\%)}}
    & \multicolumn{1}{>{\centering\arraybackslash}m{3.0cm}|}{\textbf{RD (\%) Range}}\\
    \hline
    \datasetirkis   & Many gaps     & 100 & (0; 36.88]                    \\\hline
    \datasetsst     & Many gaps     & 100 & (0; \textcolor{red}{50.60}]  \\\hline
    \datasetadcp    & Many gaps     & 100 & (0; 17.35]                    \\\hline
    \datasetelnino  & Many gaps     & 100 & (0; 50.52]                    \\\hline
    \datasetsolar   & Few gaps      & 51  & [-0.25; 1.77]                 \\\hline
    \datasethail    & No gaps       & 0   & [-0.04; 0)                    \\\hline
    \datasettornado & No gaps       & 0   & [\textcolor{blue}{-0.29}; 0)   \\\hline
    \datasetwind    & No gaps       & 0   & [-0.12; 0)                    \\\hline
    \toprule[0.1mm]
    \end{tabular}
    \caption{Relative difference between the masking and non-masking variants of each algorithm. In the last column we highlight the maximum (red) and minimum (blue) values taken by RD.}
    \label{tabla:rendimiento-relativ-NM-M}
\end{center}
\end{table}

\vspace{-5pt}


For every algorithm, on datasets with many gaps the masking variant always produces the best result, while on gapless datasets the non-masking variant always achieves the best result. On the dataset Solar, which presents few gaps, each variant is better than the other on approximately one half of the combinations of tested CAIs.


Observing the last column of Table~\ref{tabla:rendimiento-relativ-NM-M}, we notice that in every case in which the non-masking variant performs best, the relative difference is close to zero. The minimum value it takes is -0.29\%, which is obtained for the data type ``Longitude" of the dataset \datasettornado, with the APCA algorithm and error parameter $e=30$. In Figure~\ref{fig:diff-tornado} we made a blue circle around the marker associated to that minimum value. On the other hand, we also notice that for the datasets in which the masking variant performs best, the relative difference reaches higher absolute values. The maximum value (50.60\%) is obtained for the data type ``VWC" of the dataset \datasetsst, with the PCA algorithm and error parameter $e=30$. In Figure~\ref{fig:diff-sst} we made a red circle around the marker associated to that maximum value.


The experimental results presented in this section suggest that if we were interested in compressing a dataset with many gaps, we would benefit from using the masking variant of an algorithm, $a_\maskalgo$. However, even if the dataset didn't have any gaps, the performance would not be significantly worse than that obtained by using the non-masking variant of the algorithm, $a_\NOmaskalgo$. Since the masking variant $a_\maskalgo$ is more robust and performs better in general, in the following sections we will focus on its study.
