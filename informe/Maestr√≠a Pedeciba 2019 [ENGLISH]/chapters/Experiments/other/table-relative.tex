
\begin{table}[h]
\begin{center}
    \begin{tabular}{| C{2.2cm} || C{2.5cm} | C{4.2cm} | C{3.0cm} |}
    \hline
      \multicolumn{1}{|>{\centering\arraybackslash}m{2.2cm}||}{\textbf{Dataset}} 
    & \multicolumn{1}{>{\centering\arraybackslash}m{2.5cm}|}{\textbf{Dataset Characterstic}} 
    & \multicolumn{1}{>{\centering\arraybackslash}m{4.2cm}|}{\textbf{Cases where masking outperforms non-masking mode (\%)}}
    & \multicolumn{1}{>{\centering\arraybackslash}m{3.0cm}|}{\textbf{RD Range}}\\
    \hline
    \datasetirkis   & Many gaps     & 100 & (0; 36.88]                    \\\hline
    \datasetsst     & Many gaps     & 100 & (0; \textcolor{red}{50.60}]  \\\hline
    \datasetadcp    & Many gaps     & 100 & (0; 17.35]                    \\\hline
    \datasetelnino  & Many gaps     & 100 & (0; 50.52]                    \\\hline
    \datasetsolar   & Few gaps      & 51  & [-0.25; 1.77]                 \\\hline
    \datasethail    & No gaps       & 0   & [-0.04; 0)                    \\\hline
    \datasettornado & No gaps       & 0   & [\textcolor{blue}{-0.29}; 0)   \\\hline
    \datasetwind    & No gaps       & 0   & [-0.12; 0)                    \\\hline
    \toprule[0.1mm]
    \end{tabular}
    \caption{Relative performance of the masking and non-masking variants of each algorithm. In the last column we highlight the maximum (red) and minimum (blue) values taken by RD.}
    \label{tabla:rendimiento-relativ-NM-M}
\end{center}
\end{table}
