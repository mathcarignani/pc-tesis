
\vspace{-15pt}
\section{Dataset ADCP}
\label{datasets:adcp}


Dataset ADCP \cite{dataset:sst1} consists of water current velocity measurements from moorings in the Pacific Ocean. This dataset is collected by the Tropical Atmosphere Ocean project (TAO), which was established in 1985 to study annual climate variations nears the equator \cite{dataset:tao}.


The dataset consists of readings from 3 moorings, which can be located in the map presented in \cite{dataset:sst1}. In each mooring, readings are made by a total of 63 \textit{acoustic Doppler current profilers (ADCPs)}, placed at different depths below the ocean. Each ADCP measures the eastward (UCUR), northward (VCUR), and upward (WCUR) components of the water current velocity. Therefore, there are 567 ($3\times63\times3$) data samples stored for each timestamp. The velocity is measured in m/s, with a precision of three significant figures. Every sample value is transformed to the integer domain by multiplying it by $10^3$.


In Table~\ref{datasets:table:adcp} we present relevant statistics of dataset ADCP. The data for each month is stored in a separate csv file, and every row in the table contains statistics of a different file. The first three columns show the total number of rows, columns and entries (i.e. number of rows times number of columns), respectively. The fourth column specifies the number of gaps, and the percentage over the number of entries. The last five columns show the minimum, maximum, median, mean, and standard deviation, of the sample values.



\begin{table}[h]
\vspace{+5pt}
\begin{center}
    \begin{tabular}{| C{1.18cm} || C{0.78cm} | C{0.78cm} |  C{1.18cm} |  C{2.2cm} | C{0.7cm} | C{0.88cm} | C{0.590cm} | C{0.9cm} | C{0.9cm} |}
    \hline
      \multicolumn{1}{|>{\centering\arraybackslash}m{1.18cm}||}{\fsi\textbf{Month}}
    & \multicolumn{1}{>{\centering\arraybackslash}m{0.78cm}|}{\fsi\textbf{Rows}} 
    & \multicolumn{1}{>{\centering\arraybackslash}m{0.78cm}|}{\fsi\textbf{Cols}} 
    & \multicolumn{1}{>{\centering\arraybackslash}m{1.18cm}|}{\fsi\textbf{Entries}}
    & \multicolumn{1}{>{\centering\arraybackslash}m{2.2cm}|}{\fsi\textbf{Gaps (\%)}}
    & \multicolumn{1}{>{\centering\arraybackslash}m{0.7cm}|}{\fsi\textbf{Min}}
    & \multicolumn{1}{>{\centering\arraybackslash}m{0.88cm}|}{\fsi\textbf{Max}}
    & \multicolumn{1}{>{\centering\arraybackslash}m{0.590cm}|}{\fsi\textbf{Mdn}}
    & \multicolumn{1}{>{\centering\arraybackslash}m{0.9cm}|}{\fsi\textbf{Mean}}
    & \multicolumn{1}{>{\centering\arraybackslash}m{0.9cm}|}{\fsi\textbf{SD}}\\
    \hline
%% SCRIPT OUTPUT BELOW HERE
01-2015 & 744 & 567 & 421,848 & 134,544 (31.9) & -805 & 1,394 & 6 & 113.7 & 273.1 \\\hline
02-2015 & 672 & 567 & 381,024 & 125,151 (32.8) & -761 & 1,822 & 7 & 154.5 & 318.3 \\\hline
03-2015 & 744 & 567 & 421,848 & 133,110 (31.6) & -870 & 2,094 & 13 & 185.5 & 367.4 \\\hline
%% SCRIPT OUTPUT ABOVE HERE
    \toprule[0.1mm]
    \end{tabular}
    \caption{\tableExplain{in m/s}{ADCP} \ignoredGaps}
    \label{datasets:table:adcp}
\end{center}
\end{table}



