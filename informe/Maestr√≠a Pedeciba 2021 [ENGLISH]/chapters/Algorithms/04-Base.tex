

\clearpage
\section{Algorithm Base}
\label{algo:base}
\newcommand{\codeColumn}{$\text{code\_column}$}
\newcommand{\decodeColumn}{$\text{decode\_column}$}

\vspace{-10pt}
Algorithm \textit{Base} is a trivial lossless coding algorithm that serves as a base ground for comparing the performance of the rest of the algorithms. It follows the general schema presented in Figure~\ref{pseudoCodeCommon}, with a specific coding routine shown in Figure~\ref{pseudoCoderBase}. This routine iterates through every entry of a column of a csv data file. Since algorithm Base only supports the \NOmaskalgo\ variant, these entries can be either the character ``N", which represents a gap in the data, or an integer value representing an actual data sample. Every column entry is encoded independently and using a fixed number of bits, which depends on the data type and the dataset. In practice, the number of bits used for encoding a data value ultimately depends on the range and accuracy of the sampling instrument used for acquiring and storing the data. A special integer, \nodata, is reserved for encoding a gap. The decoding routine is symmetric to the coding routine.




\beginAlgorithm
\onlyInput{\columnInput\\ \outAlgo{Base}}
\ForEach{\forEachEntryCoder}{
    \eIf{\ifNoDataCoder{\entry}}{
        $\valuev = \nodata$\\
    }
    {
        $\valuev = \entry$\\
    }
    \encodeSpecific{$\valuev$}\\
}
\EndAlgorithm{Coding}{\coderBase}{\label{pseudoCoderBase}}


