
\clearpage
\section{Dataset IRKIS}
\label{datasets:irkis}


Dataset IRKIS \cite{dataset:irkis} consists of soil moisture measurements from stations installed along the Dischma valley, in the municipality of Davos, Switzerland, in the period between October 2010 and October 2013. This information is particularly useful to researchers who study the role of soil moisture in relation with the snowpack runoff and catchment discharge in high alpine terrain \cite{dataset:irkis2}. 


The dataset corresponds to seven stations, labeled 1202, 1203, 1204, 1205, 222, 333, and SLF2, which can be located in the map presented in~\cite{dataset:irkis}. For each station, the dataset contains measures of soil moisture data at different depths below the surface, namely 10, 30, 50, 80 and 120 cm. For each depth, there are moisture measures coming from two sensors, labeled A and B in the dataset files. Thus, for each station, there are 10 different data samples stored for each timestamp. The sensors measure the \textit{Volumetric Water Content (VWC)}, which is equal to the ratio of water volume to soil volume. Therefore, higher VWC values indicate a more moist soil. The original sample values have a precision of three significant figures, so they are transformed to the integer domain by multiplying them by $10^3$. 


\newcommand{\commonTable}{The first three columns show the total number of rows, columns and entries (i.e. number of rows times number of columns), respectively. The fourth column specifies the number of gaps, and the percentage of gaps over the total number of entries. The last five columns show the minimum, maximum, median, mean, and standard deviation, of the sample values }


In Table~\ref{datasets:table:irkis} we present some statistics of dataset IRKIS. The data from each station is stored in a separate csv file, and every row in the table contains statistics of a different file. \commonTable (dimensionless).



\begin{table}[h]
\vspace{+5pt}
\begin{center}
    \begin{tabular}{| C{1.2cm} || C{1.2cm} | C{1cm} |  C{1.5cm} |  C{2.2cm} | C{0.7cm} | C{0.7cm} | C{0.7cm} | C{0.9cm} | C{0.9cm} |}
    \hline
      \multicolumn{1}{|>{\centering\arraybackslash}m{1.2cm}||}{\textbf{Station}}
    & \multicolumn{1}{>{\centering\arraybackslash}m{1.2cm}|}{\textbf{\#Rows}} 
    & \multicolumn{1}{>{\centering\arraybackslash}m{1cm}|}{\textbf{\#Cols}} 
    & \multicolumn{1}{>{\centering\arraybackslash}m{1.5cm}|}{\textbf{\#Entries}}
    & \multicolumn{1}{>{\centering\arraybackslash}m{2.2cm}|}{\textbf{\#Gaps (\%)}}
    & \multicolumn{1}{>{\centering\arraybackslash}m{0.7cm}|}{\textbf{Min}}
    & \multicolumn{1}{>{\centering\arraybackslash}m{0.7cm}|}{\textbf{Max}}
    & \multicolumn{1}{>{\centering\arraybackslash}m{0.7cm}|}{\textbf{Mdn}}
    & \multicolumn{1}{>{\centering\arraybackslash}m{0.9cm}|}{\textbf{Mean}}
    & \multicolumn{1}{>{\centering\arraybackslash}m{0.9cm}|}{\textbf{SD}}\\
    \hline
%% SCRIPT OUTPUT BELOW HERE
1202 & 26,305 & 10 & 263,050 & 125,190 (47.6) & 240 & 541 & 428 & 417.1 & \ \ 48.8 \\\hline
1203 & 26,305 & 10 & 263,050 & 200,051 (76.1) & 165 & 385 & 249 & 257.7 & \ \ 40.1 \\\hline
1204 & 26,305 & 10 & 263,050 & 178,657 (67.9) & 218 & 464 & 298 & 313.3 & \ \ 70.0 \\\hline
1205 & 26,305 & 10 & 263,050 & 224,287 (85.3) & 272 & 600 & 315 & 342.1 & \ \ 63.9 \\\hline
222 & 26,304 & 10 & 263,040 & \ \ 56,007 (21.3) & 128 & 562 & 426 & 384.2 & 119.7 \\\hline
333 & 26,088 & 10 & 260,880 & \ \ 37,520 (14.4) & \ \ 38 & 451 & 327 & 291.5 & \ \ 81.1 \\\hline
SLF2 & 26,305 & 10 & 263,050 & \ \ 43,049 (16.4) & 215 & 580 & 352 & 360.9 & \ \ 65.1 \\\hline
%% SCRIPT OUTPUT ABOVE HERE
    \toprule[0.1mm]
    \end{tabular}
    \caption{Number of rows, columns, entries and gaps (total and percentual), and minimum, maximum, median, and standard deviation, of the sample values (dimensionless), for each file of the dataset IRKIS. The gaps are ignored when calculating the median, mean and standard deviation.}
    \label{datasets:table:irkis}
\end{center}
\end{table}



