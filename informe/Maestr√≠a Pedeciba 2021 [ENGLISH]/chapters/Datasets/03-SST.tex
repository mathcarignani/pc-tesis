

\vspace{-10pt}
\section{Dataset SST}
\label{datasets:sst}
\newcommand{\TAODef}{This dataset is collected by the Tropical Atmosphere Ocean project (TAO), which was established in 1985 to study annual climate variations nears the equator \cite{dataset:tao}.}

Dataset SST \cite{dataset:sst1} consists of \textit{Sea Surface Temperature (SST)} measurements from buoys floating in the Pacific Ocean. \TAODef


The dataset consists of readings from 55 buoys, which can be located in the map presented in \cite{dataset:sst1}. For each timestamp, the dataset contains the SST measurement taken in each buoy. Thus, a total of 55 data samples are stored for each timestamp. The temperature is measured in degree Celsius (°C), with a precision of three significant figures. Every sample value is transformed to the integer domain by multiplying it by $10^3$.


\clearpage


In Table~\ref{datasets:table:sst} we present some statistics of dataset SST. The data for each month is stored in a separate csv file, and every row in the table contains statistics of a different file. The first three columns show the total number of rows, columns and entries (i.e. number of rows times number of columns), respectively. The fourth column specifies the number of gaps, and the percentage of gaps over the total number of entries. The last five columns show the minimum, maximum, median, mean, and standard deviation, of the sample values (in °C).



\begin{table}[h]
\vspace{+5pt}
\begin{center}
    \begin{tabular}{| C{1.2cm} || C{1.1cm} | C{1cm} |  C{1.5cm} |  C{2.02cm} | C{0.55cm} | C{0.98cm} | C{0.98cm} | C{1.25cm} | C{1.08cm} |}
    \hline
      \multicolumn{1}{|>{\centering\arraybackslash}m{1.2cm}||}{\textbf{Month}}
    & \multicolumn{1}{>{\centering\arraybackslash}m{1.1cm}|}{\textbf{\#Rows}} 
    & \multicolumn{1}{>{\centering\arraybackslash}m{1cm}|}{\textbf{\#Cols}} 
    & \multicolumn{1}{>{\centering\arraybackslash}m{1.5cm}|}{\textbf{\#Entries}}
    & \multicolumn{1}{>{\centering\arraybackslash}m{2.02cm}|}{\textbf{\#Gaps (\%)}}
    & \multicolumn{1}{>{\centering\arraybackslash}m{0.55cm}|}{\textbf{Min}}
    & \multicolumn{1}{>{\centering\arraybackslash}m{0.98cm}|}{\textbf{Max}}
    & \multicolumn{1}{>{\centering\arraybackslash}m{0.98cm}|}{\textbf{Mdn}}
    & \multicolumn{1}{>{\centering\arraybackslash}m{1.25cm}|}{\textbf{Mean}}
    & \multicolumn{1}{>{\centering\arraybackslash}m{1.08cm}|}{\textbf{SD}}\\
    \hline
%% SCRIPT OUTPUT BELOW HERE
01-2017 & 4,392 & 55 & 241,560 & 92,866 (38.4) & 3 & 32,367 & 27,221 & 27,326.9 & 2,406.0 \\\hline
02-2017 & 3,984 & 55 & 219,120 & 84,292 (38.5) & 3 & 31,277 & 27,477 & 27,677.3 & 1,953.1 \\\hline
03-2017 & 4,362 & 55 & 239,910 & 99,541 (41.5) & 3 & 31,809 & 28,070 & 28,050.9 & 1,760.4 \\\hline
%% SCRIPT OUTPUT ABOVE HERE
    \toprule[0.1mm]
    \end{tabular}
    \caption{Statistics of dataset SST for each month. The gaps are ignored when calculating the median, mean and standard deviation of the sample values.}
    \label{datasets:table:sst}
\end{center}
\end{table}



