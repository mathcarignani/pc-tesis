
In this chapter we present our experimental results. The main goal of our experiments is to analyze the performance of each of the coding algorithms presented in Chapter~\ref{algo}, by encoding the various datasets introduced in Chapter~\ref{datasets}. 

In Section~\ref{experiments:experiments} we describe our experimental setting and define the evaluated combinations of algorithms, their variants and parameter values, and the figures of merit used for comparison. 

In Section~\ref{secX:rendimiento-relativo} we compare the compression performance of the masking and non-masking variants for each coding algorithm. The results show that on datasets with few or no gaps the performance of both variants is roughly the same, while on datasets with many gaps the masking variant always performs better, in some cases with a significative difference. These results suggest that the masking variant is more robust and performs better in general. 

In Section~\ref{secX:windows} we analyze the extent to which the window size parameter \textbf{impacts/affects} the compression performance of the coding algorithms. We compress each dataset file, and compare the results obtained when using the optimal window size (i.e. the one that achieves the best compression) for each file, with the results obtained when using the optimal window size for the whole dataset. The results indicate that the \textbf{impact/effect} of using the optimal window size for the whole dataset instead of the optimal window size for each file is rather small. 

In Section~\ref{secX:codersmask} we compare the performance of the different coding algorithms among each other and with the general purpose compression algorithm gzip. Among the tested coding algorithms, for larger error thresholds APCA is the best algorithm for compressing every data type in our experimental data set, while for lower thresholds the recommended algorithms are PCA, APCA and FR, depending on the data type. If we also consider algorithm gzip, \textbf{there isn't an algorithm that is better for compressing every data type for any threshold value / for no threshold value there exists an algorithm that is better for compresing every data type}. Depending on the data type, the recommended algorithms are APCA and gzip for larger error thresholds, and PCA, APCA, FR and gzip for lower thresholds.

