

\section{Algorithms PWLH and PWLHInt}
\label{algo:pwlh}
\newcommand{\EncodeWindow}{EncodeWindow}


Algorithm \textit{\PWLHfull} \cite{coder:pwlh} is a linear model algorithm that supports lossless and near-lossless compression. It has a window size parameter ($w$) that establishes a maximum block size in which the data are separately encoded. For PWLH we define both variants, \maskalgo\ and \NOmaskalgo. We also define algorithm \textit{PWLHInt} by introducing minor design changes to algorithm PWLH. The description in the current section applies to both algorithms, except for the specific differences that are pointed out in Subsection~\ref{algo:pwhl:int}.


Since PWLH is a linear model algorithm, its encoding process involves the encoding of a sequence of line segments. In Figure~\ref{pseudoCoderPWLHM} we present the coding routine for variant \maskalgo. It consists of a loop that iterates over all column entries, which are always integer values (the gaps are encoded separately), parsing them into consecutive windows of variable size (up to a maximum size $\win$), such that all the points representing entries in a certain window satisfy the \textit{valid hull condition}, defined in the next paragraph. The samples in each window are encoded via the auxiliary routine \EncodeWindow, shown in Figure~\ref{pwlhWindowM}. This method computes the segment that minimizes the \textit{mean square error (MSE)} for all the points representing the entries in the window, then encodes said segment through its two endpoints, using 32 bits for encoding their y-coordinates as floats (recall that the coding algorithm encodes the timestamp column, with the respective x-coordinates, first). The valid hull condition satisfied by all the points representing entries in the window, guarantees that the decoded samples satisfy the per-sample absolute error constraint established by parameter \maxerror.

\clearpage




\beginAlgorithm
\onlyInput{\columnInput\\\outVarM{PWLH}\\\cInputThreshold\\\cInputWindowAPCA\\\inputTSCol}
\createWindowOne\\
\ForEach{\forEachEntryCoder}{
    \obtainTS\\
    \letPoint\\
    \If{\firstIteratCA}{
        \appendIncoming\\
        \continueAlgo\\
    }
    Let \PWLHSet\ be the set of points in the \extendedWin\\
    Let \hull\ be the convex hull of \PWLHSet\\
    Let \validHull\ be true iff \hull\ satisfies the valid hull condition, defined in \ref{def:validHull}, for \maxerror\\
    \If{\nott\,\validHull \orr \windowFull\ }{
        \encodeWindowPWLH\\
        \createWindowTwo\\
    }
    \appendIncoming\\
}
\If{\winNoEmpty}{
    \encodeLastWindowPWLH\\
}
\EndAlgorithmVariantM{Coding}{\coderPWLH}{\label{pseudoCoderPWLHM}}






\beginAlgorithm
\onlyInput{\windowInput\\\cOutputFile{PWLH}\\\cInputWindowAPCA\\\inputTSCol}
\encodeWindowSizee\\
Let \segment\ be the line segment that minimizes the MSE for the points representing the entries in $\window$\\
Let \sO\ and \sN\ be the y-coordinates of the endpoints of \segment\\
\codeFloat{\sO}\\
\codeFloat{\sN}
\EndPseudo{Auxiliary routine \EncodeWindowCA\ for algorithm \coderPWLH.}{\label{pwlhWindowM}}



\vspace{+5pt}
\begin{defcion}
\label{def:validHull}
Let \PWLHSet\ be a set of points, with $|\PWLHSet|>1$, and let \hull\ be the convex hull of \PWLHSet. \hull\ satisfies the \textit{valid hull condition} for a maximum error threshold \maxerror, iff there exists an edge line in the boundary of \hull, for which the maximum distance from any of the points in \hull\ to said edge line is less than or equal to $2\maxerror$.
\end{defcion}


In figures \ref{example:pwlhIntro1} and \ref{example:pwlhIntro2} we present an example that illustrates how the valid hull condition is checked in the coding routine. In Figure~\ref{example:pwlhIntro1}, set \PWLHSet, which is defined in \Line 6, contains the points representing the three entries in the current window, $\window$, plus the incoming point, \incomingP{4}. In this case, the convex hull of \PWLHSet, \hull, satisfies the valid hull condition for $\maxerror = 1$, since the maximum distance from any of its points to the upper edge line of its boundary is approximately 1.1, which is less than $2\maxerror = 2$. Therefore, the condition in \Line 8 evaluates to false (we assume that $|\window|<\win$), and the current column entry is added to $\window$ (\Line 13).


In the next step, presented in Figure~\ref{example:pwlhIntro2}, set \PWLHSet\ contains the points representing the four entries in $\window$, plus \incomingP{5}. In this case, convex hull \hull\ does not satisfy the valid hull condition, since for each of the three edge lines in its boundary, there exists a point in \hull\ such that its distance to the edge line is greater than~$2\maxerror$. Therefore, the condition in \Line 8 is satisfied, so $\window$ is encoded (\Line 9), emptied (\Line 10) and added the current column entry (\Line 13).


\vspace{+5pt}
\examplePWLH


% \vspace{+5pt}
We point out that calculating the segment that minimizes the MSE for a set of points is computational more expensive than checking the valid hull condition \cite{AnEva2013}. This is the reason why the valid hull condition is checked in every iteration, when deciding whether or not an entry should be added to the window, while the segment that minimizes the MSE is only computed for the points representing the entries in a complete window.


If $\window$ is not empty after executing the last iteration of the loop in the coding routine, it is encoded in \Line 16, via the auxiliary routine \EncodeLastWindowPWLH, shown in Figure~\ref{pwlhLastWindowM}. If $\window$ consists of a single sample, its size is encoded using $\logWinSize$ bits, and the sample value is encoded using \tobitexp. On the other hand, if $\window$ consists of multiple samples, it is encoded in the same way that the previous windows are, i.e. via the auxiliary routine \EncodeWindow, shown in Figure~\ref{pwlhWindowM}.





\beginAlgorithm
\onlyInput{\windowInput\\\outVarM{PWLH}\\\cInputWindowAPCA\\\inputTSCol}
\uIf{$|\window| == 1$}{
\encodeWindowSizeeOne\\
\encodeSpecific{the y-coordinate of the single point in \window}\\
}
\Else{
\encodeWindowPWLH
}
\EndAuxVarM{\EncodeLastWindowPWLH}{PWLH}{\label{pwlhLastWindowM}}




\clearpage


The decoding routine for variant \maskalgo\ is shown in Figure~\ref{pseudoDecoderPWLHM}. It consists of a loop that repeats until every entry in the column has been decoded, which occurs when condition in \Line 2 becomes false. Recall that the coding algorithm encodes the timestamp column first (\Line \gapLine\ in Figure~\ref{pseudoCodeCommon}), so the timestamps are known to the decoding routine (input \tscol). Each iteration of the loop starts with the decoding of the window size (\Line 3). If the window size is greater than 1, then the points representing the entries in the window are modeled by a line segment. In this case, the decoding routine decodes the float representation of the y-coordinates of the segment's endpoints (\Lines 5-6), obtains the timestamps corresponding to their x-coordinates (\Line 7), and calls the auxiliary routine \decodeSegment with those inputs (\Line 8). As we recall from Figure~\ref{pseudoDecoLinear}, routine \decodeSegment returns a list consisting of the sample values that are decoded from the segment, which are then written in the decoded CSV data file (\Lines 9-11). On the other hand, if the window size is equal to 1, a column-specific fixed number of bits is read, from which a value is decoded and written to the decoded file (\Lines 13-14).




\newcommand{\forEachEntryDecoded}{\textnormal{sample in \decodedSamplesTwo, \valuevP,}}

\beginAlgorithm
\onlyInput{\inVarM{PWLH}\\\doutputfile\\\cInputWindowAPCA\\\cInputColSize\\\inputTSCol}
\nEqZero\\
\While{\whileColumnNotDecoded}{
    \decodeWindowwSize\\
    \uIf{$\sizee > 1$}{
        \decodeFloat{\sO}\\
        \decodeFloat{\sN}\\
        Obtain timestamps for the endpoints of the segment, \tsO\ and \tsN, from \tscol\\
        $\decodedSamplesTwo =  \decodeSegment$(\tscol, \tsO, \tsN, \sO, \sN) // routine in Figure~\ref{pseudoDecoLinear}\\
        \ForEach{\forEachEntryDecoded}{
            Output $\valuev$ to \out
        }
    }
    \Else{
        \decodeSpecific{\valuev}\\
        Output $\valuev$ to \out
    }
    \addNToSize\\
}
\EndAlgorithmVariantM{Decoding}{\coderPWLH}{\label{pseudoDecoderPWLHM}}





%%%%%%%%%%%%%%%%%%%%%%%%%%%%%%%%%%%%%%%%%%%%%%%%%%%%%%%%%%%%%%%%%%%%%%%%%%%%%%%%%%%%
%%%%%%%%%%%%%%%%%%%%%%%%%%%%%%%%%%%%%%%%%%%%%%%%%%%%%%%%%%%%%%%%%%%%%%%%%%%%%%%%%%%%
%%%%%%%%%%%%%%%%%%%%%%%%%%%%%%%%%%%%%%%%%%%%%%%%%%%%%%%%%%%%%%%%%%%%%%%%%%%%%%%%%%%%


\clearpage


\subsection{Example}
\label{algo:pwlh:example}
\newcommand{\exampleIntro}[1]{\exampleIntroFirst{#1}. For this example we let the distance between adjacent timestamps be equal to 60, the error threshold parameter ($\maxerror$) equal to 1, and the maximum window size ($\win$) equal to 256.}


\exampleIntro{\ref{example:pwlh:1}}


Since there are only 12 samples to encode, no window in this example can reach the maximum size (256). Therefore, a window is only encoded when the convex hull, defined in \Line 6 of the coding routine, violates the valid hull condition in \Line 7. The first iteration is the only one in which the condition in \Line 5 evaluates to false, so the algorithm just adds the first sample to the window (\Line 13). Figure~\ref{example:pwlh:1} shows this step in the graph. Observe that, besides the sample values, the convex hull for the current window is also shown.


\vspace{+5pt}
\exampleStep{pwlh}{1}{\label{example:pwlh:1}}{1}


\newcommand{\widthh}{\textit{distance}}
The boundary of the convex hull that includes the second incoming point, \incomingP{2}, consists of a single line segment, where the maximum distance from either point to the line is zero, so the valid hull condition is satisfied, and the second sample is added to the window. This step is shown in Figure~\ref{example:pwlh:2}.


\exampleStep{pwlh}{2}{\label{example:pwlh:2}}{2}


\clearpage


The following three sample values are also equal to 1. The convex hull that includes these points still consists of a single line segment, so the three samples are also added to the window. This step is shown in Figure~\ref{example:pwlh:3}.


\exampleStep{pwlh}{3}{\label{example:pwlh:3}}{3}


In the 6th iteration, sample value 2 is considered. The updated convex hull, whose boundary now consists of three edges, is shown in Figure~\ref{example:pwlh:4}. In this case, the maximum distance between the upper edge in its boundary and any of its points is approximately $0.8$, which is less than $2\maxerror=2$. Therefore, the valid hull condition is still satisfied, and the sample is added to the window.


\vspace{+5pt}
\exampleStep{pwlh}{4}{\label{example:pwlh:4}}{4}


\clearpage


The following three iterations are similar to the previous one. In every case, the convex hull is updated, and, even though the maximum distance between the upper edge in its boundary and any of its points increases, it is never greater than $2\maxerror$, so the three samples are added to the window. These steps are shown in figures~\ref{example:pwlh:5}, \ref{example:pwlh:6} and \ref{example:pwlh:7}. 

\vspace{+5pt}
\exampleStepMany{pwlh}{5}{\label{example:pwlh:5}}{5}

\vspace{-15pt}
\exampleStepMany{pwlh}{6}{\label{example:pwlh:6}}{6}

\vspace{-15pt}
\exampleStepMany{pwlh}{7}{\label{example:pwlh:7}}{7}

\clearpage


Eventually, in the 10th iteration, sample value 2 is considered. In the updated convex hull, which is shown in Figure~\ref{example:pwlh:8}, for the first time the valid hull condition is not satisfied, i.e. \validHull\ in \Line 7 becomes false. Observe that, for each of the four edges in the boundary of the convex hull, there exists a point in the hull such that its distance to the edge line is greater than $2\maxerror$. 


\vspace{+5pt}
\exampleStep{pwlh}{8}{\label{example:pwlh:8}}{8}


Since the condition in \Line 8 becomes true, the window is encoded via the auxiliary routine \EncodeWindow\ (\Line 9), and the window is emptied and added sample value 2 (\Lines 10 and 13). Routine \EncodeWindow\ models the points representing entries in the window through the line segment that minimizes the MSE for said points (recall Figure~\ref{pwlhWindowM}). This segment and its two endpoints are shown in Figure~\ref{example:pwlh:9}. Encoding the window requires $\logWinSize = \logWinSizeOpt{256}=8$ bits for encoding its size (i.e. 9), and 32 bits for encoding each of the float values corresponding to the y-coordinates of the segment's endpoints.


\vspace{+5pt}
\exampleStep{pwlh}{9}{\label{example:pwlh:9}}{9}


\clearpage


In the last two iterations of the coding routine, which correspond to the last two samples, the valid hull condition is not violated. Therefore, after executing the last iteration, $\window$ includes three samples, which are encoded via the auxiliary routine \EncodeLastWindowPWLH\ (\Line 16). The associated segment and its two endpoints are shown in Figure~\ref{example:pwlh:10}. In this figure, we also display the values of the decoded samples, which are the values that the decoding routine, shown in Figure~\ref{pseudoDecoderPWLHM}, writes to the decoded CSV data file. 


\exampleStep{pwlh}{10}{\label{example:pwlh:10}}{10}


%%%%%%%%%%%%%%%%%%%%%%%%%%%%%%%%%%%%%%%%%%%%%%%%%%%%%%%%%%%%%%%%%%%%%%%%%%%%%%%%%%%%
%%%%%%%%%%%%%%%%%%%%%%%%%%%%%%%%%%%%%%%%%%%%%%%%%%%%%%%%%%%%%%%%%%%%%%%%%%%%%%%%%%%%
%%%%%%%%%%%%%%%%%%%%%%%%%%%%%%%%%%%%%%%%%%%%%%%%%%%%%%%%%%%%%%%%%%%%%%%%%%%%%%%%%%%%


\subsection{Differences Between Algorithms PWLH and PWLHInt}
\label{algo:pwhl:int}


Recall, from Section~\ref{datasets:over}, that, in the signals we are interested in compressing, the data samples are always represented as integers. However, algorithm PWLH encodes the y-coordinates of each line segment as floats, using 32 bits (\Lines 4-5 in Figure~\ref{pwlhWindowM}). Since these y-coordinates are data samples, the compression performance of algorithm PWLH can be improved if they are instead encoded as bounded integers, using a fixed number of bits that depends on the data type and the dataset. That is precisely the idea behind the design of algorithm PWLHInt.


Algorithm PWLHInt is defined by applying three changes to algorithm PWLH. First, we change \Lines 4-5 in the auxiliary routine \EncodeWindow, shown in Figure~\ref{pwlhWindowM}, so that both y-coordinates are rounded to the nearest integer, and then encoded using a column-specific fixed number of bits. \Liness 5-6 are modified accordingly in the decoding routine, shown in Figure~\ref{pseudoDecoderPWLHM}, this being the only change in the decoding routine. Secondly, we add a new constraint to the condition in \Line 8 of the coding routine, shown in Figure~\ref{pseudoCoderPWLHM}, to make sure that the (rounded) y-coordinates of both endpoints of the approximation segment belong to the range defined for the data type being encoded. Otherwise, the fixed number of bits used for encoding the coordinates may not be enough. Lastly, rounding a coordinate value to the nearest integer, before encoding it, could represent a deviation of as much as 0.5 from its original value, which may cause a decoding error greater than the threshold \maxerror. Therefore, the coding routine of algorithm PWLHInt operates with an adjusted maximum error threshold, $\maxerror' = \maxerror - 0.5$, to make sure that the per-sample absolute error between the decoded and the original signals is less than or equal to \maxerror.


In Section~\ref{secX:codersmask}, in which we evaluate the coding algorithms, the experimental results suggest that the compression performance of algorithm PWLHInt is superior to that of algorithm PWLH.


%%%%%%%%%%%%%%%%%%%%%%%%%%%%%%%%%%%%%%%%%%%%%%%%%%%%%%%%%%%%%%%%%%%%%%%%%%%%%%%%%%%%
%%%%%%%%%%%%%%%%%%%%%%%%%%%%%%%%%%%%%%%%%%%%%%%%%%%%%%%%%%%%%%%%%%%%%%%%%%%%%%%%%%%%
%%%%%%%%%%%%%%%%%%%%%%%%%%%%%%%%%%%%%%%%%%%%%%%%%%%%%%%%%%%%%%%%%%%%%%%%%%%%%%%%%%%%


\subsection{Non-Masking (\NOmaskalgo)\ Variant}
\label{algo:pwhl:nmvariant}


The coding and decoding routines for variant \NOmaskalgo\ of algorithms PWLH and PWLHInt are similar to their respective variant \maskalgo\ counterparts, the difference being that the former routines are able to handle both sample values and gaps. In the coding routine for variant \NOmaskalgo, a window may consist either of sample values or gaps, but it cannot include both. Therefore, a new constraint is added to the condition in \Line 8 of the coding routine, shown in Figure~\ref{pseudoCoderPWLHM}, so that a window is also encoded if the new entry is character \noData\ (gap in the data) and the other entries in the window are sample values, or vice versa. To encode a window that consists of gaps, algorithm PWLH uses $\logWinSize$ bits for encoding its size, and 32 bits for encoding the special float \nodatafloat, while algorithm PWLHInt also uses $\logWinSize$ bits for encoding its size, but it uses \tobitexp\ for encoding the special integer \nodata.

